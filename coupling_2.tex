\documentclass[10pt, oneside]{article}

\input ./combined_macros.sty

\newcommand{\Cl}{\mathrm{Cl}}
\newcommand{\Dens}{\mathrm{Dens}}

\addbibresource{Twist.bib}

\usepackage{pdflscape}
\usetikzlibrary{shapes.geometric, arrows, positioning}

\tikzstyle{s16} = [rectangle, rounded corners, minimum width=1.8cm, minimum height=1cm,text centered, draw=black,fill=red!30]
\tikzstyle{s16chiral} = [s16, dashed]
\tikzstyle{s8 } = [rectangle, rounded corners, minimum width=1.8cm, minimum height=1cm,text centered, draw=black,fill=orange!30]
\tikzstyle{s4} = [rectangle, rounded corners, minimum width=1.8cm, minimum height=1cm,text centered, draw=black,fill=yellow!30]
\tikzstyle{s2chiral} = [rectangle, dashed, rounded corners, minimum width=1.8cm, minimum height=1cm,text centered, draw=black,fill=green!30]
\tikzstyle{dimension} = [circle, text centered, text width=0.7cm, minimum height=0.7cm, draw=black]
\tikzstyle{arrow} = [thick,->,>=stealth]

\title{Coupling}
\author{Chris Elliott\and Pavel Safronov \and Brian Williams}

\date{\today}
\begin{document}

Fields and supercharges live in the following places.

\begin{table}[!h]
\centering
\begin{tabular}[!h]{c|c|c|c}
&6d&4d&3d \\
\hline
$A \in V_n \otimes \gg$& $V_6 \otimes \gg$ &$V_4 \otimes \gg$ & $V_3 \otimes \gg$ \\
$\lambda \in \Sigma \otimes \gg$ &$S_+ \otimes W_+ \otimes \gg$ & $(S_+ \oplus S_-) \otimes \gg$ & $S \otimes \gg$ \\
$\psi \in \Psi$ & $S_- \otimes U$ & $(S_+ \otimes R) \oplus (S_- \otimes R^*)$ & $S \otimes R$ \\
$\phi \in \Phi$ & $W_+ \otimes U$ & $R \oplus R^*$ & $R$ \\
$Q \in \Sigma$ & $S_+ \otimes W_+$ & $S_+ \oplus S_-$ & $S$ \\
$Q^* \in S^*$ & $S_-$ & $S_+ \oplus S_-$ & $S$
\end{tabular}
\end{table}

Here $\Phi$ is a representation of $\gg$ and $\Psi$ is a representation of $\so(n) \oplus \gg$.  To define the action, including the background field $Q$, we need the following structures, all $\gg$-equivariant. 
\begin{enumerate}
 \item A first-order differential operator $\sd \dd \colon \Psi \to \Psi^*$.
 \item An operator $\rho_\Psi \colon V_n \otimes \Psi^* \to \Psi$.
 \item A gamma pairing $\Gamma \colon \sym^2 \Psi \to V_n$.
 \item A moment map pairing $\mu \colon \sym^2 \Phi \to \gg$.
 \item An additional $\gg$-equivariant map $Y \colon \Sigma \otimes \Psi \to \Phi$, or equivalently, using $\mu$, either a map $Y_\mu\colon \Sigma \otimes \Psi \otimes \Phi \to \gg$ or $Y_\mu^* \colon \Sigma \otimes \Phi \to \gg \otimes \Psi^*$.
\end{enumerate}

With this data in mind, we define the action, and the coupling to supercharges $Q$, as follows.  I'll write $\eta$ for the pairing on $V_n \otimes \gg$ induced from the metric and the Killing form, and $\kappa$ for the Killing form alone.
\begin{align*}
S_{\mr{gauge}} &= \int \left\langle -\frac{1}{4} F_A \wedge \ast F_A + \frac{1}{2}(\lambda, \sd D_A \lambda)\right\rangle - (\d_A c, A^*) + ([\lambda, c], \lambda^*) + \frac{1}{2}([c, c], c^*)\\
S_{\mr{matter}} &= \int - \frac{1}{2}  (\d \phi \wedge * \d \phi) + \frac{1}{2} (\psi , \sd \dd \psi)\\
I_{\mr{couple}} &= \int g\big( \eta(A, \Gamma (\psi, \psi)) - \frac 12 \eta(A, \d_A \mu( \phi, \phi)) - 
\kappa \circ Y_\mu(\lambda,\psi,\phi)  +  (\psi^*, [c, \psi])_U + \\
&\qquad\qquad\qquad\qquad\qquad\qquad +  (\phi^*, [c, \phi])\big)  +  g^2 \left(\mu(\phi, \phi), \mu(\phi, \phi) \right) \\
I^{(1)}_{\mr{gauge}}(Q) &= \int (\Gamma(Q, \lambda), A^*) + \frac{1}{2}(\rho(F_A), \lambda^*) \\
I^{(1)}_{\mr{matter}}(Q) &= \int (\phi^*, Y(Q, \psi)) + (\psi^*, \rho(\d \phi) Q)\\
I^{(1)}_{\mr{couple}}(Q) &= \int g (\psi^*, \kappa \circ \rho_\Psi \circ (1 \otimes Y_\mu^*)(A, Q, \phi)) + \frac{1}{2} g (\lambda^*, Q \otimes \mu(\phi,\phi)) \\
I^{(2)}_{\mr{gauge}}(Q_1,Q_2) &= \int \frac{1}{4}(\Gamma(Q_1, Q_2), \Gamma(\lambda^*, \lambda^*)) - \frac{1}{2}(Q_1, \lambda^*)(Q_2, \lambda^*) - (\iota_{\Gamma(Q_1, Q_2)} A, c^*) \\
I^{(2)}_{\mr{matter}}(Q_1,Q_2) &= \int \frac{1}{4}(\Gamma(Q_1, Q_2) , \Gamma(\psi^*, \psi^*)).
\end{align*}

With all of this, along with an appropriate version of the ``$3\psi$-rule'' -- which should say something about the compatibility of these structures, along with the claim that the map from either $\wedge^2S_+$, $S_+ \otimes S_-$ or $\sym^2 S$ to $V_n$ is an isomorphism -- I speculate the supersymmetry calculation should work symmetrically.  The $3\psi$ rule should just be the following.

\begin{prop}\label{prop: new3psi}
Suppose $Q_1,Q_2 \in \Sigma$ and $Q^* \in S$. 
Then
\[
(Q_1, Q^*) Q_2 + (Q_2, Q^*)Q_1 = \rho(\Gamma(Q_1,Q_2)) Q^*
\]
\end{prop}





%\section{Supersymmetric coupling}

%We'll explain how to prove the following.
%\begin{prop}
%The following equations hold, for the supersymmetry action of the 6d $\mc N=(1,0)$ super Yang-Mills theory coupled to a hypermultiplet valued in a symplectic representation $U$.
%\begin{equation}\label{CMEcouple}
%\begin{array}{rrrrrr}
%\{S_{\rm BV} , I^{(1)}_{\rm couple}\} + \{I_{\rm couple}, I^{(1)}_{\rm gauge} + I^{(1)}_{\rm matter}\} + \{I_{\rm couple}, I^{(1)}_{\rm couple}\} & = & 0 \\ 
%\{I_{\rm couple}, I^{(2)}_{\rm gauge} + I^{(2)}_{\rm matter}\} + \d_{CE} I^{(1)}_{\rm couple} + \{I^{(1)}_{\rm gauge} + I^{(1)}_{\rm matter}, I^{(1)}_{\rm couple}\} + \frac{1}{2} \{I_{\rm couple}^{(1)}, I_{\rm couple}^{(1)}\} & = & 0 \\
%\{I_{\rm couple}^{(1)}, I^{(2)}\} & =& 0 \\
%\{I^{(2)}, I^{(2)}\} & =& 0
%\end{array}
%\end{equation}
%\end{prop}


\end{document}