\documentclass[10pt, oneside]{article}

\input ./combined_macros.sty

\addbibresource{Twist.bib}

\title{A Taxonomy of Twists of Supersymmetric Yang--Mills Theory -- Comments on Referee Report}
\author{Chris Elliott\and Pavel Safronov \and Brian Williams}

\date{\today}

\begin{document}
\maketitle

We are very grateful to the referee for their numerous helpful comments.  Please find below a summary of the changes we have made in response to the report.

\begin{enumerate}
 \item Page 8: Added an explanatory comment.
 \item Page 18: As the referee very reasonably points out, we only use the notion of a formal moduli problem to provide geometric motivation for the constructions formulated within the paper.  In fact, by the cited result of Lurie and Pridham, the datum of a formal moduli problem is equivalent to that of an $L_\infty$ algebra, and all the results that we provide are presented in the latter language, so that knowledge of $\infty$-categories and derived geometry are not needed in order to understand the paper.  We added a remark (Remark 1.1) to point this out to the reader, and to reassure the reader that a detailed understanding of this motivation is not needed to understand the results of the paper.

 \item Page 20 points 1 and 2: Some further explanation was added for the notation here, along with some additional orienting comments on the introduction of local $L_\infty$ algebras.
 \item Page 20 point 3: Added a remark, thank you.
 \item Page 27: Added an additional remark here to clarify the definition.  In addition, in order to emphasise the difference here with the notation we use for the $L_\infty$ algebra of BF fields, we altered our notation to refer to the space $\mathcal F$ of BRST fields, where $\mathcal F$ is intended to suggest ``fields'', instead of using $\mathcal M$ as in the BV context. \item Page 29: Used $1/j$ for the root of the canonical bundle instead of $1/m$.
 \item Page 32: Clarified the condition $p^*S_N = S_M$ with a remark and fixed a small typo in the definition.
 \item Page 34 point 1: The intended map here was the pullback along $\mr{Re} \colon V \to V_\RR$, corrected here.
 \item Page 34 point 2: The parameter $\eps$ in the given equivalence corresponds to $\d z$, where $z$ is a complex coordinate on the factor $\CC$.  Because this parameter has degree 1, it is formal and the issue indicated here does not arise.
 \item Page 35: Added some clarification to the notation here.
 \item Page 36: We have clarified notations throughout section 2.1 per the referees advice. 
 We have kept the statement of Proposition 2.2 the same. In the statement $Q_i$ are elements of the vector space $\Sigma$, whereas $\lambda$ is a section of the trivial vector bundle with fiber $\Sigma$.
 \item Page 37, point 1: We have added the clarifying adjective ``Euclidean''.
 \item Page 37, point 2: We have clarified when we are working with merely alternative algebras and when we require associativity.
 \item Page 38: We have mentioned that it is a version of the Fierz identity, thank you.
 \item Page 51: Thank you, reference added.
 \item Page 52: Thank you for the remark; we have removed the confusing sentence.
 \item Page 54: Added a reference for this claim to the relevant section of [ES].
 \item Page 56: This section was written confusingly, and we have corrected it.  Both $Q_+$ and $Q_-$ individually have stabilizers isomorphic to $\mr{SU}(4)$, and the intersection of these stabilizers is isomorphic to $\mr{SU}(3)$.
 \item Page 77: Thanks, we have corrected the confusing appearance of ``six-dimensional'' which refers to the spin group of the 6-dimensional vector space, rather than the dimension of the spin representation which is 4.
\end{enumerate}

 
\end{document}
