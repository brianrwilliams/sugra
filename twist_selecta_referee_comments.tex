\documentclass[10pt, oneside]{article}

\input ./combined_macros.sty

\addbibresource{Twist.bib}

\title{A Taxonomy of Twists of Supersymmetric Yang--Mills Theory -- Comments on Referee Report}
\author{Chris Elliott\and Pavel Safronov \and Brian Williams}

\date{\today}

\begin{document}
\maketitle

We are very grateful to the referee for their numerous helpful comments.  Please find below a summary of the changes we have made in response to the report.

\begin{enumerate}
 \item Page 8: Added an explanatory comment.
 \item Page 18: \chris{I propose that the best thing to do here would be to add a longish remark on $\infty$-categories with some references, but reassuring the reader that no knowledge of details is needed to understand the content of the paper.}
 \item Page 20 points 1 and 2: Some further explanation was added for the notation here.
 \item Page 20 point 3: Added a remark, thank you.
 \item Page 27: Added an additional remark here to clarify the definition here.
 \brian{Maybe the issue is also with us calling this a BRST 'theory'? Perhaps less conflicting will be to refer to this as 'BRST data'. What do you think?} \chris{I don't think that's the issue, but maybe we should use a letter other than $\mathcal M$ for the formal moduli problem here.  Like maybe $\mathcal F$ for fields?}
 \item Page 29: Used $1/j$ for the root of the canonical bundle instead of $1/m$.
 \item Page 32: Clarified the condition $p^*S_N = S_M$ with a remark and fixed a small typo in the definition. \brian{Todo, mobius example. Not sure what they want.... Todo, category of vector bundles, I don't see what the category should have to do with issues of compactness.} \chris{He has an arrow pointing at the statement $p^*S_M = S_N$, so maybe the issue is with the way we think about action functionals.  Maybe we should use different notation to indicate that the local functional really refers to a Lagrangian density, so there's no problem defining it on something noncompact (and no need to integrate).}
 \item Page 34 point 1: The intended map here was the pullback along $\mr{Re} \colon V \to V_\RR$, corrected here. \chris{Todo, respond to the other part.}
 \item Page 34 point 2: The parameter $\eps$ in the given equivalence corresponds to $\d z$, where $z$ is a complex coordinate on the factor $\CC$.  Because this parameter has degree 1, it is formal and the issue indicated here does not arise.
 \item Page 35: Added some clarification to the notation here.
 \item Page 36: Yes, $Q_i$ and $\lambda$ are sections of the same thing, clarified this point. \chris{rest to respond to}
 \item Page 37: Added a clarifying sentence to the beginning of Section 2.1 about Euclidean signature. \brian{Todo: non associative stuff.}
 \item Page 38:
 \item Page 51: Thank you, reference added.
 \item Page 52: \chris{I talked to Brian about this, I'm not totally sure that I understand the objection.  Maybe the point is that the twisted theory can still admit real forms that could be thought of as twists of a real form of the supersymmetric theory, even if the corresponding real form didn't include a square-zero supercharge?} 
 \item Page 54: Added a reference for this claim to the relevant section of [ES].
 \item Page 56: This section was written confusingly, and we have corrected it.  Both $Q_+$ and $Q_-$ individually have stabilizers isomorphic to $\mr{SU}(4)$, and the intersection of these stabilizers is isomorphic to $\mr{SU}(3)$.
\end{enumerate}

 
\end{document}
