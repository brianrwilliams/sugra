\documentclass[10pt, oneside]{article}

\input ./combined_macros.sty

\addbibresource{Twist.bib}

\title{A Taxonomy of Twists of Supersymmetric Yang--Mills Theory -- Comments on Referee Report}
\author{Chris Elliott\and Pavel Safronov \and Brian Williams}

\date{\today}

\begin{document}
\maketitle

We are very grateful to the referee for their numerous helpful comments.  Please find below a summary of the changes we have made in response to the report.

\begin{enumerate}
 \item Page 8: Added an explanatory comment.
 \item Page 18:
 \item Page 20 points 1 and 2: Some further explanation was added for the notation here.
 \item Page 20 point 3: Added a remark, thank you.
 \item Page 27: Added an additional remark here to clarify the definition here.
 \brian{Maybe the issue is also with us calling this a BRST 'theory'? Perhaps less conflicting will be to refer to this as 'BRST data'. What do you think?}
 \item Page 29: Used $1/j$ for the root of the canonical bundle instead of $1/m$.
 \item Page 32: Clarified the condition $p^*S_N = S_M$ with a remark and fixed a small typo in the definition. \brian{Todo, mobius example. Not sure what they want.... Todo, category of vector bundles, I don't see what the category should have to do with issues of compactness.}
 \item Page 34 point 1: The intended map here was the pullback along $\mr{Re} \colon V \to V_\RR$, corrected here. \chris{Todo, respond to the other part}
 \item Page 34 point 2:
 \item Page 35: Added some clarification to the notation here.
 \item Page 36: Yes, $Q_i$ and $\lambda$ are sections of the same thing, clarified this point. \chris{rest to respond to}
 \item Page 37: Added a clarifying sentence to the beginning of Section 2.1 about Euclidean signature. \brian{Todo: non associative stuff.}
 \item Page 38:
 \item Page 51: Thank you, reference added.
 \item Page 52:
 \item Page 54: Added a reference for this claim to the relevant section of [ES].
 \item Page 56: This section was written confusingly, and we have corrected it.  Both $Q_+$ and $Q_-$ individually have stabilizers isomorphic to $\mr{SU}(4)$, and the intersection of these stabilizers is isomorphic to $\mr{SU}(3)$.
\end{enumerate}

 
\end{document}
