\documentclass[12pt]{amsart}

\usepackage[colorlinks=true]{hyperref}
\usepackage{amssymb}
\usepackage[all]{xy}
\usepackage{slashed}
\usepackage{cleveref}

\newcommand{\dvol}{\mathrm{dVol}}
\newcommand{\sD}{\slashed{D}}
\newcommand{\g}{\mathfrak{g}}

\newtheorem{thm}{Theorem}[section]
\newtheorem{lm}[thm]{Lemma}

\textwidth=16.5cm
\oddsidemargin=0cm
\evensidemargin=0cm
\textheight=22cm
\topmargin=0cm

\binoppenalty=10000
\relpenalty=10000

\begin{document}

\section{}

\subsection{Sign convention}

Each field has two gradings: the ghost grading and the fermionic number. The braiding is defined so that $xy = (-1)^{(F(x) + |x|)(F(y)+|y|)} yx$.

\subsection{Useful formulas} The Clifford relation is
\[vw + wv = (v, w).\]

The relationship between Clifford multiplication and $\Gamma$ is
\[(v, \Gamma(Q_1, Q_2)) = \langle Q_1, \rho(v) Q_2\rangle.\]

For a one-form $\beta$ we have
\[\alpha\wedge \beta = (\ast\alpha, \beta)\dvol.\]

The map $\wedge^2 V\rightarrow \mathrm{Cl}(V)$ is given by
\[v_1\wedge v_2\mapsto v_1v_2 - \frac{1}{2}(v_1, v_2).\]

If $X$ is a two-form and $Q_1, Q_2$ are two spinors, we have
\[\Gamma(Q_1, \rho(X) Q_2) + \Gamma(Q_2, \rho(X) Q_1) = \iota_{\Gamma(Q_1, Q_2)} X.\]

Since we are in the Euclidean signature, we have
\[\ast\ast\alpha = (-1)^{k(n-k)} \alpha,\]
where $\alpha$ is a $k$-form.

\begin{thm}[3-$\psi$ rule]
\[\rho(\Gamma(Q_1, Q_2))Q_3 + \rho(\Gamma(Q_2, Q_3))Q_1 + \rho(\Gamma(Q_3, Q_1))Q_2 = 0.\]
Equivalently,
\[(\Gamma(-, Q_1), \Gamma(Q_2, Q_3)) + (\Gamma(-, Q_2), \Gamma(Q_3, Q_1)) + (\Gamma(-, Q_3), \Gamma(Q_1, Q_2)) = 0.\]
\label{thm:3psi}
\end{thm}

\subsection{Computation}
To simplify the notation, we drop the integral, the pairing on $\g$ and the volume form. We denote by $\langle-, -\rangle$ the spinorial pairing.

The BRST action is
\[S_{BRST} = \frac{1}{2} F_A\wedge \ast F_A - \langle\lambda, \sD_A\lambda\rangle.\]

The BV action is
\[S_{BV} = \frac{1}{2} F_A\wedge \ast F_A - \langle\lambda, \sD_A\lambda\rangle + d_A c \wedge A^* - \langle[\lambda, c], \lambda^*\rangle - \frac{1}{2}[c, c]c^*.\]

The supersymmetry action is given by
\begin{align*}
I^{(1)}(Q) &= -2\Gamma(Q, \lambda)\wedge A^* - \langle \rho(F_A) Q, \lambda^*\rangle \\
I^{(2)}(Q_1, Q_2) &= (\Gamma(Q_1, Q_2), \Gamma(\lambda^*, \lambda^*)) - \langle Q_1, \lambda^*\rangle \langle Q_2, \lambda^*\rangle - 2(\iota_{\Gamma(Q_1, Q_2)}A) c^*.
\end{align*}

We need to check
\begin{align}
\{S_{BV}, I^{(1)}\} &= 0 \label{eq:num1} \\
\{S_{BV}, I^{(2)}\} + d_{CE} I^{(1)} + \frac{1}{2}\{I^{(1)}, I^{(1)}\} &= 0 \label{eq:num2} \\
d_{CE} I^{(2)} + \{I^{(1)}, I^{(2)}\} &= 0. \label{eq:num3}
\end{align}

\subsection{Checking \eqref{eq:num1}}

\begin{lm}[Snygg]
Suppose $X$ is an adjoint-valued $p$-form and $Q$ a constant spinor. Then
\[\sD_A(\rho(X)Q) = \rho(d_A X) Q+ (-1)^{n(p+1)}\rho(\ast d_A \ast X) Q.\]
\label{lm:snygg}
\end{lm}

\begin{lm}
\[\{I^{(1)}(Q), S_{BV}\} = (\Gamma(Q, \lambda), \Gamma(\lambda, \lambda)).\]
\end{lm}
\begin{proof}
Let us split $S_{BV} = \sum_{i=1}^5 S_{BV}^i$ into individual summands.

The first term gives
\begin{align*}
-\frac{1}{2}\{I^{(1)}(Q), S_{BV}^1\} &= d_A \Gamma(Q, \lambda)\wedge \ast F_A\\
&= \Gamma(Q, \lambda)\wedge d_A \ast F_A \\
&= (-1)^{n-1} d_A\ast F_A\wedge \Gamma(Q, \lambda) \\
&= (-1)^{n-1} (\ast d_A \ast F_A, \Gamma(Q, \lambda)).
\end{align*}

The second term gives
\begin{align*}
-\frac{1}{2}\{I^{(1)}(Q), S_{BV}^2\} &= (\lambda, \rho(\Gamma(Q, \lambda))\lambda) -(\rho(F_A) Q, \sD_A\lambda) \\
&= (\Gamma(Q, \lambda), \Gamma(\lambda, \lambda)) +\langle\lambda, \sD_A(\rho(F_A) Q)\rangle \\
&= (\Gamma(Q, \lambda), \Gamma(\lambda, \lambda)) + (-1)^n \langle\lambda, \rho(\ast d_A\ast F_A) Q\rangle,
\end{align*}
where we have used Lemma \cref{lm:snygg} and the Bianchi identity in the last line.

The third term gives
\begin{align*}
-\frac{1}{2}\{I^{(1)}(Q), S_{BV}^3\} &= [\Gamma(Q, \lambda), c]\wedge A^* + \frac{1}{2}\langle\rho(d_A d_A c)Q, \lambda^*\rangle \\
&= \Gamma(Q, [\lambda, c])\wedge A^* + \frac{1}{2}\langle\rho([F_A, c]) Q, \lambda^*\rangle.
\end{align*}

The fourth term gives
\begin{align*}
-\frac{1}{2}\{I^{(1)}(Q), S_{BV}^4\} &= -\frac{1}{2}([\rho(F_A) Q, c], \lambda^*) - \Gamma(Q, [\lambda, c])\wedge A^*.
\end{align*}

The fifth term gives
\[\{I^{(1)}(Q), S_{BV}^5\} = 0.\]
\end{proof}

By the 3-$\psi$ rule (\cref{thm:3psi}) we get that $S_{BV}$ is supersymmetric.

\subsection{Checking \eqref{eq:num2}}

We have
\begin{align*}
\{I^{(1)}, I^{(1)}\}(Q_1, Q_2) &= -2\{I^{(1)}(Q_1), I^{(1)}(Q_2)\} \\
&= -4\langle \rho(d_A \Gamma(Q_1, \lambda))Q_2, \lambda^*\rangle - 4\Gamma(Q_2, \rho(F_A)Q_1)\wedge A^* + 1\leftrightarrow 2
\end{align*}

\[
(d_{CE} I^{(1)})(Q_1, Q_2) = 2L_{\Gamma(Q_1, Q_2)}(A)\wedge A^* +2\langle\Gamma(Q_1, Q_2).\lambda, \lambda^*\rangle + 2(\Gamma(Q_1, Q_2).c) c^*
\]

\begin{align*}
\{S_{BV}, I^{(2)}(Q_1, Q_2)\} &= 2(\Gamma(Q_1, Q_2), \Gamma(\lambda^*, \sD_A \lambda + [\lambda^*, c])) - \langle Q_1, \lambda^*\rangle\langle Q_2, \sD_A \lambda + [\lambda^*, c]\rangle \\
&+ 1\leftrightarrow 2 - 2\iota_{\Gamma(Q_1, Q_2)}(d_A c) c^* -2d_A\iota_{\Gamma(Q_1, Q_2)} A\wedge A^* +2\langle[\lambda, \iota_{\Gamma(Q_1, Q_2)}A], \lambda^*\rangle \\
&+2[\iota_{\Gamma(Q_1, Q_2)} A, c] c^*
\end{align*}

We must have
\begin{align*}
-&2\iota_{\Gamma(Q_1, Q_2)} F_A +2 L_{\Gamma(Q_1, Q_2)}(A) -2 d_A\iota_{\Gamma(Q_1, Q_2)} A = 0 \\
&2(\Gamma(Q_1, Q_2).c) - 2\iota_{\Gamma(Q_1, Q_2)}(d_A c) +2[\iota_{\Gamma(Q_1, Q_2)} A, c] = 0 \\
-&2\rho([A, \Gamma(Q_1, \lambda)])Q_2 -2\rho([A, \Gamma(Q_2, \lambda)])Q_1 + 2\rho(\Gamma(Q_1, Q_2))\rho(A)\lambda \\
&- Q_2\iota_{\Gamma(Q_1, \lambda)} A - Q_1\iota_{\Gamma(Q_2, \lambda)} A +2[\lambda, \iota_{\Gamma(Q_1, Q_2)} A] = 0 \\
-&2\rho(d \Gamma(Q_1, \lambda))Q_2 - 2\rho(d \Gamma(Q_2, \lambda))Q_1 + 2\Gamma(Q_1, Q_2).\lambda \\
&+2\rho(\Gamma(Q_1, Q_2))(\sD \lambda) -\langle Q_1, \sD \lambda\rangle Q_2 -\langle Q_2, \sD \lambda\rangle Q_1 = 0
\end{align*}

The first two equations are straightforward to check. The third and fourth equations are checked in the same way, so let's just check equation 3. We have
\begin{align*}
-&2\rho([A, \Gamma(Q_1, \lambda)]) Q_2 - 2\rho([A, \Gamma(Q_2, \lambda)])Q_1- Q_2\iota_{\Gamma(Q_1, \lambda)} A - Q_1\iota_{\Gamma(Q_2, \lambda)} A \\
&= -2\rho(A)\rho(\Gamma(Q_1, \lambda))Q_2 - 2\rho(A)\rho(\Gamma(Q_2, \lambda))Q_1 \\
&= 2\rho(A)\rho(\Gamma(Q_1, Q_2))\lambda,
\end{align*}
where we have applied the 3-$\psi$ rule in the last line.

The Clifford relation then gives
\[2\rho(A)\rho(\Gamma(Q_1, Q_2))\lambda + 2\rho(\Gamma(Q_1, Q_2)) \rho(A)\lambda = 2[\iota_{\Gamma(Q_1, Q_2)} A, \lambda]\]
which cancels the last term.

\end{document}
