\documentclass[12pt]{amsart}

\usepackage[colorlinks=true]{hyperref}
\usepackage{amssymb}
\usepackage[all]{xy}
\usepackage{slashed}
\usepackage{cleveref}

\newcommand{\dvol}{\mathrm{dVol}}
\newcommand{\sD}{\slashed{D}}
\newcommand{\g}{\mathfrak{g}}

\newtheorem{thm}{Theorem}[section]
\newtheorem{lm}[thm]{Lemma}

\textwidth=16.5cm
\oddsidemargin=0cm
\evensidemargin=0cm
\textheight=22cm
\topmargin=0cm

\binoppenalty=10000
\relpenalty=10000

\begin{document}

\section{}

\subsection{Useful formulas}

The relationship between Clifford multiplication and $\Gamma$ is
\[(v, \Gamma(Q_1, Q_2)) = \langle Q_1, \rho(v) Q_2\rangle.\]

For a one-form $\beta$ we have
\[\alpha\wedge \beta = (\ast\alpha, \beta)\dvol.\]

\subsection{Computation}
To simplify the notation, we drop the integral, the pairing on $\g$ and the volume form.

The BRST action is
\[S_{BRST} = \frac{1}{2} F_A\wedge \ast F_A - \langle\lambda, \sD_A\lambda\rangle.\]

The BV action is
\[S_{BV} = \frac{1}{2} F_A\wedge \ast F_A - \langle\lambda, \sD_A\lambda\rangle + (d_A c, A^*) - \langle[\lambda, c], \lambda^*\rangle + \frac{1}{2}([c, c], c^*).\]

The supersymmetry action is given by
\begin{align*}
I^{(1)}(Q) &= \Gamma(Q, \lambda)\wedge A^*\rangle + \frac{1}{2} \langle \rho(F_A) Q, \lambda^*\rangle \\
I^{(2)}(Q_1, Q_2) &= (\Gamma(Q_1, Q_2), \Gamma(\lambda^*, \lambda^*)) + \langle Q_1, \lambda^*\rangle \langle Q_2, \lambda^*\rangle
\end{align*}

We need to check
\begin{align}
\{S_{BV}, I^{(1)}\} &= 0 \label{eq:num1} \\
\{S_{BV}, I^{(2)}\} + d_{CE} I^{(1)} + \frac{1}{2}\{I^{(1)}, I^{(1)}\} &= 0 \label{eq:num2} \\
d_{CE} I^{(2)} + \{I^{(1)}, I^{(2)}\} &= 0. \label{eq:num3}
\end{align}

\subsection{Checking \eqref{eq:num1}}

\begin{lm}[Snygg]
Suppose $X$ is an adjoint-valued $p$-form and $Q$ a constant spinor. Then
\[\sD_A(\rho(X)Q) = \rho(d_A X) Q+ (-1)^{n(p+1)}\rho(\ast d_A \ast X) Q.\]
\label{lm:snygg}
\end{lm}

\begin{lm}
\[\{I^{(1)}(Q), S_{BV}\} = (\Gamma(Q, \lambda), \Gamma(\lambda, \lambda)).\]
\end{lm}
\begin{proof}
Let us split $S_{BV} = \sum_{i=1}^5 S_{BV}^i$ into individual summands.

The first term gives
\begin{align*}
\{I^{(1)}(Q), S_{BV}^1\} &= d_A \Gamma(Q, \lambda)\wedge \ast F_A\\
&= \Gamma(Q, \lambda)\wedge d_A \ast F_A \\
&= (-1)^{n-1} d_A\ast F_A\wedge \Gamma(Q, \lambda) \\
&= (-1)^{n-1} (\ast d_A \ast F_A, \Gamma(Q, \lambda)).
\end{align*}

The second term gives
\begin{align*}
\{I^{(1)}(Q), S_{BV}^2\} &= (\lambda, \rho(\Gamma(Q, \lambda))\lambda) - (\rho(F_A) Q, \sD_A\lambda) \\
&= (\Gamma(Q, \lambda), \Gamma(\lambda, \lambda)) + (\lambda, \sD_A(\rho(F_A) Q)) \\
&= (\Gamma(Q, \lambda), \Gamma(\lambda, \lambda)) + (-1)^n (\lambda, \rho(\ast d_A\ast F_A) Q),
\end{align*}
where we have used Lemma \cref{lm:snygg} and the Bianchi identity in the last line.

The third term gives
\begin{align*}
\{I^{(1)}(Q), S_{BV}^3\} &= [\Gamma(Q, \lambda), c]\wedge A^* + \frac{1}{2}(\rho(d_A d_A c)Q, \lambda^*) \\
&= \Gamma(Q, [\lambda, c])\wedge A^* + \frac{1}{2}(\rho([F_A, c]) Q, \lambda^*).
\end{align*}

The fourth term gives
\begin{align*}
\{I^{(1)}(Q), S_{BV}^4\} &= -\frac{1}{2}([\rho(F_A) Q, c], \lambda^*) - \Gamma(Q, [\lambda, c])\wedge A^*.
\end{align*}

The fifth term gives
\[\{I^{(1)}(Q), S_{BV}^5\} = 0.\]
\end{proof}

\subsection{Checking \eqref{eq:num2}}

We have
\begin{align*}
\{I^{(1)}, I^{(1)}\}(Q_1, Q_2) &= -2\{I^{(1)}(Q_1), I^{(1)}(Q_2)\} \\
&= -\langle \rho(d_A \Gamma(Q_1, \lambda))Q_2, \lambda^*\rangle - \Gamma(Q_2, \rho(F_A)Q_1)\wedge A^* + 1\leftrightarrow 2
\end{align*}

\[
(d_{CE} I^{(1)})(Q_1, Q_2) = -L_{\Gamma(Q_1, Q_2)}(A)\wedge A^* - \langle\Gamma(Q_1, Q_2).\lambda, \lambda^*\rangle - (\Gamma(Q_1, Q_2).c) c^*
\]

\[
\{S_{BV}, I^{(2)}(Q_1, Q_2)\} = 2(\Gamma(Q_1, Q_2), \Gamma(\lambda^*, \sD_A \lambda + [\lambda^*, c])) + (Q_1, \lambda^*)(Q_2, \sD_A \lambda + [\lambda^*, c]) + 1\leftrightarrow 2
\]

\textbf{Question}. For two spinors $Q_1, Q_2$ and a two-form $X$, is the following true?
\[\Gamma(Q_1, \rho(X) Q_2) + \Gamma(Q_2, \rho(X) Q_1) = \iota_{\Gamma(Q_1, Q_2)} X\]

\end{document}
