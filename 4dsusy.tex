\documentclass[10pt, oneside]{article}

\input ./combined_macros.sty

\newcommand{\Cl}{\mathrm{Cl}}
\newcommand{\Dens}{\mathrm{Dens}}

\addbibresource{Twist.bib}

\usepackage{pdflscape}
\usetikzlibrary{shapes.geometric, arrows, positioning}

\tikzstyle{s16} = [rectangle, rounded corners, minimum width=1.8cm, minimum height=1cm,text centered, draw=black,fill=red!30]
\tikzstyle{s16chiral} = [s16, dashed]
\tikzstyle{s8 } = [rectangle, rounded corners, minimum width=1.8cm, minimum height=1cm,text centered, draw=black,fill=orange!30]
\tikzstyle{s4} = [rectangle, rounded corners, minimum width=1.8cm, minimum height=1cm,text centered, draw=black,fill=yellow!30]
\tikzstyle{s2chiral} = [rectangle, dashed, rounded corners, minimum width=1.8cm, minimum height=1cm,text centered, draw=black,fill=green!30]
\tikzstyle{dimension} = [circle, text centered, text width=0.7cm, minimum height=0.7cm, draw=black]
\tikzstyle{arrow} = [thick,->,>=stealth]

\title{4d $\cN=1$}
\author{Chris Elliott\and Pavel Safronov \and Brian Williams}

\date{\today}

\begin{document}

\maketitle

\subsection{Theories of matter}
In this subsection, $S_+ / S_-$ denote the positive/negative irreducible spin representations of $\mf{so}(4; \CC)$. 
To define the chiral multiplet we fix a complex vector space $R$.
The spinorial pairing $(-,-) : S_\pm \otimes S_\pm \to \CC$ induces a pairing between $S_\pm \otimes R$ and $S_{\pm} \otimes R^*$ that we also denote by $(-,-)_R$. 

The field content for the 4-dimensional chiral multiplet with values in a complex vector space $R$ is:
\begin{itemize}
\item a pair of scalars $\phi \in C^\infty(\RR^4 ; R)$ and $\Bar{\phi} \in C^\infty(\RR^4 ; R^*)$.
\item a positive Weyl spinor $\psi_+ \in C^\infty(\RR^4 ; R \otimes S_+)$ and a negative Weyl spinor $\psi_- \in C^\infty(\RR^4 ; R^* \otimes S_-)$.
\end{itemize}

The space of BRST fields is 
\[
F = C^\infty(\RR^4 ; R \oplus R^* \oplus R \otimes S_+ \oplus R^* \otimes S_-) .
\]

The BRST action is
\[
S (\phi, \Bar{\phi}, \psi_\pm) = \int_{\RR^4}  - (\d \phi \wedge * \d \Bar{\phi})_R + (\psi_+ , \sd \dd \psi_-)_R .
\]

\subsection{Matter multiplets}

In some dimensions $n \leq 6$ there exists the following {\em matter} representations of supersymmetry. 

\begin{itemize}
\item Dimension $n=4$, with $\Sigma = S_+ \otimes R \oplus S_- \otimes R^*$, with $R$ a complex vector space.
This is called the $\cN=1$ chiral multiplet.
\item Dimension $n=6$, with $\Sigma = S_+ \oplus W_+$ (or $\Sigma = S_- \otimes W_-$), with $W_+$ (or $W_-$) a complex, symplectic representation. This is called the $\cN=(1,0)$ (or $\cN=(0,1)$) hyper multiplet. 
\end{itemize}

\begin{lem}
\label{lem: oneform}
Let $\alpha$ be any $1$-form on $V$, $\psi \in C^\infty(V, \Sigma)$,  and $Q \in \Sigma$. 
Then
\[
\d (Q, \psi) \wedge * \alpha = * (\rho(\alpha)Q, \sd \dd \psi) .
\]
\end{lem}
\begin{proof}
Let $V_\alpha$ denote the vector field corresponding to $\alpha$. 
Note that $\d(Q,\psi) \wedge * \alpha = * (Q, V_\alpha . \psi)$. 
The result then follows from invariance of $(-,-)$ and the formula $\sd \dd \psi = \rho(\d x_i) (\partial_i \psi)$. 
\end{proof}

\subsubsection{The chiral multiplet}

In this subsection, $S_+ / S_-$ denote the positive/negative irreducible spin representations of $\mf{so}(4; \CC)$. 

The action of the odd part of the $4d$ $\cN=1$ supersymmetry algebra on the $4$-dimensional matter theory is encoded by the following linear and quadratic functionals 
\begin{align*}
I^{(1)}(Q) & = \int \<\phi^*, (Q, \psi_+)\>_R + \<\Bar{\phi}^*, (Q, \psi_-)\>_R + \<\psi_+^*, \rho(\d \phi)Q \>_R +  \<\psi_-^*, \rho(\d \Bar{\phi})Q \>_R \\
I^{(2)}  (Q_1 \otimes Q_2) & = \int \frac{1}{2}(\Gamma(Q_1, Q_2), \Gamma(\psi_+^*, \psi_-^*)) - (Q_1, \psi_+^*)(Q_2, \psi_-^*) - \int (Q_2, \psi_+^*)(Q_1, \psi_-^*) 
\end{align*}
for $Q, Q_1, Q_2 \in S_+ \oplus S_-$

\begin{thm}
The functional $\fS = S + I^{(1)} + I^{(2)}$ satisfies the classical master equation
\begin{equation}\label{CMEchiral}
\d_{\rm Lie} \fS + \frac{1}{2} \{\fS, \fS\} = 0 .
\end{equation}
\end{thm}

Before proving the theorem, we decompose the classical master equation (\ref{CMEchiral}) into the following equations:
\begin{equation}\label{CMEchiral2}
\begin{array}{rrrrrr}
\{S , I^{(1)}\} & = & 0 \\ 
\{S, I^{(2)}\} + \d_{CE} I^{(1)} + \frac{1}{2} \{I^{(1)}, I^{(1)}\} & = & 0 \\
\d_{CE} I^{(2)} + \{I^{(1)}, I^{(2)}\} & =& 0 \\
\{I^{(2)}, I^{(2)}\} & =& 0
\end{array}
\end{equation}

The last equation is automatically satisfied since $I^{(2)}$ is independent of $\phi, \Bar{\phi}$, and $\psi_{\pm}$. 

The first equation in (\ref{CMEchiral2}) states that the classical action for the chiral multiplet is supersymmetric. 

\begin{lemma} 
One has $\{S, I^{(1)}\} (Q) = 0$ for all $Q \in S_+ + S_-$. 
\end{lemma}
\begin{proof}
The BV bracket involving terms in $S$ depending on $\phi,\Bar{\phi}$ is:
\begin{align*}
- \left\{\d \phi \wedge * \d \Bar{\phi} \; , \; I^{(1)} \right\}  (Q) & = \d (Q, \psi_+) \wedge * \d \Bar{\phi} + \d \phi \wedge * \d (Q_-, \psi_-) 
%\\
%& = \Gamma(\sd \dd \psi_+, Q) \wedge * \d \Bar{\phi} \pm \d \phi \wedge * \Gamma(\sd \dd \psi_-, Q_-) .
\end{align*}
The BV bracket involving terms in $S$ depending on $\psi_\pm$ is:
\begin{align*}
\left\{ (\psi_+ , \sd \dd \psi_-) \; , \; I^{(1)} \right\} (Q) & = (\rho(\d \phi) Q, \sd \dd \psi_-) + (\psi_+, \sd \dd (\rho(\d \Bar{\phi}) Q)) \\ & =  (\rho(\d \phi) Q, \sd \dd \psi_-) - (\sd \dd \psi_+, (\rho(\d \Bar{\phi}) Q) \\ & = \d \phi \wedge * \Gamma(\sd \dd \psi_-, Q) - \d \Bar{\phi} \wedge * \Gamma(\sd \dd \psi_+, Q)
\end{align*}
Adding the two terms up, we see that $\{S,I^{(1)}\}(Q) = 0$ by Lemma \ref{lem: oneform} as desired. 
\end{proof}

Next, we move on to the second equation in (\ref{CMEchiral2}). 

\begin{lemma} 
One has
\begin{equation}\label{CMEchiral3}
\{S, I^{(2)}\} + \d_{CE} I^{(1)} + \frac{1}{2} \{I^{(1)}, I^{(1)}\} = 0 .
\end{equation}
\end{lemma}
\begin{proof}
Evaluating the equation (\ref{CMEchiral3}) on $v_1,v_2 \mf{iso}(V)$ reduces to the claim that (??) defines a strict Lie action. 
Evaluating on $v \in \mf{iso}(V)$ and $Q \in S_+ \oplus S_-$, the claim reduces to the fact that $I^{(1)}$ is Poincar\'{e} invariant.
So, the only nontrivial term to check is the evaluation on $Q_1,Q_2 \in S_+ \oplus S_-$. 

The individual terms are:
\begin{align*}
\{I^{(1)}, I^{(1)}\}(Q_1, Q_2) = & - \phi^* (Q_1, \rho(\d \phi) Q_2) - \phi^* (Q_2, \rho(\d \phi)Q_1) \\ & - \Bar{\phi}^* (Q_1, \rho(\d \Bar{\phi}) Q_2) - \Bar{\phi}^* (Q_2, \rho(\d \Bar{\phi})Q_1) \\ &  -  (\psi_+^*, \rho(\d(Q_1,\psi_+))Q_2) - (\psi_+^*, \rho(\d(Q_2, \psi_+)) Q_1) \\ & - (\psi_-^*, \rho(\d(Q_1, \psi_-)) Q_2) - (\psi_-^*, \rho(\d(Q_2, \psi_-))Q_1) 
\end{align*}

\begin{align*}
(\d_{CE}I^{(1)})(Q_1,Q_2) = & - \phi^* L_{\Gamma(Q_1,Q_2)} (\phi) - \Bar{\phi}^* L_{\Gamma(Q_1,Q_2)} \Bar{\phi} \\ &  - (\psi_+^* , \Gamma(Q_1,Q_2) . \psi_+) - (\psi_-^*, \Gamma(Q_1,Q_2) . \psi_-)
\end{align*}

and
\begin{align*}
\{S, I^{(2)}(Q_1,Q_2)\} = & 
\end{align*}

We first collect all terms in Equation (\ref{CMEchiral3}) proportional to $\phi^*$:
\[
- \frac{1}{2} (Q_1, \rho(\d\phi) Q_2) - \frac{1}{2} (Q_2, \rho(\d \phi) Q_1) - L_{\Gamma(Q_1,Q_2)} \phi .
\]
By Proposition \ref{??}
\brian{ $v \wedge \Gamma(Q_1, Q_2) = (Q_1 , \rho(v) Q_2)$} we observe that the first two terms cancel with the third term. 
Similarly, all terms proportional to $\Bar{\phi}^*$ also vanish. 

Next, we collect all terms in Equation (\ref{CMEchiral3}) proportional to $\psi_+^*$:
\begin{align*}
& - \frac{1}{2} \rho(\d(Q_1,\psi_+))Q_2 - \frac{1}{2} \rho(\d(Q_2, \psi_+)) Q_1 - \Gamma(Q_1,Q_2) . \psi_+
\\ 
& \pm \frac{1}{2} \rho(\Gamma(Q_1,Q_2)) \sd \dd \psi_+ \mp \frac{1}{2} (Q_2, \sd \dd \psi_+)Q_1 \mp \frac{1}{2} (Q_1, \sd \dd \psi_+)Q_2
\end{align*}

\end{proof}

\begin{lemma}
\[\{I^{(1)}, I^{(2)}\}(Q_1, Q_2, Q_3) = 0\]
for every $Q_1, Q_2, Q_3\in S_+ \oplus S_-$.
\end{lemma}
\begin{proof}
We have
\begin{align*}
\{I^{(1)}(Q_1), I^{(2)}(Q_2, Q_3)\} = 
\end{align*}

$\{I^{(1)}, I^{(2)}\}$ is obtained by cyclically symmetrizing the above expression. By Proposition \ref{prop:3psi} the cyclic symmetrization of the term with $c^*$ is zero. The Clifford relation implies that
%\begin{align*}
%\frac{1}{2} (\Gamma(Q_2, Q_3), \Gamma(\rho(A^*) Q_1, \lambda^*)) &= -\frac{1}{2}(\Gamma(Q_2, Q_3), \Gamma(\rho(A^*)\lambda^*, Q_1)) + (\Gamma(Q_2, Q_3), A^*) (Q_1, \lambda^*) \\
%&= -\frac{1}{2}(\rho(\Gamma(Q_2, Q_3)) Q_1, \rho(A^*)\lambda^*) + (\Gamma(Q_2, Q_3), A^*) (Q_1, \lambda^*).
%\end{align*}
%Therefore, again using Proposition \ref{prop:3psi} we see that the cyclic symmetrization of the terms with $A^*$ vanishes.
\end{proof}

\section{scrap}

The 4-dimensional $\mc N=1$ pure super Yang-Mills theory has BRST fields given by a ghost $c$, a 4d gauge field $A$, and a pair of opposite chirality Lie algebra valued Weyl spinor fields $\lambda_\pm$:
In addition, $4$-dimensional supersymmetry supports a chiral (or anti-chiral) matter multiplet, which comprises a complex scalar $\phi$ and a left (right) Weyl spinor $\psi_+$ ($\psi_-$).

The most general theory $4$-dimensional $\mc N = 1$ theory we will consider is super Yang-Mills theory valued in a Lie algebra $\fg$ minimally coupled to a chiral multiplet with values in a complex $\fg$-representation $R$. 
The field content for the BRST fields in the gauge sector is:
\[
(c, A, \lambda_\pm) \in (\Omega^0(\RR^4) \oplus \Omega^1(\RR^4) \oplus \Omega^0(\RR^4) \otimes \Pi S_\pm) \otimes \fg .
\]

For the chiral multiplet, the fields $\phi, \psi_+$ take values in $R$ and $\Bar{\phi}, \psi_-$ take values in $R^*$:
\[
(\phi, \psi_+) \in \Omega^0(\RR^4) \otimes (R \oplus R \otimes \Pi S_+) \;\;\; , \;\;\; (\Bar{\phi}, \psi_-) \in \Omega^0(\RR^4) \otimes (R^* \oplus R^* \otimes \Pi S_-) .
\]

Th full BV action for $\cN=1$ supersymmetric pure Yang-Mills theory on $\RR^4$ was defined in Section \ref{YM_section}, which we call $S_{\rm BV, gauge}$ in this section.

The kinetic part of the action functional for $4$-dimensional $\mc N = 1$ chiral multiplet is of the form
\[
S_{\rm matter} (\phi, \Bar{\phi}, \psi_{\pm}) = \int \d^4 x \; \<\d \phi , * \d \Bar{\phi}\>_R + \int \d^4 x \; \<\psi_+ , \sd \dd \psi_-\>_R .
\]
Here $\<-,-\>_R$ denotes the duality pairing between the representation $R$ and its dual $R^*$. 

There is also a term $S_{\rm couple}$ describing the coupling between Yang-Mills and the matter fields, which we record here:
\begin{align*}
S_{\rm couple} & = \pm g \int \<\psi_+, [\lambda_+, \phi]\>_R \pm g \int \<\psi_-, [\lambda_-, \Bar{\phi}]\>_{R^*} \\
& \pm g \int [A, \phi] \wedge * \d_{A} \Bar{\phi} \pm g \int [A, \Bar{\phi}] \wedge * \d_A \phi
\end{align*}
\brian{add terms involving antifields}
\brian{There is a term of the form $\pm g^2 \int \<(\phi \Bar{\phi}), * (\phi \Bar{\phi})\>_R$ that is usually written down.
This indeed changes the EOM, and is needed to compensate an extra term in the definition of $I_{\rm gauge}^{(1)}$ when we couple to matter. My claim is that if we disregard both of these terms we will get a consistent supersymmetric system still.
}

The action
\[
S_{\rm BV} = S_{BV}(\phi, \Bar{\phi}, \psi_\pm , A, \lambda_\pm, \; {\rm a.f's}) = S_{\rm BV, gauge}(A, \lambda_\pm, \; {\rm a.f's}) + S_{\rm matter}(\phi, \Bar{\phi}, \psi_\pm) + S_{\rm couple} (\phi, \Bar{\phi}, \psi_\pm , A, \lambda_\pm, \; {\rm a.f's}) 
\]
is the full BV action of $\mc N = 1$ super Yang-Mills coupled to matter.

The full BV action $S_{\rm BV}$ is clearly Poincar\'{e} invariant, so there is a functional $I_{\rm Poin}$ encoding the action by the $4$-dimensional Poincar\'{e} algebra.

The action of supersymmetry on the $4$-dimensional matter theory is encoded by a linear and quadratic functional:
\begin{align*}
I^{(1)}_{\rm matter} (Q_+ + Q_-) & = \int \<\phi^*, (Q_+, \psi_+)\>_R + \int \<\Bar{\phi}^*, (Q_-, \psi_-)\>_R + \int \<\psi^*_+, \rho(\d \phi) Q_- \>_R + \int \<\psi_-^*, \rho(\d \Bar{\phi}) Q_+\>_R \\
I^{(2)}_{\rm matter} (Q_+ \otimes Q_-) & = \int \<\psi^*_- , \rho(\Gamma(Q_+, Q_-)) \psi_+^*\>_R + \brian{write this differently?}.
\end{align*}
where $Q_\pm \in S_{\pm}$. 
As usual, we will use the notation $\delta_Q$ to denote the endomorphism on fields satisfying
\[
I^{(1)}_{\rm matter} = \int \<\phi^* + \Bar{\phi}^* + \psi^*_+ + \psi_-^*, \delta_Q(\phi + \Bar{\phi} + \psi_+ + \psi_-) \> .
\]

The action of supersymmetry on the gauge sector is encoded by the functionals:
\begin{align*}
I^{(1)}_{\rm gauge} (Q_+ + Q_-) & = \int \<A^* , \Gamma(Q_+ + Q_-, \lambda_+ + \lambda_-)\> + \<\lambda_+^*+\lambda_-^*, \sd F_A (Q_+ + Q_-)\> \\
I^{(2)}_{\rm gauge} (Q_+ \otimes Q_-) & = \int \left\<\lambda^*, \rho(\Gamma(Q_+, Q_-)) \lambda^* + \frac{1}{2} ((Q_+, \lambda_+^*+\lambda_-^*) Q_- + (Q_-, \lambda^*_+ + \lambda_-^*)Q_+)\right\> .
\end{align*}
\brian{C and P claim there should be some extra terms in $I^{(2)}_{\rm gauge}$ coming from reduction. 
If a few of those terms are of the form $\<\lambda_-^*, \lambda_-^*\>\<Q_-,Q_-\>$ and $\<\lambda_+^*, \lambda_+^*\>\<Q_+,Q_+\>$ we'd be in business. 
The issue is that in the twist calculation below has an extra copy of a complex involving components of $\lambda_-^*$ in degree zero and $\lambda_-$ in degree $+1$.
Such terms would turn this acyclic.
Can we see why such terms are {\em necessary} to preserve off-shell SUSY?
}

These functionals prescribe an off-shell action of $4$-dimensional $\cN=1$ super Yang-Mills coupled to matter, as we summarize in the following proposition.

%The result appears in \cite{SWchar}, and we do not repeat the proof here. 

\begin{prop}[\cite{SWchar}]
Let $I^{(1)} = I^{(1)}_{\rm matter} + I^{(1)}_{\rm gauge}$ and $I^{(2)} = I^{(2)}_{\rm matter} + I^{(2)}_{\rm gauge}$. 
Then, the functional
\[
\fS = S_{\rm BV} + I_{\rm Poin} + I^{(1)} + I^{(2)} \in \clie^\bu(\mf A) \otimes \cloc^\bu(\fL) [-1]
\]
satisfies the Maurer-Cartan equation
\[
\left(\d_{\rm Lie} \fS + \frac{1}{2} \{\fS, \fS\} \right) = 0 .
\]
\end{prop}

Notice that we have not introduced any auxiliary fields, which is at the cost of us formulating the action of superysmmetry as an $L_\infty$ action by the super Lie algebra of supertranslations. 


\begin{proof}
Throughout this proof we drop the pairing $\<-,-\>_R$ from the notation.
Also, we continue to denote the spinorial pairing by $\<-,-\>$ and the standard inner product by $(-,-)$. 

First, consider the pure matter sector $\fS_{\rm matter} = S_{\rm matter} + I_{\rm Poin} + I^{(1)}_{\rm matter} + I^{(2)}_{\rm matter}$. 
We will show that $\fS_{\rm matter}$ satisfies the classical master equation. 
From the form of $S_{\rm matter}$ and by Poincar\'{e} invariance of the matter action, it is clear that
\[
\d_{\rm Lie} S_{\rm matter} + \d_{\rm Lie} (I^{(1)}_{\rm matter} + I^{(2)}_{\rm matter}) + \frac{1}{2} \{S_{\rm matter} + I_{\rm Poin}, S_{\rm matter} + I_{\rm Poin}\} = 0 .
\]
So, we only need to consider the terms involving $\{S_{\rm matter}, I_{\rm matter}^{(1)}\}$, $\{S_{\rm matter}, I^{(2)}_{\rm matter}\}$, $\d_{\rm Lie} I_{\rm Poin}$, and $\{I^{(1)}_{\rm matter}, I^{(1)}_{\rm matter}\}$. 

The equality $\{S_{\rm matter}, I_{\rm matter}^{(1)}\} = 0$ simply says that $S_{\rm matter}$ is invariant under the usual linear supersymmetric action.
The argument for this is completely standard, see for instance \brian{refs}, but we repeat it here for completeness. 
The BV bracket involving terms in $S_{\rm matter}$ depending on $\phi,\Bar{\phi}$ is:
\begin{align*}
\frac{1}{2} \left\{ \int \d \phi \wedge * \d \Bar{\phi} \; , \; I_{\rm matter}^{(1)} \right\} & = \int \d (\<Q_+, \psi_+\>) \wedge * \d \Bar{\phi} + \int \d \phi \wedge * \d (\<Q_-, \psi_-\>) \\
& = \pm \int (\Gamma(\sd \dd \psi_+, Q_+), \d \Bar{\phi}) \pm \int (\d \phi, \Gamma(\sd \dd \psi_-, Q_-)) .
\end{align*}
The BV bracket involving terms in $S_{\rm matter}$ depending on $\psi_\pm$ is:
\begin{align*}
\frac{1}{2} \left\{\int \<\psi_+ , \sd \dd \psi_-\> \; , \; I_{\rm matter}^{(1)} \right\} & = \int \<\rho(\d \phi) Q_-, \sd \dd \psi_-\> = \int \<\psi_+, \sd \dd (\rho(\d \Bar{\phi}) Q_+)\> \\ & =  \int \<\rho(\d \phi) Q_-, \sd \dd \psi_-\> = \int \<\sd \dd \psi_+, (\rho(\d \Bar{\phi}) Q_+)\> \\ & = \int (\d \phi, \Gamma(\sd \dd \psi_-, Q_-)) + \int (\d \Bar{\phi}, \Gamma(\sd \dd \psi_+, Q_+)) 
\end{align*}
\brian{Get the signs right and these terms should cancel.}

The computation of the remaining terms in the matter sector are summarized in the following lemma. 
\begin{lem}
\[
\d_{CE} I_{\rm Poin} + \{S_{\rm matter}, I_{\rm matter}^{(2)}\} + \frac{1}{2} \{I_{\rm matter}^{(1)}, I_{\rm matter}^{(1)}\} = 0.
\]
\end{lem}
\brian{extra terms in $I^{(2)}$?}
\begin{proof}

First, we compute the BV bracket $\{I_{\rm matter}^{(1)}, I_{\rm matter}^{(1)}\}$:
\[
\{I^{(1)}_{\rm matter}, I^{(1)}_{\rm matter}\}(Q_+ + Q_-, Q_+' + Q_-') = 2 \int \left\<\phi^* + \Bar{\phi}^* + \psi^*_+ + \psi_-^*, [\delta_{Q_+ + Q_-}, \delta_{Q_+' + Q_-'}] \left(\phi + \Bar{\phi} + \psi_+ + \psi_-\right) \right\>  .
\]
We first focus on terms involving variations of $\phi, \Bar{\phi}$:
\begin{align*} 
[\delta_{Q_+ + Q_-}, \delta_{Q_+' + Q_-'}] (\phi + \Bar{\phi}) & = (\delta_{Q_+} + \delta_{Q_-}) \left(\<Q'_+, \psi_+\>_+ + \<Q'_-, \psi_-\>_-\right) - (\delta_{Q'_+} + \delta_{Q'_-}) \left(\<Q_+, \psi_+\>_+ + \<Q_-, \psi_-\>_-\right) \\ & = \<Q'_+, \rho(\d \phi) Q_-\>_+ + \<Q'_-, \rho(\d \Bar{\phi}) Q_+\>_- -  \<Q_+, \rho(\d \phi) Q'_-\>_+ - \<Q_-, \rho(\d \Bar{\phi}) Q'_+\>_- \\ & = (\d (\phi + \Bar{\phi}), \Gamma(Q_+ + Q_-, Q_+' + Q_-')) \\ & = L_{[Q_++Q_-, Q_+'+Q_-']} (\phi + \Bar{\phi}) %\\ & = \delta_{[Q_+ + Q_-, Q_+' + Q_-']} (\phi + \Bar{\phi})
\end{align*}
In the third line we have used the identity $(v, \Gamma(Q_1, Q_2)) = \<\rho(v) Q_1, Q_2\>$ where $Q_i \in S$, $v \in V$, $\<-,-\>$ is the spinor pairing, and $(-,-)$ is the inner product on $V$. 
The third line follows from the relation $(\d \phi, v) = L_v (\phi)$ where $L_v$ is the Lie derivative. 

Let $I_{\rm Poin}(\phi,\Bar{\phi})$ be the piece of $I_{\rm Poin}$ depending on $\phi, \Bar{\phi}$ and their antifields.
The last line above is simply the Lie derivative with respect to the translation invariant vector field $[Q_++Q_-, Q_+'+Q_-']$ on the field $\phi + \Bar{\phi}$, which is precisely the symmetry encoded by $I_{\rm Poin}$.
Thus, this calculation implies
\[
\left(\d_{CE} I_{\rm Poin}(\phi, \Bar{\phi}) + \frac{1}{2} \left\{I_{\rm matter}^{(1)}(\phi, \Bar{\phi}), I_{\rm matter}^{(1)}(\phi, \Bar{\phi})\right\} \right) (Q_+ + Q_- , Q_+' + Q_-') = 0
\]
for all $Q_\pm, Q_\pm'$. 

Next, we focus on the terms in the statement of the lemma involving the functionals which on the fields $\psi_{\pm}$: $I_{\rm Poin}(\psi_{\pm})$, $I^{(1)}_{\rm matter}(\psi_{\pm})$, and $I^{(2)}_{\rm matter}$. 
We must show
\[
\left(\d_{CE} I_{\rm Poin}(\psi_\pm) + \{S_{\rm matter}, I_{\rm matter}^{(2)}\} + \frac{1}{2} \{I_{\rm matter}^{(1)}(\psi_\pm), I_{\rm matter}^{(1)}(\psi_\pm)\}\right) (Q_+, Q_-, Q_+' + Q_-') = 0.
\]
for all $Q_\pm, Q_\pm'$.

We start by computing the variation of $\psi_+ + \psi_-$:
\begin{align*}
[\delta_{Q_+ + Q_-}, \delta_{Q_+' + Q_-'}] (\psi_+ + \psi_-) & = \brian{Fierz / 3 \psi} \\ & = \delta_{[Q_+ + Q_-, Q_+' + Q_-']} (\psi_+ + \psi_-) \pm \rho(\Gamma(Q_++Q_-, Q_+' + Q_-')) \cdot \sd \dd (\psi_+ + \psi_-) .
\end{align*}
\end{proof} 

Next, we move on to the pure gauge sector $\fS_{\rm gauge} = S_{\rm BV, gauge} + I_{\rm Poin} + I^{(1)}_{\rm gauge} + I^{(2)}_{\rm gauge}$. 
\end{proof}

\end{document}
