\documentclass[12pt]{amsart}

\usepackage{eucal}
\usepackage{amssymb}
\usepackage{stmaryrd}

\usepackage{color}
\definecolor{e-mail}{rgb}{0,.40,.80}
\definecolor{reference}{rgb}{.20,.60,.22}
\definecolor{citation}{rgb}{0,.40,.80}

\usepackage[colorlinks=true,
            linkcolor=reference,
            citecolor=citation,
            urlcolor=e-mail]{hyperref}
\usepackage{cleveref}
\usepackage[all]{xy}

\textwidth=16.5cm
\oddsidemargin=0cm
\evensidemargin=0cm
\textheight=22cm
\topmargin=0cm

\binoppenalty=10000
\relpenalty=10000

\newcommand{\C}{\mathbf{C}}
\renewcommand{\d}{\mathrm{d}}
\newcommand{\cD}{\mathcal{D}}
\newcommand{\cF}{\mathcal{F}}
\newcommand{\g}{\mathfrak{g}}
\newcommand{\cL}{\mathcal{L}}
\newcommand{\cO}{\mathcal{O}}
\renewcommand{\P}{\mathrm{P}}
\newcommand{\R}{\mathbf{R}}
\newcommand{\T}{\mathrm{T}}

\newcommand{\End}{\mathrm{End}}
\newcommand{\GL}{\mathrm{GL}}
\newcommand{\LGr}{\mathrm{LGr}}
\newcommand{\ML}{\mathrm{ML}}
\newcommand{\SL}{\mathrm{SL}}
\newcommand{\Spin}{\mathrm{Spin}}
\newcommand{\Sym}{\mathrm{Sym}}

\newtheorem{thm}{Theorem}[section]
\newtheorem{prop}[thm]{Proposition}
\newtheorem{lm}[thm]{Lemma}
\newtheorem{cor}[thm]{Corollary}
\theoremstyle{definition}
\newtheorem{defn}[thm]{Definition}
\theoremstyle{remark}
\newtheorem{remark}[thm]{Remark}
\newtheorem{example}[thm]{Example}

\newcommand{\defterm}[1]{\textbf{\emph{#1}}}

\begin{document}
\title{Pure spinor formalism}
\author{Pavel Safronov}
\maketitle

\section{Pure spinors}

Let $V=\C^{10}$ and $S_+, S_-$ be the semi-spin representations of $\Spin(V)$. We have a nondegenerate $\Spin(V)$-equivariant pairing $\Gamma\colon \Sym^2(S_+)\rightarrow V$.

For a vector space $L$ we denote by $\ML(L)$ the metalinear group, i.e. the $2:1$ cover of $\GL(L)$ given by the pullback
\[
\xymatrix{
\ML(L) \ar[r] \ar[d] & \GL(1) \ar^{z\mapsto z^2}[d] \\
\GL(L) \ar^{\det}[r] & \GL(1)
}
\]

\textbf{Fact}: the choice of a spin structure on $V$ endows any Lagrangian subspace $L\subset V$ with a metalinear structure, i.e. a choice of $\det(L)^{1/2}$.

\begin{prop}$ $
\begin{enumerate}
\item The group $\Spin(V)$ acts transitively on the set $\LGr(V)$ of Lagrangian subspaces $L\subset V$.

\item The stabilizer of a Lagrangian subspace $L\subset V$ is a parabolic subgroup $G_L\subset \Spin(V)$ which fits into an exact sequence
\[1\longrightarrow \wedge^2 L\longrightarrow G_L\longrightarrow \ML(L)\longrightarrow 1.\]
The choice of a Lagrangian complement $L^*\subset V$ to $L\subset V$ determines a splitting of this exact sequence, i.e. it gives an identification $G_L\cong \ML(L)\ltimes \wedge^2 L$.

\item Under the restriction $G_L\subset \Spin(V)$ the semi-spin representations split as
\[S_+ = (\C\oplus \wedge^2(L^*)\oplus \wedge^4(L^*))\otimes \det(L)^{1/2},\qquad S_- = (\C\oplus \wedge^2L\oplus \wedge^4 L)\otimes \det(L)^{-1/2}.\]
\end{enumerate}
\end{prop}

The tangent bundle $\T_{\LGr(V)}$ to $\LGr(V)$ is naturally $\Spin(V)$-equivariant. Its fiber at $L\in\LGr(V)$ is isomorphic to
\[\wedge^2(L\oplus L^*) / (\End(L)\oplus \wedge^2 L)\cong \wedge^2 L^*\]
as a $G_L$-representation (here $\wedge^2 L$ acts trivially). In particular, $\dim(\LGr(V)) = 10$.

We have $\det(\wedge^2 L^*) \cong \deg(L)^{-4}$. This representation is not $G_L$-invariant, so $\LGr(V)$ does not have a $\Spin(V)$-invariant Calabi--Yau structure.

\begin{prop}$ $
Let $\P$ be the set of nonzero elements $Q\in S_+$ satisfying $\Gamma(Q, Q) = 0$ and $\tilde{\P} = \P\cup \{0\}$.
\begin{itemize}
\item For $Q\in \P$ the image of $\Gamma(Q, -)\colon S_+\rightarrow V$ is a Lagrangian subspace. In particular, we have a projection $\P\rightarrow \LGr(V)$.

\item The natural action of $\C^\times$ on $\P$ by scaling gives $\P\rightarrow \LGr(V)$ the structure of a $\C^\times$-torsor. The fiber of $\P\rightarrow \LGr(V)$ at $L\subset V$ may be identified with nonzero elements $Q\in\det(L)^{1/2}$.
\end{itemize}
\end{prop}

The tangent bundle $\T_\P$ to $\P$ is naturally $\Spin(V)$-equivariant. Its fiber at $Q\in \P$ is isomorphic to
\[\wedge^2(L\oplus L^*) / (\End_0(L)\oplus \wedge^2 L)\cong \wedge^2 L^* \oplus \C\]
as a $\SL(L)\ltimes \wedge^2 L$-representation ($\wedge^2 L$ acts from the first to the second summand). In particular, $\det(\T_{\P, Q})\cong \det(L)^{-4}$ which is trivial as a $\SL(L)\ltimes \wedge^2 L$-representation. In particular, there is a unique $\Spin(V)$-invariant Calabi--Yau structure on $\P$.

Choose a point $Q\in P$. We can introduce a coordinate chart near $Q$ in the following way. We split
\[S_+ = (\C\oplus \wedge^2(L^*)\oplus \wedge^4(L^*))\otimes \det(L)^{1/2}.\]
Let $(\ell, A, M)\in S_+$ be components of a spinor with respect to this splitting. The pure spinor constraint is
\begin{align*}
\ell M + \Lambda\wedge \Lambda &= 0,\\
\langle \Lambda, M\rangle &= 0,
\end{align*}
where in the last line the pairing is $\wedge^2 L^*\otimes \wedge^4 L^*\rightarrow \det(L)^*\otimes L^*$.

In particular, in a neighborhood of $Q$ (i.e. in a neighborhood of $\Lambda=0$, $M=0$ and $\ell\neq 0$) the pair $(\ell, \Lambda)$ gives a coordinate chart. We may identify
\[\det(\wedge^2(L^*)\otimes \det(L)^{1/2})\cong \det(L),\]
so in this chart the unique $\Spin(V)$-invariant Calabi--Yau structure has the form
\[\Omega = \ell^{-3} \d \ell \d^{10} \Lambda.\]

\section{Pure spinor formulation of 10d SYM}

Let
\[T = \Pi\Sigma_+\oplus V\]
be the supertranslation Lie algebra and $G_T$ the supertranslation group. Then $C^\infty(G_T)$ carries two commuting $T$-actions given by left and right translations. For $\sigma\in S_+$ denote by $Q_\sigma$ and $\cD_\sigma$ the corresponding vector fields.

We assign the ghost number number $1$ and odd fermionic degree to coordinates on $\P$. The fields in our theory are
\[\cF = C^\infty(G_T)\otimes \cO(\P)\otimes \g[1].\]
The differential at $\sigma\in \P$ is given by $\cD_\sigma$. There is a residual supersymmetry action on $\cF$ given by $Q_\sigma$.

The differential $\cD$ can be split as $\cD^0 + \cD^1$, where $\cD^0$ is $\cD$ with $\Gamma=0$. The differential $\cD^0$ does not act on $C^\infty(V)$, so it just becomes an overall factor.

\begin{defn}
The \defterm{zero-mode cohomology} is the cohomology of $C^\infty(\Pi S)\otimes \cO(\P)$ with respect to $\cD^0$.
\end{defn}

\end{document}
