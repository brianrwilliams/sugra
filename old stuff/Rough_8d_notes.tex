\documentclass[10pt, oneside]{article}

\input ./combined_macros.sty

\newcommand{\Dens}{\mathrm{Dens}}

\addbibresource{Twist.bib}

\usepackage{pdflscape}
\usetikzlibrary{shapes.geometric, arrows, positioning}

\tikzstyle{s16} = [rectangle, rounded corners, minimum width=1.8cm, minimum height=1cm,text centered, draw=black,fill=red!30]
\tikzstyle{s16chiral} = [s16, dashed]
\tikzstyle{s8 } = [rectangle, rounded corners, minimum width=1.8cm, minimum height=1cm,text centered, draw=black,fill=orange!30]
\tikzstyle{s4} = [rectangle, rounded corners, minimum width=1.8cm, minimum height=1cm,text centered, draw=black,fill=yellow!30]
\tikzstyle{s2chiral} = [rectangle, dashed, rounded corners, minimum width=1.8cm, minimum height=1cm,text centered, draw=black,fill=green!30]
\tikzstyle{dimension} = [circle, text centered, text width=0.7cm, minimum height=0.7cm, draw=black]
\tikzstyle{arrow} = [thick,->,>=stealth]

\title{Twists of supersymmetric gauge theories}
\author{Chris Elliott\and Pavel Safronov \and Brian Williams}

\date{\today}

\begin{document}

As we break 10d SYM from $\spin(10)$ to further subgroups of the spin group, the super Yang-Mills fields decompose as follows:
\[
\xymatrix{
&\lambda \ar[dl]_{\to \Spin(8)} \ar[dr]^{\to \SU(5)} &\\
\lambda_+, \lambda_- \ar[dr]_{\to \SU(4)} \ar[d]_{\to \Spin(7)} &&B_{0,2}, \psi_{1,0}, \chi \ar[dl]^{\to \SU(4)} \\
(A_7^+, \wt \chi^+), (A_7^-, \wt \chi^-) \ar[dr] &(B_{0,2}^+, \psi_{0,1}^-), (\psi_{1,0}^+, \wt \chi), \chi \ar[d]_{\to \SU(3)} &\\
 &((\psi_{1,0}^+, \psi_{0,1}^+), (\psi_{0,1}^-, \chi^-)), ((\psi_{1,0}^-, \wt \chi^-), \wt \chi^+), \chi^+ &
}\]
for the fermions and
\[\xymatrix{
&A \ar[dl]_{\to \Spin(8)} \ar[dr]^{\to \SU(5)} & \\
A,\phi, \wt \phi \ar[dr]_{\to \SU(4)} \ar[d]_{\to \Spin(7)} &&A_{0,1}, A_{1,0} \ar[dl]^{\to \SU(4)} \\
\lambda_7, \phi, \wt \phi \ar[dr] &A_{0,1},A_{1,0},\phi, \wt \phi \ar[d]_{\to \SU(3)} & \\
&A_{0,1},A_{1,0},(\phi_1,  \phi_2),(\wt \phi_1, \wt \phi_2)&
}
\]
for the bosons.  Here, for the fermions, the $+,-$ superscripts indicate whether we came from $\lambda_+$ or $\lambda_-$ (and the way the fields decompose otherwise is indicated by the brackets in a way I hope is more-or-less clear).  

If we reduce the holomorphically twisted 10d theory to 8d, so restrict from $\SU(5)$ to $\SU(4)$, we can pick up a $\ZZ$-grading, where the fields coming from $\lambda_{\pm}$ have weight $\pm 1$, and where the surviving bosonic scalar $\phi$ has weight $-2$.  The components in the BV complex look like
\[\xymatrix{
&c \ar[r] &A_{0,1} \ar[r] &B_{0,2} \ar[r] &\psi_{0,1}^\vee \ar[r] &\phi^\vee \\
\phi \ar[r] &\psi_{0,1} \ar[r] &B_{0,2}^\vee \ar[r] &A_{0,1}^\vee \ar[r] &c^\vee &
}\]
in degrees $-2, \ldots, 3$.  Now, if we reduce further to $\SU(3)$ representations, this can be rewritten as 
\[\xymatrix{
&c \ar[r] \ar[dr] &A_{0,1} \ar[r] \ar[dr] &\psi^+_{1,0} \ar[r] \ar[dr] &\chi^{-\vee} \ar[dr] & \\
&&\phi_2 \ar[r]  &\psi^+_{0,1} \ar[r]  &\psi^{-\vee}_{0,1} \ar[r] &\phi_1^\vee \\
\phi_1 \ar[r] \ar[dr] &\psi_{0,1}^- \ar[r] \ar[dr] &\psi^{+\vee}_{0,1} \ar[r] \ar[dr] &\phi_2^\vee \ar[dr] && \\
&\chi^- \ar[r]  &\psi^{+\vee}_{1,0} \ar[r]  &A_{0,1}^\vee \ar[r] &c^\vee&
}\]
Note that this looks like the shifted cotangent to the complex $(\Omega^{0,\bullet}(\CC^3) \otimes \Omega^\bullet(\RR^2))[1]$.

If we want to turn on the isomorphism between the two shifted copies of the complex, it looks like we need a pair of functionals that look something like
\begin{align*}
I^{(1)}_t &= t \int \langle A_{0,1}^\vee, \psi_{0,1}^- \rangle + \langle \phi_2^\vee, \chi^- \rangle \\
I^{(2)}_t &= t \int \langle c^\vee, \phi_1 \rangle + \dvol \Lambda (\psi_{1,0}^{+\vee} \wedge \psi_{0,1}^{+\vee}).
\end{align*}

%Let's now restrict the fields in the untwisted classical BV theory now, not to $\SU(4)$ but to $\Spin(7)$, and try to determine the action %of the impure twist using the fact that it is supposed to be compatible with $\Spin(7)$.  This theory should also be $\ZZ$-graded.  The %components in the BV complex look like
%\[\xymatrix{
%. 
%}\]
%again in degrees 

\end{document}
