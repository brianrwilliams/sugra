\documentclass[10pt, oneside]{article}
\input math_headers.sty

\title{6d $\cN = (1,0)}
\author{Brian}
\date{\today}

\def\brian{\textcolor{blue}{BW: }\textcolor{blue}}

\def\tensor{{\otimes}}

\begin{document}






The underlying supervector space for the $6d$ $\mc N = (1,0)$ supersymmetry algebra has the following description. 
\begin{enumerate}
\item The bosonic piece of the 6d $\mc N = (1,0)$ supersymmetry algebra is
\[
(\mf {so} (6, \CC) \times \mf {sl}(2, \CC)_R) \ltimes V
\]
where 
\begin{itemize}
\item $V = \CC^6$ is the fundamental representation of $\mf {so}(6,\CC)$.
\item $\mf {sl}(2, \CC)_R$ is the $R$-symmetry algebra.
It acts trivially on $V$. 
\end{itemize}
\item The fermionic piece of the algebra is 
\[
S_+ \otimes_\CC W
\]
where $S_+ = \CC^4$ is the chiral irreducible spin representation of $\so(6, \CC)$ and $W_R = \CC^2$ is the two-dimensional vector space. 
\end{enumerate}

The Lie bracket for the superalgebra is determined by the rules:
\begin{enumerate}
\item
There is a term in the Lie bracket for the fundamental action of $\mf {so}(6, \CC)$ on $V$ and the spinorial action of $\mf {so}(6,\CC)$ on $S_+$. 
\item 
There is a term in the Lie bracket given by the action of $\mf {sl} (2,\CC)$ on $S_+ \otimes_\CC W_R$ by rotating the $W_R = \CC^2$ factor. 
\item
The final term in the Lie bracket is defined by the composition
\[
\Gamma : (S_+ \otimes \CC^2) \otimes (S_+ \otimes \CC^2) = (S_+ \otimes S_+) \otimes (\CC^2 \otimes \CC^2) \xrightarrow{\wedge \otimes \omega_{std}} V 
\]
where we have used the isomorphism $\wedge^2 S_+ \cong V$ and the standard symplectic form on $\CC^2$. 
\end{enumerate}

\subsection{Pure gauge theory}

The field content for $6d$ $\mc N = (1,0)$ pure gauge theory consists of a gauge field
\[
A \in C^\infty(\RR^4, V) \tensor \mf g = \Omega^1(\RR^6, \mf g)
\]
and a spinor
\[
\psi \in C^\infty(\RR^4, S_+ \tensor W_R) \tensor \mf g .
\]
As a convention, we choose a basis for $W_R$ given by symbols $\{+, -\}$ and we accordingly decompose the spinor as $\psi = \psi_+ + \psi_-$ where $\psi_{\pm} \in C^\infty(\RR^4, S_+) \tensor \mf g$.
\brian{yikes, bad notation?}

\subsection{The hypermultiplet}

\subsection{Holomorphic language}

A complex structure on $\RR^6 \cong \CC^3$ determines an embedding $\mf {gl} (3, \CC) \hookrightarrow \mf {so} (6, \CC)$. 
With respect to this inclusion, the $\mf {so}(6,\CC)$ representations split as
\begin{align*}
V & = V^{1,0} \oplus V^{0,1} \\
S_+ & = \CC \oplus \wedge^2(V^{0,1})
\end{align*}
where $V^{1,0}$ is the fundamental representation of $\mf {gl}(3, \CC)$ and $V^{0,1}$ its complex conjugate. 

Accordingly, the fields of the pure gauge theory decompose as
\[
A = A^{1,0} + A^{0,1} \in \Omega^{1,0}(\CC^3, \mf g) \oplus \Omega^{0,1}(\CC^3, \mf g)
\]
and 
\[
\psi = \phi + \chi^{0,2} \in C^\infty(\CC^3, W_R) \tensor \mf g \oplus \Omega^{0,2}(\CC^3, W_R) \tensor \mf g .
\]
In our notation above, after we choose a basis $\{+,-\}$ for $W_R$ we can decompose the spinor further into $\phi_{\pm} \in C^\infty(\CC^3) \tensor \mf g$ and $\chi^{0,2}_{\pm} \in \Omega^{0,2}(\CC^3) \tensor \mf g$. 

\end{document}