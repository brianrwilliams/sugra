\documentclass[12pt]{amsart}

\usepackage[colorlinks=true]{hyperref}
\usepackage{amssymb}
\usepackage{eucal}
\usepackage[all]{xy}

\textwidth=16.5cm
\oddsidemargin=0cm
\evensidemargin=0cm
\textheight=22cm
\topmargin=0cm

\binoppenalty=10000
\relpenalty=10000

\newcommand{\B}{\mathrm{B}}
\newcommand{\cE}{\mathcal{E}}
\newcommand{\g}{\mathfrak{g}}
\newcommand{\cN}{\mathcal{N}}
\newcommand{\T}{\mathrm{T}}
\newcommand{\Z}{\mathbf{Z}}

\newcommand{\ch}{\mathrm{ch}}
\newcommand{\Dol}{\mathrm{Dol}}
\newcommand{\dR}{\mathrm{dR}}
\newcommand{\Map}{\mathrm{Map}}

\newcommand{\defterm}[1]{\textbf{\emph{#1}}}

\newtheorem{thm}{Theorem}[section]
\newtheorem{prop}[thm]{Propostiion}
\theoremstyle{definition}
\newtheorem{defn}[thm]{Definition}
\theoremstyle{remark}
\newtheorem{remark}[thm]{Remark}
\newtheorem{example}[thm]{Example}

\begin{document}
\title{Twists}
\author{Pavel Safronov}
\maketitle

Compactification:
\begin{itemize}
\item $\Map((N\times S^1)_{\dR}, X)$ compactifies to $\Map(N_{dR}, \T[-1] X)$.

\item If $\Sigma$ is a complex curve and $L$ a 1-manifold, $\Map(\Sigma, X)$ compactifies to $\Map(L_{\dR}, X)$.
\end{itemize}

If an anomaly is not listed, the theory is not anomalous.

\section{$d=2$}

To define an $\cN=(2, 0)$ $\sigma$-model need a complex manifold $X$ with a holomorphic vector bundle $\cE$. If $\cE=\T_X$, we can enhance it to $\cN=(2, 2)$ supersymmetry. Let $E\rightarrow X$ be the total space.

\begin{itemize}
\item The rank $(1, 0)$ twist on $\Sigma$ gives $\T^*[-1]\Map(\Sigma, E[-1])$. It only has a one-loop anomaly which is given by $\chi(\Sigma) c_1(E[-1])$ and $\ch_2(E[-1])$.

\item If $\cE=\T_X$, this is
\[\T^*[-1]\Map(\Sigma, \T[-1]X)\cong \Map(\Sigma_{\Dol}, \T^*[1] X).\]
It only has a one-loop anomaly given by $\chi(\Sigma) c_1(X)$.

\item The rank $(1, 1)$ A-twist gives $\T^*[-1]\Map(\Sigma, X)_{\dR}$.

\item The rank $(1, 1)$ B-twist gives $\Map(\Sigma_{\dR}, \T^*[1] X)$. The anomaly is the same as for the holomorphic twist.
\end{itemize}

To define an $\cN=(2, 2)$ Yang--Mills theory we need a compact group $K$ and a complex $K$-representation $V$. Let $G$ be the complexification of $K$.

\begin{itemize}
\item The rank $(1, 0)$ twist on $\Sigma$ gives
\[\T^*[-1]\Map(\Sigma, \T[-1][V/G])\cong \Map(\Sigma_{\Dol}, \T^*[1][V/G]).\]
It only has a one-loop anomaly given by $\chi(\Sigma) c_1([V/G])$.

\item The rank $(1, 1)$ A-twist gives $\T^*[-1]\Map(\Sigma, [V/G])_{\dR}$.

\item The rank $(1, 1)$ B-twist gives $\Map(\Sigma_{\dR}, \T^*[1] [V/G])$. The anomaly is the same as for the holomorphic twist
\end{itemize}

\section{$d=3$}

To define an $\cN=2$ Yang--Mills theory we need a compact group $K$ and a complex $K$-representation $V$. If $V$ is the adjoint representation, the supersymmetry is enhanced to $\cN=4$.

\begin{itemize}
\item The rank $1$ twist on $\Sigma\times L$ where $\Sigma$ is a CY curve and $L$ is a 1-manifold gives $\T^*[-1]\Map(\Sigma\times L_{\dR}, [V/G])$.

\item For $V=\g$ get
\[\T^*[-1]\Map(\Sigma\times L_{\dR}, \T[-1]\B G)\cong \Map(\Sigma_{\Dol}\times L_{\dR}, \T^*[2] \B G).\]

\item The rank $2$ A-twist gives $\T^*[-1]\Map(\Sigma\times L_{\dR}, \B G)_{\dR}$.

\item The rank $2$ B-twist on a 3-manifold $M$ gives $\Map(M_{\dR}, \T^*[2] \B G)$.
\end{itemize}

We can also add $\cN=4$ matter. For this pick a complex $K$-representation $W$. If $W=\g$ we obtain $\cN=8$ supersymmetry. Then:

\begin{itemize}
\item The rank $2$ A-twist gives $\T^*[-1]\Map(\Sigma\times L_{\dR}, [W/G])_{\dR}$.

\item The rank $2$ B-twist gives $\Map(M_{\dR}, \T^*[2] [W/G])$.
\end{itemize}

\section{$d=4$}

To define an $\cN=1$ Yang--Mills theory we need a compact group $K$ and a complex $K$-representation $V$. The theta-term is zero. If $V$ is the adjoint representation, the supersymmetry is enhanced to $\cN=2$. We consider a complex surface $M^4=\Sigma_1\times \Sigma_2$.

\begin{itemize}
\item The rank $(1, 0)$ twist gives $\T^*[-1]\Map(M^4, [V/G])$. It only has a one-loop anomaly which in flat space is given by $\ch_3([V/G])$.

\item For $V=\g$ get
\[\T^*[-1]\Map(M, \T[-1]\B G)\cong \T^*[-1]\Map((\Sigma_1)_{\Dol}\times \Sigma_2, \B G).\]
It only has a one-loop anomaly given by $\chi(\Sigma_1) \ch_2(\B G)$.

\item The rank $(2, 0)$ twist gives $\T^*[-1]\Map(M^4, \B G)_{\dR}$. It compactifies to the 3d A-twist.

\item The rank $(1, 1)$ twist gives $\T^*[-1]\Map((\Sigma_1)_{\dR}\times \Sigma_2, \B G)$. It compactifies to the 3d B-twist or the 3d holomorphic twist. The anomaly is the same as in the holomorphic twist.
\end{itemize}

We can again add $\cN=2$ matter given a complex $K$-representation $W$:
\begin{itemize}
\item The rank $(2, 0)$ twist gives $\T^*[-1]\Map(M^4, [W/G])_{\dR}$.

\item The rank $(1, 1)$ twist gives $\T^*[-1]\Map((\Sigma_1)_{\dR}\times \Sigma_2, [W/G])$.
\end{itemize}

In the case $W=\g$ (i.e. we have $\cN=4$)
\begin{itemize}
\item The rank $(1, 0)$ twist gives
\[\T^*[-1]\Map((\Sigma_1)_{\Dol}\times (\Sigma_2)_{\Dol}, \B G).\]

\item The rank $(1, 1)$ twist gives
\[\T^*[-1]\Map((\Sigma_1)_{\dR}\times (\Sigma_2)_{\Dol}, \B G).\]

\item The rank $(2, 0)$ twist gives
\[\T^*[-1]\Map((\Sigma_1)_{\Dol}\times \Sigma_2, \B G)_{\dR}.\]

\item The special rank $(2, 2)$ twist gives
\[\T^*[-1]\Map((\Sigma_1)_{\dR}\times (\Sigma_2)_{\dR}, \B G).\]

\item The generic rank $(2, 2)$ twist gives
\[\T^*[-1]\Map((\Sigma_1)_{\dR}\times \Sigma_2, \B G)_{\dR}.\]
\end{itemize}

\section{$d=5$}

We consider $M^5=M^4\times L$ where $M^4=\Sigma_1\times \Sigma_2$ is a complex surface and $L$ is a 1-manifold.

\begin{itemize}
\item The rank $1$ twist gives $\T^*[-1]\Map(M^4\times L_{\dR}, [W/G])$.

\item If $W=\g$ get $\T^*[-1]\Map(M\times L_{\dR}, \T[-1]\B G)$.

\item The rank $2$ topological twist gives $\T^*[-1]\Map(M^4\times L_{\dR}, \B G)_{\dR}$.

\item The rank $2$ partially topological twist gives $\T^*[-1]\Map((\Sigma_1)_{\dR}\times \Sigma_2\times L_{\dR}, \B G)$.

\item The rank $4$ twist gives $\Map(M^5_{\dR}, \B G)$. This is $\Z/2$-graded.
\end{itemize}

\section{$d=6$}

Let $M^6=M^4\times \Sigma$ be a complex 3-fold.

$\cN=(1, 0)$ Yang--Mills:
\begin{itemize}
\item The rank $(1, 0)$ twist gives $\T^*[-1]\Map(M^6, [W/G])$. It only has a one-loop anomaly. In flat spacetime it is given by $\ch_4([W/G])$.
\end{itemize}

$\cN=(1, 1)$ Yang--Mills:
\begin{itemize}
\item The rank $(1, 1)$ topological twist gives $\T^*[-1]\Map(M^6, \B G)_{\dR}$.

\item The rank $(1, 1)$ partially topological twist gives $\T^*[-1]\Map(M^4\times \Sigma_{\dR}, \B G)$. It only has a one-loop anomaly when $\chi(\Sigma)\neq 0$ and $c_1(M)\neq 0$ in which case it is $\ch_2(\B G)$.

\item The rank $(2, 2)$ twist gives $\Map(M^2\times (M^4)_{\dR}, \B G)$. This is $\Z/2$-graded. It has a one-loop anomaly given by $\chi(M^4) \ch_2(\B G)$.
\end{itemize}

\section{$d=7$}

\begin{itemize}
\item The topological twist is $\Map(M^6\times L_{\dR}, \B G)_{\dR}$.

\item The holomorphic rank 1 twist is $\T^*[-1]\Map(M^6\times L_{\dR}, \B G)$.

\item The rank 2 twist is $\Map(M^4\times (M^3)_{\dR}, \B G)$. This is $\Z/2$-graded.
\end{itemize}

\section{$d=8$}

\begin{itemize}
\item The twist by a pure spinor rank $(1, 0)$ supercharge gives $\T^*[-1]\Map(M^8, \B G)$.

\item The twist by an impure spinor rank $(1, 0)$ supercharge gives $\Map(M^8, \B G)_{\dR}$.

\item The twist by a rank $(1, 1)$ supercharge gives $\Map((M^6)\times (M^2)_{\dR}, \B G)$. This is $\Z/2$-graded. Its one-loop anomaly vanishes for instance when $M^2$ is flat. If $M^2$ is not flat while $M^6$ is, the one-loop anomaly is given by $\ch_4(\B G)$.
\end{itemize}

\section{$d=9$}

\begin{itemize}
\item The twist by a pure spinor supercharge gives $\Map(M^8\times L_{\dR}, \B G)$. This is $\Z/2$-graded.
\end{itemize}

\section{$d=10$}

\begin{itemize}
\item The rank $1$ twist gives $\Map(M^{10}, \B G)$. This is $\Z/2$-graded. Its one-loop anomaly in flat spacetime is $\ch_6(\B G)$.
\end{itemize}

\end{document}
