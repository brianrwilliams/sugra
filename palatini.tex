\section{The Palatini formulation}
\brian{I first want to recall the palatini action and then formulate it using BV-BRST.}
 
\subsection{The physical theory}
In this section we construct a perturbative BV theory modeling the first-order Palatini approach to $N=1$ supegravity in four dimensions. 

The na\"{i}ve fields in the first-order formalism are the following: 
\begin{align*}
e & \in \Omega^1(\RR^4 ; T \RR^4) \\
\omega & \in \Omega^1(\RR^4 ; \so(4)) \\
\psi & \in \Omega^1(\RR^4 ; \Pi S) .
\end{align*} 

The na\"{i}ve action (the one involving only the fields above) splits up into two parts: $S (e, \omega, \psi) = S_{\rm Pal}(e, \omega) + S_{\rm fer}(e, \omega, \psi)$ where $S_{\rm Pal}$ is the ordinary Palatini action from general relativity
\ben
S_{\rm Pal} (e, \omega) = \int \epsilon(F_\omega \wedge e \wedge e)
\een
and
\ben
S_{\rm fer}(e,\omega,\psi) = \int \<\psi, e \cdot \d_\omega \psi\> .
\een
We explain the notation in the functionals above. 
In the top line the $\wedge$ product denotes the usual product on forms together with the exterior tensor product in the respective tensor bundles that the forms take values in. 
The symbol $\epsilon$ denotes the composition $\tensor^4 \RR^4 \to \wedge^4 \RR^4 \cong \RR$. 
In the second line $e \wedge \d_\omega \psi$ denotes wedge product of forms combined with Clifford multiplication $T \RR^n \tensor S \to S$. 
The pairing in the second line is the canonical \brian{finish}. 

If we did not have the term $S_{\rm fer}$, the equations of motion for the variation of the field $\omega$ result in the condition that $\omega$ is the spin connection associated to the vielbien $e$. 
Plugging this in to the equation of motion for $e$ result in the usual Einstein field equations. 

Suppose now that we turn on the term $S_{\rm fer}$. 
Since $S_{\rm fer}$ involves a dependence on $\omega$, the equations of motion for variation of $\omega$ do not quite say that $\omega$ is the spin connection for $e$. 
Instead, it states that the following equation holds
\ben
\omega = \omega(e) + K(\psi)
\een
where $K(\psi)$ depends quadratically on the field $\psi$ is the spin 
Explicitly, the function $K$ is of the form
\ben
K(\psi) = ?? 
\een 
\brian{I can't seem to find a coordinate independent way of writing this, but the formula I found is
\ben
K(\psi)_{\mu \nu \rho} = -\frac{1}{4} (\Bar{\psi}_\mu \gamma_\rho \psi_\nu - \Bar{\psi}_\nu \gamma_\mu \psi_\rho + \Bar{\psi}_\rho \gamma_\nu \psi_\mu). 
\een
}
\brian{The remaining equations of motion should yield the Eienstein equations with torsion. Not sure of the best way to interpret $K$ as a torsion term.}

\subsection{Symmetries}

There are two types of symmetries present in the theory, bosonic and fermionic. 
The bosonic gauge Lie algebra is of the form
\ben
(f , \varphi) \in \Omega^0(\RR^4 ; \mathfrak{i}\fs\fo(4)) = \Omega^0(\RR^4 ; \fs\fo(4)) \ltimes \Omega^0(\RR^4 ; \RR^4) . 
\een 
Given such an element the action on fields is of the form
\ben
\begin{pmatrix}
e \\ \omega \\ \psi 
\end{pmatrix} 
\mapsto
\begin{pmatrix}
e + L_\varphi e + [f,e] \\
\omega + L_\varphi \omega + \d_\omega f \\
\psi + L_\varphi \psi + [f,\psi] .
\end{pmatrix} .
\een
Here, $L_\varphi (-)$ denotes Lie derivative of the natural tensor and $\d_\omega = \d + \omega$ is the covariant derivative. 
We interpret this action as saying that the vector field $\varphi$ acts by infinitesimal diffeomorphisms, and the Lie algebra valued function $f$ acts by an ordinary gauge symmetry. 

There are also fermionic, or {\em super}, symmetries present. 
The fermionic factor of the gauge Lie algebra is of the form 
\ben
\chi \in \Omega^0(\RR^4 ; \Pi S) .
\een
The action of $\chi$ on the fields is of the form
\ben
\begin{pmatrix}
e \\ \omega \\ \psi 
\end{pmatrix} 
\mapsto
\begin{pmatrix}
e + \Gamma(\chi, e) \\
\omega \\
\psi + \d_\omega \chi
\end{pmatrix} .
\een 
Here $\Gamma$ is obtained from $\fs\so(4)$-invariant vector valued map $\Gamma_0 : S \tensor S \to \RR^4$ by tensoring with functions $C^\infty(\RR^4)$. 
It is a classical calculation to show that $S_{\rm Pal} + S_{\rm fer}$ is invariant under this supersymmetry, see Section 9.4 of \cite{freedman} for instance. 

\subsection{BV-BRST}

We now proceed to formulate the Palatini action in the BV-BRST formalism. 
We will work perturbatively from this point on, so it will be necessary to fix a solution to the classical equations of motion $(e_0,\omega_0,\psi_0)$. 

We will write down a local $L_\infty$ algebra on $\RR^4$ whose Maurer-Cartan elements coincide with the solutions to the classical equation of motion.
Closure of $L_\infty$ brackets under the the generalized Jacobi relations will encode the gauge symmetries in the previous section. 
The final piece of data is that of a (-3)-shifted symplectic structure on this $L_\infty$ algebra. 
This pairing allows us to encode the action functional described above. 
Note that this $L_\infty$ algebra will have two gradings: an internal cohomological grading that we will refer to as the ``degree'', and a $\ZZ/2$ grading that we will refer to as the ``super degree". 

The degree zero part of the $L_\infty$ algebra is the $\ZZ/2$ graded space of gauge symmetries $\Omega^0(\RR^4 ; \fs\fo(4)) \oplus \Omega^0(\RR^4 ; \RR^4) \oplus \Omega^0(\RR^4 ; \Pi S)$. 
The degree one part of the $L_\infty$ algebra is simply the physical space of fields $\Omega^1(\RR^4 ; \fs\fo(4)) \oplus \Omega^1(\RR^4 ; \RR^4) \oplus \Omega^1(\RR^4 ; \Pi S)$. 

\[\xymatrix{
   \Omega^0(X; \CC^4) \ar[r]^\d & \Omega^1(X; \CC^4) \ar[r]^{e_0 \wedge \d} &\Omega^3(X; \wedge^2 \CC^4) \ar[r]^\d &\Omega^4(X; \wedge^2 \CC^4) \\
   \Omega^0(X; \so(4;\CC)) \ar[r]^\d \ar[ur]^{\cdot  e_0} & \Omega^1(X; \so(4;\CC)) \ar[r]^{e_0 \wedge \d} \ar[ur]^{\cdot(e_0 \wedge e_0)} &\Omega^3(X; \wedge^3 \CC^4) \ar[r]^\d \ar[ur]^{e_0 \wedge} &\Omega^4(X; \wedge^3 \CC^4) \\
   \Pi\Omega^0(X; S) \ar[r]^\d & \Pi\Omega^1(X; S) \ar[r]^{\rho(e_0) \d} &\Pi\Omega^3(X; S) \ar[r]^\d &\Pi\Omega^4(X; S) \\
}\]

The first two rows are bosonic, and the last row is fermionic. 
The operator $\cdot e_0 : \Omega^0(\RR^4 ; \fs \fo(4)) \to \Omega^1(\RR^4 ; \CC^4)$ is given by acting on $e_0$. 
The dual operator $e_0 \wedge : \Omega^3(\RR^4 ; \wedge^3 \CC^4) \to \Omega^4(\RR^4 ; \wedge^2 \CC^4)$ first identifies $\wedge^3 \CC^4 \cong \CC^4$ using the metric and then wedges with $e_0$. 
The map $\cdot (e_0 \wedge e_0)$ is given by acting on $e_0 \wedge e_0$ via the second exterior power of the fundamental representation of $\fs \fo (4)$. 

