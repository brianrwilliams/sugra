A general theory of matter for $\cN = (1,0)$ supersymmetry arranges into a hypermultiplet which has the following BRST description.
Let $S_-^{6d}$ be the negative $\so(6)$ irreducible spin representation and suppose $R$ is a (complex) symplectic $\fg$-representation, with symplectic pairing labeled by $\<-,-\>_R$. 
The BRST fields of the $(1,0)$ hypermultiplet consist of a $W \otimes R$-valued scalar
\[
\phi \in \Omega^0(\RR^6) \otimes W \otimes R
\]
and an $R$ valued negative Weyl spinor
\[
\psi_- \in \Omega^0(\RR^6) \otimes S_-^{6d} \otimes R .
\]
Denote the induced symplectic pairing on $W \otimes R$ by $\<-,-\>_{W \otimes R}$. 
The action of the free hypermultiplet is given by
\[
S_{\rm matter} (\phi, \psi_-) = \frac{1}{2} \int_{\RR^6} \d^6 x \; \<\partial_{x_i} \phi , \partial_{x_i} \phi\>_{W \otimes R} + \frac{1}{2} \int_{\RR^6} \d^6 x \;\<\psi_-, \sd \dd \psi_-\>_R .
\]

The action of supersymmetry on the $6$-dimensional matter theory is encoded by a linear and quadratic functional:
\begin{align*}
I^{(1)} (Q) & = \int \<\phi^*, (Q, \psi_-)\>_{W \otimes R} + \int \<\psi^*_-, \rho(\d \phi) Q\>_R \\
I^{(2)} (Q_1 \otimes Q_2) & = \int \<\psi^*_- , \rho(\Gamma(Q_1, Q_2)) \psi_-\>_R 
\end{align*}
where we recall that the pairing is of the form $\Gamma : \Sigma^{\otimes 2} = (S_+ \otimes W)^{\otimes 2} \to V$. \brian{I'm also using $(S_+)^* = S_-$.}
These functionals provide an off-shell action of $\cN = (1,0)$ supersymmetry on the free hypermultiplet, as the following proposition shows.
Below, $\fL_{\rm matter}$ is the abelian local Lie algebra describing the free hypermultiplet, and $\mf A$ is the $(0,1)$ supertranslation algebra.  
As always, $I_{\rm Poin}$ is the action functional encoding ordinary Poincar\'{e} invariance. 

\begin{prop}[\cite{SWchar}]
The functional
\[
\fS_{\rm matter} = S_{\rm matter} + I_{\rm Poin} + I^{(1)} + I^{(2)} \in \clie^\bu(\mf A) \otimes \cloc^\bu(\fL_{\rm matter}) [-1]
\]
satisfies the Maurer-Cartan equation
\[
\left(\d_{\rm Lie} \fS_{\rm matter} + \frac{1}{2} \{\fS_{\rm matter}, \fS_{\rm matter}\} \right) = 0 .
\]
\end{prop}
\begin{proof}
\brian{Do this already! It's almost identical to the proof in 4d, just need to do a bit of translation.} 
\end{proof}

Next, we move on to the $6$-dimensional $\cN=(1,0)$ super Yang-Mills. 
The theory has BRST fields given by a ghost $c$, a $6$-dimensional gauge field $A$, and a Lie algebra valued spinor
\[
\lambda_+ \in \Omega^0(\RR^6) \otimes S_+^{6d} \otimes W \otimes \fg 
\]
where $W$ is a (complex) $2$-dimensional symplectic vector space. 
The tensor product $S_+^{6d} \otimes W$ forms the $8$-dimensional symplectic Weyl representation. 
The action of supersymmetry on the $6$-dimensional matter theory is encoded by a linear and quadratic functional:
\begin{align*}
I_{\rm gauge}^{(1)}(Q) & = \int \<A^*, \Gamma(Q, \lambda_+)\> + \<\lambda^*, \sd F_A Q\> \\
I_{\rm gauge}^{(2)}(Q_1 \otimes Q_2) & = \int \left\<\lambda^* \;,\; \rho(\Gamma(Q_1,Q_2)) \lambda^* + \frac 12 \left((Q_2, \lambda^*)Q_1 + (Q_1, \lambda^*)Q_2\right)\right\>  .
\end{align*}
These functionals define an off-shell action of the $\cN = (0,1)$ supersymmetry on the vector multiplet, which we proved above in Theorem \label{SUSY_YM_theorem}, and state below as a proposition. 
Below, $\fL_{\rm gauge}$ denotes the local Lie algebra describing super Yang-Mills, and $S_{\rm BV, gauge}$ is the full BV action. 

 implies the following proposition for $6d$
\begin{prop}[special case of Theorem \ref{SUSY_YM_theorem}]
The functional
\[
\fS_{\rm gauge} = S_{\rm BV, gauge} + I_{\rm Poin} + I_{\rm gauge}^{(1)} + I_{\rm gauge}^{(2)} \in \clie^\bu(\mf A) \otimes \cloc^\bu(\fL_{\rm gauge}) [-1]
\]
satisfies the Maurer-Cartan equation
\[
\left(\d_{\rm Lie} \fS_{\rm gauge} + \frac{1}{2} \{\fS_{\rm gauge}, \fS_{\rm gauge}\} \right) = 0 .
\]
\end{prop}
