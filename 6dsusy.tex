\documentclass[10pt, oneside]{article}

\input ./combined_macros.sty

\newcommand{\Cl}{\mathrm{Cl}}
\newcommand{\Dens}{\mathrm{Dens}}

\addbibresource{Twist.bib}

\usepackage{pdflscape}
\usetikzlibrary{shapes.geometric, arrows, positioning}

\tikzstyle{s16} = [rectangle, rounded corners, minimum width=1.8cm, minimum height=1cm,text centered, draw=black,fill=red!30]
\tikzstyle{s16chiral} = [s16, dashed]
\tikzstyle{s8 } = [rectangle, rounded corners, minimum width=1.8cm, minimum height=1cm,text centered, draw=black,fill=orange!30]
\tikzstyle{s4} = [rectangle, rounded corners, minimum width=1.8cm, minimum height=1cm,text centered, draw=black,fill=yellow!30]
\tikzstyle{s2chiral} = [rectangle, dashed, rounded corners, minimum width=1.8cm, minimum height=1cm,text centered, draw=black,fill=green!30]
\tikzstyle{dimension} = [circle, text centered, text width=0.7cm, minimum height=0.7cm, draw=black]
\tikzstyle{arrow} = [thick,->,>=stealth]

\title{4d $\cN=1$}
\author{Chris Elliott\and Pavel Safronov \and Brian Williams}

\date{\today}

\begin{document}

\maketitle

\subsection{The (half) hyper multiplet}

In this subsection, $S_+ / S_-$ denote the positive/negative irreducible spin representations of $\mf{so}(6; \CC)$. 
The spinorial representation we will be interested in is $\Sigma = S_+ \otimes W_+$, where $\dim_{\CC}(W_+) = 2$ is a two-dimensional multiplicity space. 
To define the half hyper multiplet we fix a complex, symplectic vector space $U$ whose symplectic form we call $\omega_U$.
We combine the symplectic forms $\omega_U, \omega_{W_+}$ to define a pairing 
\[
(-,-)_{W_+ \otimes U}  := \omega_{W_+} \otimes \omega_U : (W_+ \otimes U)^{\otimes 2} \to \CC .
\]
The spinorial pairing $(-,-) : S_+ \otimes S_- \to \CC$ together with the obvious symplectic form $\omega_{W_+}$ on $W_+$, and the symplectic form $\omega_U$ on $U$ induces a pairing (that we denote by the same symbol)
\[
(-,-)_U : (S_\pm \otimes U) \otimes (S_\mp \otimes U) \to \CC .
%(-,-)_U : (S_\pm \otimes W_+ \otimes U)^{\otimes 2} \to \CC  
\]

\begin{remark}
The terminology ``half" hyper multiplet is to be consistent with the terminology in the physics literature. 
The (full) hyper multiplet refers to a multiplet which depends on the choice of a (not necessarily symplectic) vector space $R$. 
The relationship between the two is obtained by looking at the cotangent space $U = T^*R$.
That is, the half hyper multiplet with values in the symplectic vector space $U = T^* R$ (with its obvious symplectic structure) is the (full) hyper multiplet for $R$. 
\end{remark}

The field content for the 6-dimensional half hyper multiplet with values in $U$ is:
\begin{itemize}
\item a scalar valued in $W_+ \otimes U$; $\phi \in C^\infty(\RR^4 ; W_+ \otimes U)$;
\item a negative Weyl spinor valued in $U$; $\psi \in C^\infty(\RR^4 ; S_- \otimes U)$ .
\end{itemize}

The space of BRST fields is thus
\[
F = C^\infty(\RR^4 ; W_+ \otimes U \oplus S_- \otimes U) .
\]

The BRST action is
\[
S (\phi, \psi) = \int_{\RR^4} - \frac{1}{2}  (\d \phi \wedge * \d \phi)_{W_+ \otimes U} + \frac{1}{2} (\psi , \sd \dd \psi)_U .
\]
As usual, we denote the antifields by $\phi^*$ and $\psi^*$. 

The action of supersymmetry is given by a linear and quadratic functional:
\begin{align*}
I^{(1)} (Q) & = \int (\phi^*, (Q, \psi))_{W_+ \otimes U} + (\psi^*, \rho(\d \phi) Q)_U \\
I^{(2)} (Q_1 , Q_2) & =  \int \frac{1}{4}(\Gamma(Q_1, Q_2)_{W_+} , \Gamma(\psi^*, \psi^*)_{U}) \textcolor{red}{-\frac{1}{2} (Q_1, \psi^*)_U(Q_2, \psi^*)_U  }.
\end{align*}
where $Q, Q_1,Q_2 \in \Sigma = S_+ \otimes W_+$. 
We point out some conventions in the definition of $I^{(2)}$. 
Define $\Gamma (Q_1,Q_2)_{W_+} \in V$ as the image of $Q_1 \otimes Q_2$ under the composition
\[
\Sigma \otimes \Sigma = (S_+ \otimes W_+) \otimes (S_+ \otimes W_+) \cong (S_+ \otimes S_+) \otimes (W_+ \otimes W_+) \xto{\Gamma \otimes \omega_{W_+}} V .
\]
%Similarly, $\Gamma(\psi^*, \psi^*)_U \in V$ is defined as the image of $\psi^* \otimes \psi^*$ under the composition 
%\[
%(S_+ \otimes U) \otimes (S_+ \otimes U) \cong (S_+ \otimes S_+) \otimes (U \otimes U) \xto{\Gamma \otimes \omega_{U}} V .
%\]
%The pairing in the first term in $I^{(2)}$ is then induced from the natural one on $V$. 
%Also, $(Q_1, \psi^*)$ is defined as the image of $Q_1 \otimes \psi^*$ under the composition \brian{is this one zero???}

The following result states that these functionals encode an off-shell action of six-dimensional $\cN=(1,0)$ supersymmetry on the half hypermultiplet valued in $U$. 
The proof is very similar to the case of the four-dimensional $\cN=1$ chiral multiplet. 

\begin{thm}
The functional $\fS = S + I^{(1)} + I^{(2)}$ satisfies the classical master equation
\begin{equation}\label{CMEhyper}
\d_{\rm Lie} \fS + \frac{1}{2} \{\fS, \fS\} = 0 .
\end{equation}
\end{thm}

We decompose the classical master equation (\ref{CMEhyper}) into the following equations:
\begin{equation}\label{CMEhyper2}
\begin{array}{rrrrrr}
\{S , I^{(1)}\} & = & 0 \\ 
\{S, I^{(2)}\} + \d_{CE} I^{(1)} + \frac{1}{2} \{I^{(1)}, I^{(1)}\} & = & 0 \\
\d_{CE} I^{(2)} + \{I^{(1)}, I^{(2)}\} & =& 0 \\
\{I^{(2)}, I^{(2)}\} & =& 0
\end{array}
\end{equation}

The last equation is automatically satisfied since $I^{(2)}$ is independent of $\phi$ and $\psi$. 

The first equation in (\ref{CMEhyper2}) states that the classical action for the chiral multiplet is supersymmetric. 

\begin{lemma} 
One has $\{S, I^{(1)}\} (Q) = 0$ for all $Q \in \Sigma$. 
\end{lemma}
\begin{proof}
Observe
\[
- \{\d \phi \wedge * \d \phi, \phi^*  (Q, \psi) \} = - 2 \d (Q, \psi) \wedge * \d \phi .
\]
and
\[
\{(\psi, \sd \dd \psi) , (\psi^*, \rho(\d \phi) Q) \} = 2 (\rho(\d \phi) Q, \sd \dd \psi) \d^4 x .
\]
The sum of these two terms is zero by Lemma \ref{lem: oneform}. 
\end{proof}

Next, we move on to the second equation in (\ref{CMEhyper2}). 

\begin{lemma} 
One has
\begin{equation}\label{CMEhyper3}
\{S, I^{(2)}\} + \d_{CE} I^{(1)} + \frac{1}{2} \{I^{(1)}, I^{(1)}\} = 0 .
\end{equation}
\end{lemma}
\begin{proof}
Evaluating the equation (\ref{CMEhyper3}) on $v_1,v_2 \mf{iso}(V)$ reduces to the claim that (??) defines a strict Lie action. 
Evaluating on $v \in \mf{iso}(V)$ and $Q \in \Sigma$, the claim reduces to the fact that $I^{(1)}$ is Poincar\'{e} invariant.
So, the only nontrivial term to check is the evaluation on $Q_1,Q_2 \in \Sigma$. 

The individual terms are:
\begin{align*}
\{I^{(1)}, I^{(1)}\}(Q_1, Q_2) = & - \phi^* (Q_1, \rho(\d \phi) Q_2) - \phi^* (Q_2, \rho(\d \phi)Q_1) \\ &  -  (\psi^*, \rho(\d(Q_1,\psi))Q_2) - (\psi^*, \rho(\d(Q_2, \psi)) Q_1) 
\end{align*}

\begin{align*}
(\d_{CE}I^{(1)})(Q_1,Q_2) = & - \phi^* L_{\Gamma(Q_1,Q_2)} (\phi) - (\psi^*, \Gamma(Q_1,Q_2) . \psi)
\end{align*}

and
\begin{align*}
\{S, I^{(2)}(Q_1,Q_2)\} = & \frac{1}{2} \Gamma(Q_1,Q_2) \Gamma(\psi^*, \sd \dd \psi) \textcolor{red}{- \frac{1}{2} (Q_1, \psi^*) (Q_2, \sd \dd \psi) - \frac{1}{2} (Q_1, \sd \dd \psi) (Q_2, \psi^*) }.
\end{align*}
Note that we can rewrite this as $\frac{1}{2} \rho(\Gamma(Q_1, Q_2)) \sd \dd \psi$. 

We first collect all terms in Equation (\ref{CMEhyper3}) proportional to $\phi^*$:
\[
- \frac{1}{2} (Q_1, \rho(\d\phi) Q_2) - \frac{1}{2} (Q_2, \rho(\d \phi) Q_1) - L_{\Gamma(Q_1,Q_2)} \phi .
\]
By the Clifford identity
\brian{ $v \wedge \Gamma(Q_1, Q_2) = (Q_1 , \rho(v) Q_2)$} we observe that the first two terms cancel with the third term.  

Next, we collect all terms in Equation (\ref{CMEhyper3}) proportional to $\psi^*$:
\begin{align*}
& - \frac{1}{2} \rho(\d(Q_1,\psi))Q_2 - \frac{1}{2} \rho(\d(Q_2, \psi)) Q_1 - \Gamma(Q_1,Q_2) . \psi
\\ 
& + \frac{1}{2} \rho(\Gamma(Q_1,Q_2)) \sd \dd \psi \textcolor{red}{ - \frac{1}{2} (Q_2, \sd \dd \psi)Q_1 - \frac{1}{2} (Q_1, \sd \dd \psi)Q_2}
\end{align*}
By Proposition \ref{prop:cliffordactionproperties} item (2), the first, second, and fourth terms combine to give
\[
- \frac{1}{2} \left( \sd \dd \rho(\Gamma(Q,\psi)) Q_2 + \sd \dd \rho(\Gamma(Q_2, \psi))Q_1 \right)  .
\]
This term is equal to $\frac{1}{2} \sd \dd \rho(\Gamma(Q_1,Q_2)) \psi$ by \ref{prop:3psi}. 
The Clifford relation implies this term cancels with the remaining terms. 

\end{proof}

\begin{lemma}
\[\{I^{(1)}, I^{(2)}\}(Q_1, Q_2, Q_3) = 0\]
for every $Q_1, Q_2, Q_3\in S_+ \oplus S_-$.
\end{lemma}
\begin{proof}
We have
\begin{align*}
\{I^{(1)}(Q_1), I^{(2)}(Q_2, Q_3)\} = \frac{1}{2} (\Gamma(Q_2, Q_3), \Gamma(\rho(\d
\end{align*}

$\{I^{(1)}, I^{(2)}\}$ is obtained by cyclically symmetrizing the above expression. By Proposition \ref{prop:3psi} the cyclic symmetrization of the term with $c^*$ is zero. The Clifford relation implies that

\end{proof}
\newpage

\section{SCRAP}

A general theory of matter for $\cN = (1,0)$ supersymmetry arranges into a hypermultiplet which has the following BRST description.
Let $S_-^{6d}$ be the negative $\so(6)$ irreducible spin representation and suppose $R$ is a (complex) symplectic $\fg$-representation, with symplectic pairing labeled by $\<-,-\>_R$. 
The BRST fields of the $(1,0)$ hypermultiplet consist of a $W \otimes R$-valued scalar
\[
\phi \in \Omega^0(\RR^6) \otimes W \otimes R
\]
and an $R$ valued negative Weyl spinor
\[
\psi_- \in \Omega^0(\RR^6) \otimes S_-^{6d} \otimes R .
\]
Denote the induced symplectic pairing on $W \otimes R$ by $\<-,-\>_{W \otimes R}$. 
The action of the free hypermultiplet is given by
\[
S_{\rm matter} (\phi, \psi_-) = \frac{1}{2} \int_{\RR^6} \d^6 x \; \<\partial_{x_i} \phi , \partial_{x_i} \phi\>_{W \otimes R} + \frac{1}{2} \int_{\RR^6} \d^6 x \;\<\psi_-, \sd \dd \psi_-\>_R .
\]

The action of supersymmetry on the $6$-dimensional matter theory is encoded by a linear and quadratic functional:
\begin{align*}
I^{(1)} (Q) & = \int \<\phi^*, (Q, \psi_-)\>_{W \otimes R} + \int \<\psi^*_-, \rho(\d \phi) Q\>_R \\
I^{(2)} (Q_1 \otimes Q_2) & = \int \<\psi^*_- , \rho(\Gamma(Q_1, Q_2)) \psi_-\>_R 
\end{align*}
where we recall that the pairing is of the form $\Gamma : \Sigma^{\otimes 2} = (S_+ \otimes W)^{\otimes 2} \to V$. \brian{I'm also using $(S_+)^* = S_-$.}
These functionals provide an off-shell action of $\cN = (1,0)$ supersymmetry on the free hypermultiplet, as the following proposition shows.
Below, $\fL_{\rm matter}$ is the abelian local Lie algebra describing the free hypermultiplet, and $\mf A$ is the $(0,1)$ supertranslation algebra.  
As always, $I_{\rm Poin}$ is the action functional encoding ordinary Poincar\'{e} invariance. 

\begin{prop}[\cite{SWchar}]
The functional
\[
\fS_{\rm matter} = S_{\rm matter} + I_{\rm Poin} + I^{(1)} + I^{(2)} \in \clie^\bu(\mf A) \otimes \cloc^\bu(\fL_{\rm matter}) [-1]
\]
satisfies the Maurer-Cartan equation
\[
\left(\d_{\rm Lie} \fS_{\rm matter} + \frac{1}{2} \{\fS_{\rm matter}, \fS_{\rm matter}\} \right) = 0 .
\]
\end{prop}
\begin{proof}
\brian{Do this already! It's almost identical to the proof in 4d, just need to do a bit of translation.} 
\end{proof}

Next, we move on to the $6$-dimensional $\cN=(1,0)$ super Yang-Mills. 
The theory has BRST fields given by a ghost $c$, a $6$-dimensional gauge field $A$, and a Lie algebra valued spinor
\[
\lambda_+ \in \Omega^0(\RR^6) \otimes S_+^{6d} \otimes W \otimes \fg 
\]
where $W$ is a (complex) $2$-dimensional symplectic vector space. 
The tensor product $S_+^{6d} \otimes W$ forms the $8$-dimensional symplectic Weyl representation. 
The action of supersymmetry on the $6$-dimensional matter theory is encoded by a linear and quadratic functional:
\begin{align*}
I_{\rm gauge}^{(1)}(Q) & = \int \<A^*, \Gamma(Q, \lambda_+)\> + \<\lambda^*, \sd F_A Q\> \\
I_{\rm gauge}^{(2)}(Q_1 \otimes Q_2) & = \int \left\<\lambda^* \;,\; \rho(\Gamma(Q_1,Q_2)) \lambda^* + \frac 12 \left((Q_2, \lambda^*)Q_1 + (Q_1, \lambda^*)Q_2\right)\right\>  .
\end{align*}
These functionals define an off-shell action of the $\cN = (0,1)$ supersymmetry on the vector multiplet, which we proved above in Theorem \label{SUSY_YM_theorem}, and state below as a proposition. 
Below, $\fL_{\rm gauge}$ denotes the local Lie algebra describing super Yang-Mills, and $S_{\rm BV, gauge}$ is the full BV action. 

 implies the following proposition for $6d$
\begin{prop}[special case of Theorem \ref{SUSY_YM_theorem}]
The functional
\[
\fS_{\rm gauge} = S_{\rm BV, gauge} + I_{\rm Poin} + I_{\rm gauge}^{(1)} + I_{\rm gauge}^{(2)} \in \clie^\bu(\mf A) \otimes \cloc^\bu(\fL_{\rm gauge}) [-1]
\]
satisfies the Maurer-Cartan equation
\[
\left(\d_{\rm Lie} \fS_{\rm gauge} + \frac{1}{2} \{\fS_{\rm gauge}, \fS_{\rm gauge}\} \right) = 0 .
\]
\end{prop}

\end{document}
