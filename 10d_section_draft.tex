\documentclass[10pt, oneside]{article}

\input ./combined_macros.sty
\newcommand{\hCS}{\mathrm{hCS}}

\addbibresource{Twist.bib}

\title{Twists of supersymmetric gauge theories}
\author{Chris Elliott\and Pavel Safronov \and Brian Williams}

\date{\today}

\begin{document}

\section{Background Structure}
\chris{This section consists of material common to our twisting calculations, to be edited and perhaps inserted before the 10d section}

\chris{As you pointed out, these lemmas should go with other discussion of functional analysis issues.}
\begin{lemma}[Free to Interacting Spectral Sequence] \label{free_int_ss_lemma}
Let $(\mc E^\bullet, Q, \omega, I)$ be a classical field theory with polynomial interaction.  There is a convergent spectral sequence whose $E_1$ page is the complex of classical observables of the underlying free theory, and whose $E_\infty$ page is equivalent to the complex of classical observables of the interacting theory.
\end{lemma}

\begin{lemma} \label{invert_quis_lemma}
Any quasi-isomorphism of cochain complexes admits a quasi-inverse.
\end{lemma}

\subsection{Dimensional Reduction} \label{dim_red_section}
\chris{I'm including this with the aim to not have to prove supersymmetry invariance of non-maximal theories.  It's a pretty rough draft, so apologies there}
The idea of the \emph{dimensional reduction} of a rotation invariant classical field theory to a linear subspace of $\RR^n$ is to restrict attention to only those fields with are constant in directions perpendicular to the subspace. 

\begin{definition}
Let $i\colon \RR^m \inj \RR^n$ be the inclusion of a linear subspace, and let $p \colon \RR^n \to \RR^m$ be the corresponding orthogonal projection. Let $(E, \omega, Q, I)$ be a rotation invariant classical field theory on $\RR^n$.  The \emph{dimensional reduction} of the classical field theory from $\RR^n$ to $\RR^m$ is the classical field theory whose underlying bundle of fields is the pullback bundle $i^*E$, with the following structure.  The symplectic pairing is defined as the composite
\[(i^*\mc E \otimes i^*\mc E)(U) \iso (p^*i^*\mc E \otimes p^*i^*\mc E)(U \times (\RR^m)^\perp) \to (\mc E \otimes \mc E)(U \times (\RR^m)^\perp) \overset \omega \to \dens(U \times (\RR^m)^\perp),\]
with the observation that the density we obtain splits canonically into a density on $U$ and a constant density on $(\RR^m)^\perp$.  The classical differential $Q \colon i^*\mc E(U) \to i^*\mc E(U)$ is defined by
\[\phi \mapsto i^*(Q(p^*(\phi))).\]
Finally, the classical interaction is defined similarly: the pullback $p^*$ defines a map from $\mc O(\mc E)$ to $\mc O(i^*\mc E)$ that preserves locality \footnote{Intuitively, given a local functional on fields on $\RR^n$, restrict to a local functional on those fields constant in directions perpendicular to $\RR^m$.}.
\end{definition}

Rotation invariance guarantees that the dimensional reduction is independent of the choice of $m$-dimensional subspace of $\RR^n$.

\begin{prop} \label{dim_red_SUSY_prop}
The dimensional reduction of a supersymmetric classical field theory on $\RR^n$ with supersymmetry algebra $\mf A = (\so(n) \oplus \gg_R) \ltimes (\RR^n \oplus \Pi \Sigma)$ is a supersymmetric classical field theory on $\RR^m$ with supersymmetry algebra $\mf A' = (\so(m) \oplus \gg_R) \ltimes (\RR^m \oplus \Pi \Sigma')$, where $\Sigma'$ is the restriction of the representation $\Sigma$ of $\so(n)$ to a representation of the subalgebra $\so(m)$ of rotations of the subspace $\RR^m \sub \RR^n$.
\end{prop}

\begin{proof}
 \chris{I think this will be straightforward.}
\end{proof}

\chris{we could alternatively write this in terms of BRST fields.}

\chris{By the way, I see no obstruction to proving the following.}
\begin{lemma}
If $\mf A$ is an $n$-dimensional supersymmetry algebra, and $Q \in \mf A$ is a square-zero supercharge, then the operations of dimensional reduction from $\RR^n$ to $\RR^m$ and twisting by $Q$ commute.
\end{lemma}


\subsection{Supersymmetric Yang-Mills Theory}
We'll begin with a detailed description of the maximal (complexified) supersymmetric Yang-Mills theories, which exist in dimensions 3,4, 6 and 10.  Using the methods described in Section \ref{dim_red_section}, the construction of these theory, with the supersymmetry actions, will descend to supersymmetric theories in all lower dimensions.  

Let us begin with a general description of supersymmetric Yang-Mills theory on $\RR^n$, with spinorial matter.  Fix a complex semisimple Lie group $G$ with Lie algebra $\fg$, and a spinorial representation $\Sigma$ of $\so(n;\CC)$ with symmetric equivariant pairing $\Gamma \colon \Sigma \otimes \Sigma \to \CC^n$.  The ordinary fields of super Yang-Mills theory on $\RR^{n}$ consist of:
\begin{itemize}
\item A connection $A \in \Omega^1(\RR^{n} ; \fg)$ on the trivial $G$-bundle;
\item A $\gg$-valued section $\lambda \in \Omega^0(\RR^{n}) \otimes \Pi \Sigma \otimes \fg$ of the spinor bundle associated to the representation $\Sigma$
\footnote{If we didn't complexify we would instead consider $G_\RR$ a compact connected Lie group, and a section of a real spinor bundle, whose structure is signature dependent.  For our purposes it's interesting enough to just consider the complexified theory and avoid signature issues.}.  
\end{itemize}
These fields are acted upon by the group of gauge transformations -- $G$-valued functions on $\RR^{n}$. 
Hence, there is a single ghost for the theory given by a $\gg$-valued section of the trivial $G$-bundle $c \in \Omega^0(\RR^{n} ; \fg)$. 

We can model the stack of fields modulo gauge transformations infinitesimally near the point $0$ by the corresponding BRST complex.  This is the local super Lie algebra
\[
L \;\;\; = \begin{array}{ccccc}
& \ul{0} & & \ul{1} & \\ 
& & & & \\
& \Omega^0(\RR^{n}; \gg) & \to & \Omega^1(\RR^{n}; \gg) \oplus \Omega^0(\RR^{n}; \Pi \Sigma \otimes \gg) & 
\end{array}
\]
with the de Rham differential, placed in cohomological degrees 0 and 1, with bracket induced from the Lie bracket on $\gg$.

The action functional in 10d super Yang-Mills is given by
\[S(A,\lambda) = \int_{\RR^{n}} \langle \frac{1}{2} F_A \wedge \ast F_A - (\lambda, \sd D_A \lambda)\rangle,\]
where $\langle - \rangle_\fg$ denotes an invariant pairing on $\gg$, and where the latter term is obtained using the vector-valued pairing $\Gamma$ on $\Sigma$ to obtain a Dirac operator $\sd D_A \colon \Sigma \to \Sigma^*$. \chris{Brian suggested defining this notation in the section on supersymmety more broadly.  I agree.}

We can re-encode this data in terms of the classical BV complex (see also \cite[Section 3.1]{ElliottYoo1}).  
This is the local $L_\infty$-algebra $\fL$ on $\RR^{n}$ whose underlying cochain complex takes the form
\[
\xymatrix{
& & \ul{0} & \ul{1} & \ul{2} & \ul{3} \\
\fL & = & \Omega^0(\RR^{n}; \gg) \ar[r]^{\d} &\Omega^1(\RR^{n}; \gg) \ar[r]^{\d \ast \d} &\Omega^{n-1}(\RR^{n}; \gg) \ar[r]^{\d} &\Omega^{n}(\RR^{n}; \gg) \\
& & &\Omega^0(\RR^{n}; \Pi \Sigma \otimes \gg) \ar[r]^{\ast \sd \d} &\Omega^{n}(\RR^{n}; \Pi \Sigma^* \otimes \gg), &
}\]
with degree $(-3)$ invariant pairing $\<-,-\>$ induced by the invariant pairing on $\gg$ and the evaluation pairing $(-,-)$ between $\Sigma$ and $\Sigma^*$, and with degree 2 and 3 brackets given by the action of $\Omega^0(\RR^{n}; \gg)$ on everything along with
\begin{align*}
\ell_2^{\mr{Bos}} \colon \Omega^1(\RR^{n};\gg) \otimes \Omega^1(\RR^{n};\gg) &\to \Omega^{n-1}(\RR^{n};\gg) \\
(A \otimes B) &\mapsto [A \wedge \ast \mr d B] + [\ast \mr d  A \wedge B] + \mathrm{d} \ast[A \wedge B] \\
\ell_2^{\mr{Fer}} \colon \Omega^1(\RR^{n};\gg) \otimes \Omega^0(\RR^{n}; \Sigma \otimes \gg) &\to \Omega^{n}(\RR^{10}; \Sigma* \otimes \gg) \\
(A \otimes \lambda) &\mapsto \ast \sd A \lambda
\end{align*}
in degree 2, and the map
\begin{align*}
\ell_3 \colon \Omega^1(\RR^{n};\gg) \otimes \Omega^1(\RR^{n};\gg) \otimes \Omega^1(\RR^{n};\gg) &\to \Omega^{n-1}(\RR^{n};\gg) \\
(A \otimes B \otimes C) &\mapsto [A \wedge \ast[B \wedge C]] + [B \wedge \ast[C \wedge A]] + [C \wedge \ast[A \wedge B]]
\end{align*}
in degree 3.

We obtain the BV action by the formula
\[
S_{BV} (\alpha) = \frac{1}{2} \<\alpha , Q_{BV} \alpha\> + \sum_{n \geq 2} \frac{1}{n!} \<\alpha, \ell_n(\alpha,\ldots, \alpha)\> 
\]
where $\alpha$ is a general BV field and $Q_{BV}$ is the linear BV differential. 
The statement that $S_{BV}$ is gauge invariant is encoded by the fact that it satisfies that classical master equation $\{S_{BV}, S_{BV}\} = 0$, which is equivalent to the statement that $S_{BV}$ determines a Mauer-Cartan element in the dg Lie algebra $\cloc^\bu(\fL)[-1]$.

\subsubsection{Supersymmetry Actions}
For minimal choices of the spinorial representation, Yang-Mills theory admits an action of the supersymmetry algebra when $n=3,4,6$ or 10, so $\Sigma$ is the 2-dimensional Dirac representation $S$, the 4-dimensional Dirac representation $S_+ \oplus S_-$, the 8-dimensional symplectic Weyl representation $S_+ \otimes W$ where $W$ is a 2-dimensional symplectic vector space, and the 16-dimensional Weyl representation $S_+$ respectively.  We will now proceed to construct this supersymmetry algebra action. \chris{can we do it all in one go, using the fact that the 3 $\psi$ rule \cite[Theorem 11]{BaezHuerta} works exactly in these dimensions?  If not we can just write the calculation out four times.  We've already started to write it in the 10-dimensional case, though it's incomplete.  How would you suggest proceeding?}

\subsection{Standard Equivalences Between Classical Field Theories}
In this section, we will collect some standard lemmas that we'll use to simplify the descriptions of twisted supersymmetric field theories below.

\begin{lemma} \label{symplectomorphism_lemma}
A linear symplectomorphism $F \colon \mc E^\bullet \to \mc E^\bullet$ induces an equivalence of theories between $(\mc E^\bullet, Q, \omega, I)$ and $(\mc E^\bullet, F^*Q, \omega, F^*I)$.
\end{lemma}

\begin{lemma} \label{inclusion_and_projection_lemma}
Let $(\mc E^\bullet, Q)$ be a cochain complex, and let $(\mc C^\bullet, Q')$ be a contractible cochain complex.
\begin{enumerate}
 \item Let $F \colon \mc C^\bullet \to \mc E^\bullet$ be a degree 1 map making $(\mc C^\bullet \overset F\to \mc E^\bullet)$ into a cochain complex.  Then the canonical inclusion $\mc E^\bullet \to (\mc C^\bullet \overset F\to \mc E^\bullet)$ is a quasi-isomorphism.
 \item Let $F' \colon \mc E^\bullet \to \mc C^\bullet$ be a degree 1 map making $(\mc E^\bullet \overset {F'}\to \mc C^\bullet)$ into a cochain complex.  Then the canonical projection $(\mc E^\bullet \overset {F'}\to \mc C^\bullet) \to \mc E^\bullet$ is a quasi-isomorphism.
\end{enumerate}
\end{lemma}

\begin{lemma} \label{symplectic_composite_lemma}
Let $(\mc E^\bullet,Q)$ be a cochain complex, let $(\mc C_1^\bullet, Q_1)$ and $(\mc C_2^\bullet, Q_2)$ be contractible cochain complexes, and let $\mc C_1^\bullet \overset{F_1}\to \mc E^\bullet \overset{F_2}\to \mc C_2^\bullet$ be a pair of degree 1 maps so that the differential $Q + Q_1 + Q_2 + F_1 + F_2$ on the total complex squares to 0.  Suppose the graded vector space $\mc E^\bullet \oplus \mc C_1^\bullet \oplus C_2^\bullet$ is equipped with a $-1$-shifted symplectic structure so that $\mc C_1^\bullet \oplus \mc C_2^\bullet$ is a symplectic subspace.  Then the cochain map 
\begin{equation}
\label{symp_composite_eqn}\mc E^\bullet (\mc C_1^\bullet \overset{F_1}\to \mc E^\bullet \overset{F_2}\to \mc C_2^\bullet)
\end{equation}
obtained as the composite of the projection from Lemma \ref{inclusion_and_projection_lemma} (2) with a quasi-inverse to the inclusion from Lemma \ref{inclusion_and_projection_lemma} (1) is a symplectomorphism.
\end{lemma}

\begin{lemma} \label{interaction_pullback_lemma}
In the set-up of Lemma \ref{symplectic_composite_lemma}, suppose we're given an interaction $I$ on the graded vector space $\mc E^\bullet \oplus \mc C_1^\bullet \oplus C_2^\bullet$, which pulls back to $I'$ under the inclusion of $\mc E^\bullet$.  If $\mc C_2^\bullet = 0$, or if $\mc C_1^\bullet$ and $\mc C_2^\bullet$ are both isotropic subspaces of $\mc E^\bullet \oplus \mc C_1^\bullet \oplus C_2^\bullet$, then the map \ref{symp_composite_eqn} is compatible with the interactions $I$ and $I'$
\end{lemma}

\section{Twists}
\subsection{Dimension 10}
\chris{background, recall there is only one class of twists, by a pure spinor.  There's a trivial twisting homomorphism from $\SU(5)$, and so it's defined on CY5s.  Give names to the BV component fields as irreducible $\SU(5)$ representations: this is important for the argument below.}

\begin{theorem}
The holomorphic twist of 10d super Yang-Mills theory on a Calabi-Yau 5-fold $X$ is equivalent to holomorphic Chern-Simons theory on $X$.
\end{theorem}

\begin{proof}
\chris{recall the steps.  1) Modify $\chi^\vee$ by a linear change of variables $\chi^\vee \mapsto \chi^vee + \Lambda F_{1,1}$. 2) Integrate out the contractible doublet $\chi, \chi^\vee$.  3) Note that the pairs $A_{1,0}, \psi$ and $\psi^\vee, A_{1,0}^\vee$ can be killed in pairs, and the composite of these operations defines a symplectomorphism.  4) These steps each preserve the action, so that we have a map of interacting theories. 5) Using the spectral sequence, this map is an equivalence.}
\end{proof}


\pagestyle{bib}
\printbibliography

\end{document}