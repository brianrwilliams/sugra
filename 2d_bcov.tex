\documentclass[10pt]{amsart}

\usepackage{macros,slashed}

\linespread{1.25}

%\usepackage{tikz}
%\usetikzlibrary{arrows,shapes}
%\usetikzlibrary{trees}
%\usetikzlibrary{matrix,arrows}
%\usetikzlibrary{positioning}
%\usetikzlibrary{calc,through}
%\usetikzlibrary{decorations.pathreplacing}
%\usepackage{pgffor}

\def\PV{{\rm PV}}

\title{BCOV theory on a complex surface}

\def\brian{\textcolor{blue}{BW: }\textcolor{blue}}


\begin{document}
\maketitle
\tableofcontents

Suppose $X$ is a complex manifold of complex dimension $2$. equipped with a holomorphic symplectic form $\omega \in \Omega^{2,hol}(X)$.
We study BCOV theory on $X$, which should be thought of as the theory of ``divergence-free holomorphic polyvector fields" with respect to the fixed symplectic structure.

Let $\PV^{k, hol}(X)$ be the space of holomorphic $k$-polyvector fields; these are holomorphic sections of the $k$th exterior power of the holomorphic tangent bundle $\wedge^k T^{1,0}X$.
The holomorphic volume form identifies
\ben
\PV^{0,hol}(X) \cong_\omega \Omega^{2,hol}(X) \;\;, \;\; \PV^{1,hol}(X) \cong_\omega \Omega^{1,hol}(X) \;\; , \;\; \PV^{2,hol}(X) \cong_\omega \Omega^{0,hol}(X) = \sO^{hol}(X) .
\een
In particular, the holomorphic de Rham operator $\partial : \Omega^{i,hol}\to \Omega^{i+1,hol}$ defines an operator
\be\label{partial}
\partial : \PV^{k,hol}(X) \to \PV^{k-1,hol}(X) .
\ee
At the crudest level, the fields consist of those holomorphic polyvector fields in the kernel of the operator $\partial$. 

The problem with this definition is that it does not fit the usual definition of the space of fields for a classical field theory.
In particular, the space of fields is not equal to the smooth sections of a vector bundle. 
The solution to this problem consists of two steps:
\begin{itemize}
\item Resolve the space of holomorphic polyvector fields via its Dolbeualt resolution. 
This amounts to replacing ${\rm PV}^{k,hol}(X)$ by the complex $\PV^{k,*}(X)$ equipped with its Dolbeualt differential 
\ben
\dbar : \PV^{k,l}(X) \to \PV^{k,l+1}(X) .
\een
As sheaves one has $\PV^{k,*} \simeq \PV^{k,hol}$. 
We equip the double complex $\PV^{*,*}(X)$ with the grading coming from totalization.
Notice that since the operator $\partial$ in (\ref{partial}) is a holomorphic differential operator it extends in a natural way to $\PV^{*,*}(X)$. 
\item The next step is to resolve the kernel
\ben
\ker \partial \subset \PV^{*,hol}(X),
\een
or more accurately its Dolbeualt version
\ben
(\ker \partial, \dbar) \subset \left(\PV^{*,*}(X), \dbar\right)
\een
One introduces the formal parameter $t$ of cohomological degree $2$ ...
\end{itemize}

The non-local form of the action functional is
\ben
S(\alpha) = \frac{1}{2} \int_X \alpha \dbar (\partial^{-1} \alpha) + \frac{1}{3} \int_X \alpha^3 .
\een
Notice that in this form only elements in $\PV^{\geq 1, *}$ appear in the action functional.
There is a change of coordinates that will put this action functional in a more familiar form.

Suppose $\alpha^{1,*}$ is in the kernel of $\partial : \PV^{1,*} (X) \to \PV^{0,*}(X)$.
Then, locally, we can assume that $\alpha^{1,*} = \partial (\alpha^{2,*})$ for some $\alpha^{2,*} \in \PV^{2,*}(X)$.
Under the isomorphisms between polyvector fields and Dolbeualt forms, we redefine our fields via
\begin{align*}
\PV^{2,*}(X) & \cong_\omega \Omega^{0,*}(X)\\
\alpha^{2,*} & \leftrightarrow A^{0,*} 
\end{align*}
and
\begin{align*}
PV^{0,*}(X) & \cong_\omega \Omega^{2,*}(X) = \omega \cdot \Omega^{0,*}(X) \\
\alpha^{0,*} & \leftrightarrow \omega \cdot B^{0,*}
\end{align*}
In particular, the new fields $A,B$ are now Dolbeault forms. 
Keeping track of the cohomological degree, we find that the our new space of fields is
\ben
A + \epsilon B \in \Omega^{0,*}(X)[\epsilon][1]
\een
where $\epsilon$ is a formal parameter of degree $+1$.
In particular, the fields in degree zero are $A^{0,1} \in \Omega^{0,1}(X)$ and $B^0 \in \Omega^0(X)$. 
In terms of these new fields, the action functional takes the form
\be\label{action2}
S(A,B) = \int_X \omega B \dbar A + \frac{1}{2} \int_X B \partial A \partial A .
\ee

There is yet another way we can recast this classical theory that more clearly reflects the dependence of the theory on the symplectic form. 
As above, suppose $X$ is a symplectic surface with holomorphic symplectic form $\omega$. 
We will consider the Lie algebra of holomorphic vector fields $\cX^{hol}(X)$, or more accurately, its Dobleault resolution $\cX^{*} (X) = \Omega^{0,*}(X, T^{1,0}X)$.

There is a sub Lie algebra of $\cX^{hol}(X)$ consisting of holomorphic {\em symplectic} vector fields
\ben
\{X \in \cX^{hol}(X) \; | \; L_X \omega = 0\} \subset \cX^{hol}(X) .
\een
Since the symplectic form is holomorphic, there is a resolution via the Dolbeualt complex.
Indeed, there is a sub dg Lie algebra 
\ben
\cX_{symp}^* (X) = \{\xi \in \Omega^{0,*}(X, T^{1,0}X) \; | \; L_\xi \omega = 0\} \subset \cX^*(X) .
\een
When one equips the left hand side with the $\dbar$ differential, this is a resolution for holomorphic symplectic vector fields. 

There is a related Lie algebra of holomorphic {\em Hamiltonian} vector fields. 
This Lie algebra has underlying vector space the space of holomorphic functions $\sO^{hol}(X)$ and is equipped with Lie bracket given by the holomorphic Poisson bracket $\{-,-\}_\omega$. 
As we've done several times already, this Lie algebra admits a Dolbeualt resolution as a dg Lie algebra 
\be\label{hamiltonian}
\left(\Omega^{0,*}(X), \dbar, \{-,-\}_\omega \right).
\ee

Locally, every holomorphic symplectic vector field is Hamiltonian. 
Thus, as sheaves of dg Lie algebras we might as well replace $\cX_{symp}^{*}(X)$ with the dg Lie algebra (\ref{hamiltonian}). 

Given any dg Lie algebra, there is a standard construction to define a classical BV theory.

The fields are
\ben
(A, B) \in \Omega^{0,*}(X) [1] \oplus \Omega^{2,*}(X) 
\een
The action functional is
\ben
\int_X B \dbar A + \frac{1}{2} \int_X B \{A, A\} .
\een
6
\end{document}

