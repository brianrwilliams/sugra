\documentclass[10pt, oneside]{article}

\input ./combined_macros.sty

\addbibresource{Twist.bib}

\title{Note on Superconformal Chern--Simons}
\author{Chris Elliott\and Pavel Safronov \and Brian Williams}

\date{\today}

\begin{document}
 
\section*{Rough Note on Supersymmetry in Superconformal Chern--Simons}
Choose a real semisimple Lie algebra $\gg$, a real representation $U$ of $\gg$ equipped with an inner product, and a superpotential $\mf W \colon U \to \RR$.  Let $\Sigma = S_\RR \otimes \Delta$ be a real spinorial representation of $\so(1,2)$, where $S_\RR$ is the 2-dimensional Majorana spin representation and $\Delta$ is equipped with an inner product (so we are considering $\mc N = \dim(\Delta)$ supersymmetry).  Superconformal Chern--Simons will have the following BRST fields in addition to the ghost $c$.
\begin{align*}
A &\in \Omega^0(\RR^3; \gg) \\
X &\in \Pi \Omega^0(\RR^3; U) \\
\Psi &\in \Pi \Omega^0(\RR^3; U \otimes \Sigma).
\end{align*}
Choose a supercharge $Q \in \Sigma$.  We define on-shell supersymmetry transformations by $Q$ to be
\begin{align*}
\delta_Q X &=  \langle Q, \Psi \rangle\\
\delta_Q \Psi &= -\frac 12\left(\rho(\d_A X)Q + \nabla\mf W(X)Q\right)\\
\delta_Q A &= k\, T(X, \Gamma( Q, \Psi )).
\end{align*}
Here $\langle - , - \rangle$ is the skew-symmetric scalar pairing on $\Sigma$ and $T \colon U \otimes U \to \gg$ is the skew-symmetric map obtained from the action using the inner products on $U$ and $\gg$.  Also we've included a number $k$ to be determined, which may depend on $\mc N$.  

Let's sketch the anticommutation relations of the $\delta_Q$ operators.  I'm initially going to assume that $\mc N=\dim \Delta = 1$.  Start with the action on $X$.
\begin{align*}
\delta_{Q} \delta_{Q'}X &= \langle Q', \delta_{Q}\Psi \rangle \\
&= - \frac 12 \langle Q', \rho(\d_A X)Q \rangle - \frac 12 \nabla\mf W(X) \langle Q', Q \rangle \\
\text{so } [\delta_Q, \delta_{Q'}]X &= - \frac 12\langle Q', \rho(\d_A X)Q\rangle - \frac 12\langle Q, \rho(\d_A X)Q'\rangle \\
&= \frac 12\langle \rho(\d_A X)Q, Q'\rangle + \frac 12\langle \rho(\d_A X)Q', Q\rangle \\
&= \frac 12g(\rho(\d_A X), \Gamma(Q',Q) + \Gamma(Q,Q')) \\
&=  g(\rho(\d_A X), \Gamma(Q',Q)) \\
&= (\dd_i + A_i)X,
\end{align*}
where $\Gamma(Q',Q)$ is translation in the $x^i$ direction.  This looks pretty good.

Now let's do the analogous calculation for $A$.
\begin{align*}
\delta_{Q} \delta_{Q'}A &= k \left( T(\delta_Q X, \Gamma( Q, \Psi )) + T(X, \Gamma( Q', \delta_{Q}\Psi )) \right) \\
&= k \left( T(\langle Q', \Psi \rangle, \Gamma( Q, \Psi )) + \frac 12T(X, \Gamma( Q', \delta_{Q}\Psi )) \right) \\
&= k \left( T(\langle Q', \Psi \rangle, \Gamma( Q, \Psi )) + \frac 12T(X, \Gamma( Q', \rho(\d_A X)Q + \Gamma(Q',Q)\nabla \mf W(X)))\right).
\end{align*}
The first two terms are anti-symmetric in $Q$ and $Q'$ \chris{I think?  I checked it fairly easily for the second term but I'm getting confused with the first term}.  Therefore
\[[\delta_Q,\delta_{Q'}]A = k\Gamma(Q,Q')T(X, \nabla \mf W(X)).\]

The Chern--Simons gauge field $A$ is supposed to be topological so translations should act trivially on $A$.  This term is, therefore, something that should vanish on-shell.

If we compute $[\delta_Q, \delta_{Q'}]\Psi$ we get something more complicated.  
\begin{align*}
\delta_{Q} \delta_{Q'}\Psi &=-\frac 12\left( \rho(\d_A (\delta_Q X)Q' + \rho(\delta_Q(A)\cdot X)Q' + \nabla \mf W(\delta_Q X)Q' \right) \\
&= -\frac 12\left( \rho(\langle Q, \d_A(\Psi)\rangle)Q' + k \rho(T(X, \Gamma( Q, \Psi ))\cdot X)Q' + \nabla \mf W(\langle Q, \Psi\rangle)Q' \right).
\end{align*}
If we take the first term alone and symmetrize then we get
\begin{align*}
-\frac 12 \left(\rho(\d_A (\delta_Q X)Q' + \rho(\d_A (\delta_{Q'} X)Q \right) &= \frac 12\left( g(\d_A \Psi, \Gamma(Q, Q') + \Gamma(Q',Q)\right) \\
&= (\dd_i + A_i)\Psi,
\end{align*}
so this term alone gives the Lie derivative.  The other two terms should be $Q_{\mr{BV}}$-exact.  So in total we're claiming that
\begin{align*}
T(X, \nabla \mf W(X)) &\in \Omega^0(\RR^3; \gg) \\\
T(X, \Psi ))\cdot X &\in \Omega^0(\RR^3;U \otimes S) \\
\text{ and } T(X, \nabla \mf W(\Psi)) &\in \Omega^0(\RR^3; \gg \otimes S) 
\end{align*}
all vanish on-shell.  The vanishing of the first two follow from the claims in the paper that the supersymmetry closes off-shell when you insert auxiliary fields, which on-shell take the values $\chi = \frac 12 T(X,\Psi)$ and $C = \nabla \mf W(X)$ (suppressing the pairing on $U$).  I'm not immediately sure about the third.

Let's try to work it out in the BV formalism.  Let me first of all try the simplest, abelian, case where $\gg = \mf u(1)$ and $\mf W$ is quadratic.  Say $\left| \nabla \mf W(X)\right|^2 = (X, m^2(X))$, for some $m^2 \in \GL(U)$.  The classical BV complex looks like
\[\xymatrix{
\Omega^0(\RR^3)_c \ar[r]^\d &\Omega^1(\RR^3)_A \ar[r]^\d &\Omega^2(\RR^3)_{A^*} \ar[r]^d &\Omega^3(\RR^3)_{c^*} \\
&\Omega^0(\RR^3;U)_X \ar[r]^{\Delta + m^2} &\Omega^0(\RR^3; U)_{X^*} & \\
&\Pi \Omega^0(\RR^3; S_\RR)_\Psi \ar[r]^{\sd \d - m^2} &\Pi \Omega^0(\RR^3; S_\RR)_{\Psi^*}.
}\]
Even in this case the theory isn't quite free: it has a quartic interaction of the form $|T(X,\Psi)|^2$. \chris{I'm gonna keep thinking about this.}





% \begin{itemize}
%  \item There should only be an $\mc N=\dim(\Delta)$ supertranslation action given appropriate constraints on $\gg$, $U$ and $\mf W$.
%  \item Even when these constraints are satisfied, $([\delta_Q, \delta_{Q'}] - \mc L_{\Gamma(Q,Q')})\Psi$ should not equal zero, but rather something $Q_{\mr{BV}}$-exact.
% \end{itemize}



 
\end{document}
