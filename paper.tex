\documentclass[10pt, oneside]{article}

\input ./combined_macros.sty

\newcommand{\Dens}{\mathrm{Dens}}

\addbibresource{Twist.bib}

\usepackage{pdflscape}
\usetikzlibrary{shapes.geometric, arrows, positioning}

\tikzstyle{s16} = [rectangle, rounded corners, minimum width=1.8cm, minimum height=1cm,text centered, draw=black,fill=red!30]
\tikzstyle{s16chiral} = [s16, dashed]
\tikzstyle{s8 } = [rectangle, rounded corners, minimum width=1.8cm, minimum height=1cm,text centered, draw=black,fill=orange!30]
\tikzstyle{s4} = [rectangle, rounded corners, minimum width=1.8cm, minimum height=1cm,text centered, draw=black,fill=yellow!30]
\tikzstyle{s2chiral} = [rectangle, dashed, rounded corners, minimum width=1.8cm, minimum height=1cm,text centered, draw=black,fill=green!30]
\tikzstyle{dimension} = [circle, text centered, text width=0.7cm, minimum height=0.7cm, draw=black]
\tikzstyle{arrow} = [thick,->,>=stealth]

\title{Twists of supersymmetric gauge theories}
\author{Chris Elliott\and Pavel Safronov \and Brian Williams}

\date{\today}

\begin{document}

\maketitle

\section*{Introduction}

\section{The BV-BRST Formalism}

Throughout the paper we will mostly work with a $\ZZ\times\ZZ/2$-grading. \emph{Degree} will refer to the first (cohomological) grading and \emph{odd} or \emph{even} to the second (fermionic) grading.

Given a vector bundle $E\rightarrow M$ we denote by $\cE$ the space of smooth sections of $E$. We denote by $\oloc(\cE)$ the space of local functionals on $\cE$ (see \cite[Definition 4.5.1.1]{Book2}). We denote by $\oloc^+(\cE)\subset \oloc(\cE)$ the subspace of local functionals which are at least cubic.

\subsection{Classical BV Theories}

\begin{definition}
A {\bf free BV theory} on a manifold $M$ is the data:
\begin{itemize}
\item a finite rank $\ZZ\times\ZZ/2$-graded vector bundle $E \to M$ equipped with an even differential operator of cohomological degree $+1$
\[
Q \colon \cE \to \cE [1] 
\]
such that $(1)$: $Q^2 = 0$ and $(2)$: the pair $(\cE , Q)$ is an elliptic complex;
\item a map of bundles
\[
\omega\colon E \otimes E \to \Dens_M [-1]
\]
that is
\begin{enumerate}
\item[$(1)$] fiberwise nondegenerate,
\item[$(2)$] graded skew symmetric, and
\item[$(3)$] satisfies $\int_M \omega\<e_0, Q e_1\> = (-1)^{|e_0|} \int_M \omega(Q e_0, e_1)$ where $e_i$ are compactly supported sections of $E$ .
\end{enumerate}
\end{itemize}
\end{definition}

We call $\cE$ the {\em space of BV fields}.  
The pairing $\omega$ equips the algebra of local functionals on $E$ with a shifted bracket, in the following way.
In Section 3 of \cite{CosRenorm}, it is shown that to every {\em local} functional $f \in \cO(\cE)$ there exists a symplectic vector field $X_f$ on $\cE$ whose Hamiltonian function is $f$. 
Note that if $f$ is degree $n$, then $X_f$ is of degree $n+1$ due to the degree of $\omega$. 

\begin{definition}
The {\bf BV antibracket} $\{-,-\}$ is the bilinear map of cohomological degree $+1$
\[
\{-,-\} : \oloc(\cE) \times \oloc(\cE) \to \oloc(\cE)
\]
defined by sending a pair $(f,g)$ to the local functional $X_f (g)$.
\chris{Also cite \cite{WittenAntibracket}.}
\end{definition}

This pairing is graded symmetric and satisfies the graded Jacobi identity.
Note that since local functionals do not form an algebra, it is not a (shifted) Poisson bracket. 

\begin{definition}
A {\bf classical BV field theory} (or simply, classical field theory) is a free BV theory $(E, Q, \omega)$ equipped with an even functional
\[I \in \oloc^+(\cE)\]
of cohomological degree zero satisfying the classical master equation 
\[Q I + \frac{1}{2} \{I,I\} = 0 .\]
\end{definition}

Given a classical field theory $(E, Q, \omega, I)$ we denote by
\[S_{\mr{BV}} = \frac{1}{2} \int_M \omega(e, Q e) + I\in \oloc(E)\]
the BV action of the theory.

\begin{remark}
We will also consider {\bf $\ZZ/2$-graded classical field theories} which are defined as before, but where $E$ has only a single $\ZZ/2$-grading and, correspondigly, $Q$ is simply an odd operator.
\end{remark}

\begin{remark}
The data of a classical BV theory can be equivalently encoded in an elliptic $L_\infty$ algebra $L=E[-1]$ equipped with a symplectic isomorphism $L\cong L^![-3]$.
\end{remark}

A local functional $I$ induces an endomorphism $\{I,-\}$ on the space of all local functionals $\oloc(E)$. 
It also induces an endomorphism on the larger space of {\em all} functionals $\cO(\cE)$\footnote{Although, for two arbitrary functionals $F, G \in \cO(\cE)$, the bracket $\{F,G\}$ is not defined.}, see Definition \ref{dfn: fnl} in the appendix. 
The classical master equation implies that the operator $Q + \{I,-\}$ squares to zero on $\cO(\cE)$, and hence every classical field theory defines a cochain complex
\begin{equation}\label{bvcplx}
\left(\cO(\cE), Q + \{I,-\}\right) .
\end{equation}
In fact, this has the structure of a differentiable pro-cochain complex, see Definition \ref{dfn: pro}. 
We will refer to this as the {\em classical BV complex} of the theory. 

\begin{definition}
A {\bf morphism} $F\colon (E, Q, \omega, I) \to (E', Q', \omega', I')$ of classical field theories over the same manifold $M$ is a linear map of vector bundles
\[
F\colon E \to E'
\]
that intertwines the differentials $Q, Q'$, the pairings $\omega, \omega'$, and the interactions $I,I'$. 
\end{definition}

Note that a map of classical field theories induces a map of BV complexes, that is, a map of pro-cochain complexes:
\[
F^*\colon \left(\cO (\cE')[-1], Q' + \{I',-\} \right) \to \left(\cO (\cE)[-1], Q + \{I,-\} \right).
\]
This allows us to make the following definition of an equivalence between classical theories. 

\begin{definition} \label{equivalence_def}
A morphism of classical field theories $F\colon (E, Q, \omega, I) \to (E', Q', \omega', I')$ is an {\bf equivalence} if it induces a quasi-isomorphism of differentiable pro-cochain complexes
\[
F^* \colon  \left(\cO(\cE')[-1], Q' + \{I',-\} \right) \xto{\simeq} \left(\cO(\cE)[-1], Q + \{I,-\} \right) .
\]
\end{definition}

For a short background on homological algebra with differentiable pro-cochain complexes, including the notion of quasi-isomorphism, we refer to Appendix \ref{appx: top}.
For a more thorough introduction, and for which our conventions are based on, we refer to \cite{Book1}.

\subsection{Symmetries in the Classical BV Formalism} \label{symmetry_section}

\chris{The idea, and a description of the kind of infinitesimal symmetries we're talking about (maybe just by a super Lie algebra in this paper, although we allow $L_\infty$ actions).  More generally we might consider the action of general local $L_\infty$ algebras.}

\begin{definition} \label{infinitesimal_action_def}
Let $(E, Q,\omega, I)$ be a classical field theory 
A {\bf strict action} of a super Lie algebra $\fg$ on $(E, Q,\omega, I)$ is a map of super dg Lie algebras (in differentiable pro-cochain complexes)
\[
\rho\colon \fg \to \left(\oloc(E)[-1] , \{-,-\}_\omega, Q + \{I,-\}_\omega\right) .
\]
More generally, an $L_\infty$ {\bf action} is an $\infty$-morphism of super Lie algebras
\[
\rho\colon \fg \rightsquigarrow \left(\oloc(E)[-1] , \{-,-\}_\omega, Q + \{I,-\}_\omega\right) .
\]
\end{definition}

We denote components of an $\infty$-morphism by $\rho_2, \rho_3, \dots$, so that $\rho_2$ is a morphism of (super) chain complexes (see Appendix \ref{appx:infinity}).
\brian{Is there a reason $\rho_2$ is the linear map. I'd thought $\rho_1$...}

Concretely, an action $\rho$ of a super dg Lie algebra $\fg$ on a classical field theory is given by an element
\[I_{\rho} = \sum_{k\geq 1} I_{\rho}^{(k)}\in \oloc(E)\otimes \widehat{\sym}^{\geq 1}(\fg^*[-1])\]
which satisfies the Maurer--Cartan equation
\[\{S_{\mr{BV}}, I_\rho^{(k+1)}\} + \d_{CE} I_{\rho}^{(k)} + \frac{1}{2}\sum_{a+b=k+1} \{I_{\rho}^{(a)}, I_{\rho}^{(b)}\} = 0\]
for every $k\geq 0$.

\begin{definition}
\chris{The definition of the Chevalley-Eilenberg complex with coefficients, in particular the cup product / bracket, with its correct sign.  Pavel suggested saying that $C^\bullet(\gg; V) = \hom(\sym^\bullet(\gg[1]), V)$, and following the sign rule he used in \cite[Section 2.1]{SafronovCoisoInt}.}
\end{definition}

\begin{definition}
\chris{The definition of the equivariant observables in a classical BV theory with a symmetry as the relative Chevalley Eilenberg complex with coefficients in local functionals, with its anti-bracket.}
\end{definition}


\begin{remark}
\chris{Remark on the d\'ecalage isomorphism with a reference, then give an interpretation of the relative Chevalley-Eilenberg complex after applying this isomorphism as the inclusion of background fields valued in the constant sheaf with value $\gg$.}
\end{remark}

\chris{mention smooth group actions, with a reference.}

\subsection{The Classical Factorization Algebra}

Let $(E, Q, \omega, I)$ be a classical BV theory. We recall the formalism of \cite{Book1,Book2} which extracts a factorization algebra on spacetime.

\begin{definition}
Let $(E, Q, \omega, I)$ be a classical field theory on $M$. The {\bf classical factorization algebra of observables} $\Obs$ assigns to an open set $U \subset M$ the commutative dg algebra (in differentiable pro-cochain complexes)
\[
\Obs(U) = \left(\cO(\cE(U)) , Q + \{I,-\}\right) .
\]
\end{definition}

Notice that the BV complex we defined in the previous section, Equation (\ref{bvcplx}), is the global sections of the factorization algebra along $M$.
The BV complex can be thought of as the complex of functions on the derived critical locus of the action functional.
It makes sense to restrict fields to any open set $U \subset M$. 
Thus, likewise, the value of the factorization algebra on an open set $U \subset M$ can be interpreted as functions on the critical locus of the action functional where we restrict the theory to $U$. 

The fact that the assignment $U \mapsto \Obs(U)$ defines a factorization algebra can be found in \cite[Section 3]{Book2}. 

\subsection{From BRST to BV}

\begin{definition}
A {\bf classical BRST theory} on a manifold $M$ consists of the following data:
\begin{itemize}
\item a $\ZZ\times\ZZ/2$-graded vector bundle $F$ together with the structure of a local $L_\infty$ algebra on the shift $F[-1]$;
\item A local functional $S_{BRST} \in \oloc(\cF)$ of polynomial degree $\geq 2$.
\end{itemize}
Together, these data must satisfy the equation
\[Q_{BRST} S_{BRST} = 0,\]
where $Q_{BRST}$ is the Chevalley--Eilenberg differential defined by the $L_\infty$ structure on $F[-1]$. 
\end{definition}

We call $\cF$ the {\bf space of BRST fields}.

From a classical BRST theory $(\cF, S_{BRST})$, one can construct a classical BV theory as follows. Let $\{\ell_k\}_{k\geq 1}$ be the $L_\infty$ structure maps underlying the local Lie algebra $F[-1]$.

First, we define the free BV theory. Split $S_{BRST} = S^{free}_{BRST} + I_{BRST}$, where $I_{BRST}\in\oloc^+(\cF)$ and $S^{free}_{BRST}$ is a quadratic local functional which we may view as defining a map
\[S^{free}_{BRST}\colon F\rightarrow F^!.\]
The underlying bundle of the BV theory is
\[
E = F \oplus F^! [-1].
\]
The differential of the free BV theory is
\[
Q = \ell_1 + S^{free}_{BRST}.
\]
The BV pairing $\omega$ on $E$ is defined in terms of the natural pairing between $F$ and $F^!$.

The interacting theory is constructed as follows. First, note that for $k \geq 2$ the $L_\infty$ structure maps $\{\ell_k\}_{k \geq 2}$ on $\cF$ pull back to multilinear maps on $\cE$ via the obvious projection $p\colon \cE\rightarrow \cF$. These structure maps assemble into a local functional $I_F \in \oloc^+(\cE)$ defined by
\[
I_F (e) = \sum_{k \geq 2} \frac{1}{(k+1)!} \int_M \omega_F(e, (p^*\ell_k) (e, \ldots, e))
\] 
which is linear along $\cF^!$. Likewise, the BRST action $I_{BRST}$ pulls back to $\cE$, and we define the BV interaction as the sum
\[
I_{\mr{BV}} = I_{F} + p^* I_{BRST} \in \oloc^+(\cE) .
\]

\begin{lemma}
Suppose $(F, S_{BRST})$ is a classical BRST theory such that $(\cE, Q)$ defined above is an elliptic complex. Then $(E, Q, \omega, I)$ is a classical BV theory.
\end{lemma}

We refer to the classical BV theory $(E, Q, \omega, I)$ from the above statement as the {\bf $(-1)$-shifted cotangent bundle} of $\cF$ and by abuse of notation we often denote it simply by $\cE = T^*[-1] \cF$. In the case $S_{BRST} = 0$ we refer to the theory $T^*[-1] \cF$ as a theory of {\bf cotangent type}.

\subsection{Examples of BV-BRST Theories}

\begin{comment}
\brian{should we make these definitions, recollections, or just conventions?}

If $M$ is a smooth manifold, its {\em de Rham stack}, as a dg ringed space, is 
\[
M_{\rm dR} = \left(M , (\Omega^\bu_M, \d_{\rm dR})\right) .
\] 

If $X$ is a complex manifold, there are two naturally associated dg ringed spaces that will be important for us. 
The first is the $\dbar$-{\em stack}, defined by
\[
X_{\dbar} = \left(M , (\Omega^{0,\bu}_X, \dbar) \right) .
\]
The next is the {\em Dolbeault stack}, defined by
\[
X_{\rm Dol} = \left(M, (\Omega^{\bu,\bu}_X, \dbar)\right) .
\] 
Note that as sheaves of dg modules, there is a quasi-isomorphism $(\Omega^{p,*}_M, \dbar) = \Omega^{p, hol}_M$.
Thus, one can think of the underlying ring of the Dolbeault stack as functions on the shifted holomorphic tangent bundle $T^{1,0} [1] M$ of $M$.
\end{comment}

\subsubsection{Generalized BF theory} \label{gen_BF_section}

\begin{definition}
Let $X$ and $Y$ be complex manifolds and $M$ a smooth manifold. Fix an $L_\infty$ algebra $\fg$. The {\bf generalized BF theory} is the $(-1)$-shifted cotangent bundle of the following classical BRST theory:
\begin{itemize}
\item The spacetime is the smooth manifold $X\times Y\times M$.

\item The bundle of BRST fields is the $\ZZ$-graded bundle $F = \Omega^{0,\bu}_X \otimes \Omega^{\bu,\bu}_Y \otimes \Omega^\bu_M \otimes \fg[1]$, where every field is considered even with respect to the $\ZZ/2$-grading. $F[-1]$ is equipped with a natural local $L_\infty$ algebra structure from $\fg$.

\item The BRST action is $S_{BRST} = 0$.
\end{itemize}
\end{definition}

Let us unpack the definition. Let $d = \dim_\CC(X) + 2\dim_\CC(Y) + \dim(M)$. Then the bundle of BV fields is
\[E = \Omega^{0,\bu}_X \otimes \Omega^{\bu,\bu}_Y \otimes \Omega^\bu_M \otimes \fg[1]\oplus \Omega^{\dim(X),\bu}_X \otimes \Omega^{\bu,\bu}_Y \otimes \Omega^\bu_M\otimes \fg^*[d-2],\]
where we denote the two fields by $A$ and $B$. The BV action is
\[S = \int_{X\times Y\times M} \langle B\wedge (\dbar_X + \dbar_Y + \d_{\dR, M}) A\rangle + \sum_{k\geq 1}\frac{1}{k!} \int_{X\times Y\times M} \langle B\wedge \ell_k(A, \dots, A)\rangle,\]
where $\langle -, -\rangle$ is the natural pairing between $\fg^*$ and $\fg$ and $\ell_k$ denote the components of the $L_\infty$ structure on $\fg$.

\begin{example}
For $X=Y=\pt$ and $\fg$ an ordinary Lie algebra we recover the usual topological BF theory with the BV action
\[S = \int_M \left\langle B\wedge \left(\d_\dR A+ \frac{1}{2}[A\wedge A]\right)\right\rangle.\]
\end{example}

\subsubsection{Generalized Chern--Simons theory} \label{gen_CS_section}

The next class of examples of classical BV theories we give are generalizations of Chern-Simons theory. Unlike the example of the generalized BF theory, these theories are not of cotangent type.

\begin{definition}
Let $X$ and $Y$ be complex manifolds and $M$ a smooth manifold. Fix an $L_\infty$ algebra $\fg$. We assume $X$ is equipped with a holomorphic volume form $\Omega_X \in\Omega^{\dim(X), 0}(X)$ and $\fg$ is equipped with a nondegenerate invariant symmetric pairing $\langle-, -\rangle\colon \fg\otimes\fg\rightarrow \CC[\dim_\CC(X) + 2\dim_\CC(Y) + \dim(M) - 3]$. The {\bf generalized Chern--Simons theory} is the following classical BV theory:
\begin{itemize}
\item The spacetime is the smooth manifold $X\times Y\times M$.

\item The bundle of BV fields is the $\ZZ$-graded bundle $E = \Omega^{0,\bu}_X \otimes \Omega^{\bu,\bu}_Y \otimes \Omega^\bu_M \otimes \fg[1]$, where every field is considered even with respect to the $\ZZ/2$-grading.

\item $Q = \dbar_X + \dbar_Y + \d_{\dR, M} + \ell_1$.

\item The pairing $\omega\colon E\otimes E\rightarrow \Dens_M[-1]$ is given by the combination of the wedge product of differential forms, integration $\int_{X\times Y\times M} \Omega_X\wedge (-)$ and the pairing $\langle -, -\rangle$ on $\fg$.

\item The interaction term is
\[I = \sum_{k\geq 2}\frac{1}{(k+1)!} \int_{X\times Y\times M} \Omega_X\wedge \langle A\wedge \ell_k(A, \dots, A)\rangle.\]
\end{itemize}
\label{def:generalizedBF}
\end{definition}

We may also consider a $\ZZ/2$-graded version of the above theory where $\fg$ is merely $\ZZ/2$-graded.

\begin{example}
For $X=Y=\pt$, $M$ a 3-manifold and $\fg$ an ordinary Lie algebra we recover the usual 3-dimensional Chern--Simons theory with the BV action
\[S = \int_M \left(\frac{1}{2}\langle A\wedge \d_{\dR} A\rangle + \frac{1}{6}\langle A\wedge [A\wedge A]\rangle\right).\]
\end{example}

\begin{example}
For $Y=M=\pt$, $X$ a Calabi-Yau 3-fold and $\fg$ an ordinary Lie algebra we recover the holomorphic Chern--Simons theory with the BV action
\[S = \int_X \Omega_X\wedge \left(\frac{1}{2}\langle A\wedge \dbar A\rangle + \frac{1}{6}\langle A\wedge [A\wedge A]\rangle\right).\]
\end{example}

\begin{example}
If $\fh$ is an $L_\infty$ algebra, $\fg = \fh \oplus \fh^*[d-3]$ carries a natural $L_\infty$ structure given by combining the original $L_\infty$ structure on the first term and the coadjoint action of the first term on the second term. The $L_\infty$ algebra $\fg$ carries a natural symmetric pairing of degree $d-3$ given by the obvious pairing between $\fh$ and $\fh^*$. Generalized Chern--Simons theory for $\fg$ in this case recovers the generalized BF theory from Definition \ref{def:generalizedBF}.
\end{example}

\subsubsection{Super Yang--Mills Theory} \label{YM_section}

Let us now give an example of a theory which is not partially holomorphic. We will give a general description of the Yang--Mills theory on $M=\RR^n$ with adjoint spinorial matter. Fix a Lie algebra $\fg$ equipped with a nondegenerate symmetric bilinear pairing $\langle -, -\rangle$ and a spinorial representation $\Sigma$ of $\so(n; \CC)$ with an $\so(n)$-equivariant pairing $\Gamma \colon \Sym^2(\Sigma)\to \CC^n$ (see Section \ref{sec: susy} for more details on supersymmetry). The fields of super Yang--Mills theory are the following:
\begin{itemize}
\item A connection $A \in \Omega^1(\RR^n; \fg)$ on the trivial bundle.

\item A section $\lambda \in \Omega^0(\RR^n; \Pi \Sigma \otimes \fg)$.

\item The ghost field $c\in\Omega^0(\RR^n; \fg)[1]$.
\end{itemize}

Thus, the bundle of BRST fields is
\[F = (\Omega^1(M)\oplus \Omega^0(M; \Pi\Sigma)\oplus \Omega^0(M)[1])\otimes \fg.\]
The corresponding local super dg Lie algebra $L=F[-1]$ is
\[
L \;\;\; = \begin{array}{ccccc}
& \ul{0} & & \ul{1} & \\ 
& & & & \\
& \Omega^0(M; \gg) & \to & \Omega^1(M; \gg) \oplus \Omega^0(M; \Pi \Sigma \otimes \gg) & 
\end{array}
\]
whose differential is the de Rham differential $\Omega^0(M; \gg)\rightarrow \Omega^1(M; \gg)$ and the bracket is induced from the Lie bracket on $\gg$.

Denote by $F_A = \d_{\dR} A + \frac{1}{2}[A\wedge A]$ the curvature of $A$ and let $\sd D_A \colon \Sigma \to \Sigma^*$ be the Dirac operator obtained from $\Gamma$ (see Section \ref{sec: susy}). We denote by $\sd\d$ the Dirac operator for the zero connection.

The BRST action is given by
\[S(A, \lambda) = \int_M \left\langle \frac{1}{2} F_A \wedge \ast F_A - (\lambda, \sd D_A \lambda)\right\rangle.\]

The associated BV theory $(E, Q, \omega, I)$ is given as follows (see also \cite[Section 3.1]{ElliottYoo1}). Let $A^*, \lambda^*, c^*$ be the antifields corresponding to $A, \lambda, c$ respectively. The bundle of BV fields is
\[E = \Omega^1(M; \fg)\oplus \Omega^0(M; \Pi\Sigma\otimes \fg)\oplus \Omega^0(M; \fg)[1] \oplus \Omega^{n-1}(M; \fg^*)[-1]\oplus \Omega^n(M; \Pi\Sigma^*\otimes \fg^*)[-1]\oplus \Omega^n(M; \fg^*)[-2].\]
The pairing $\omega$ on $E$ is induced by the evaluation pairings $\gg^*\otimes \gg\rightarrow \CC$ and $\Sigma^*\otimes\Sigma\rightarrow\CC$. The BV action is given by
\[S_{BV} = \int_M \left\langle \frac{1}{2} F_A \wedge \ast F_A - (\lambda, \sd D_A \lambda)\right\rangle + (A^*, \d_A c) + (\lambda^*, [\lambda, c]).\]

We'll conclude this section with a discussion of the Poincar\'e invariance of classical Yang-Mills theory.

\begin{definition}
The $n$-dimensional {\bf Poincar\'e group} is the group $\mr{ISO}(n) = \SO(n) \ltimes \RR^n$ of isometries of $\RR^n$ with its flat Riemannian metric.  The $n$-dimensional {\bf Poincar\'e algebra} $\mf{iso}(n)$ is the Lie algebra of the Poincar\'e group.
\end{definition}

The $n$-dimensional Poincar\'e algebra acts, in the sense of Definition \ref{infinitesimal_action_def}, on Yang--Mills theory on $\RR^n$ for any choice $\Sigma$ of matter representation.
\chris{define the interaction functional $I_\mr{Poin}$ associated to the Poincar\'e action on Yang-Mills theory.}

\begin{prop} \label{YM_Poincare_invariant_prop}
Yang--Mills theory is \emph{Poincar\'e invariant}.  That is, the classical BV action functional satisfies the Maurer-Cartan equation
\[\{S_{BV}, I_{\mr{Poin}}\} + \d_{CE} I_{\mr{Poin}} + \frac 12 \{I_{\mr{Poin}}, I_{\mr{Poin}}\} = 0.\]
\end{prop}
\begin{proof}
\chris{...}
\end{proof}

\subsection{Dimensional Reduction, take 2}

\begin{definition}
A classical field theory $(E_N, Q_N, \omega_N, I_N)$ on a manifold $N$ is a {\bf dimensional reduction} of the classical field theory $(E_M, Q_M, \omega_M, I_M)$ on a manifold $M$ if we are given the following data:
\begin{itemize}
\item A submersion $p\colon M\rightarrow N$ equipped with a fiberwise volume form, i.e. an isomorphism $p^*\Dens_N\cong \Dens_M$.

\item An isomorphism $p^* E_N\cong E_M$.
\end{itemize}
These data are required to satisfy the following conditions:
\begin{itemize}
\item The diagram
\[
\xymatrix{
p^* E_N\otimes p^* E_N \ar^{\omega_N}[r] \ar^{\sim}[d] & p^*\Dens_N[-1] \ar^{\sim}[d] \\
E_M\otimes E_M \ar^{\omega_M}[r] & \Dens_M[-1]
}
\]
is commutative.

\item The diagram
\[
\xymatrix{
\cE_N \ar^{Q_N}[r] \ar^{p^*}[d] & \cE_N[1] \ar^{p^*}[d] \\
\cE_M \ar^{Q_M}[r] & \cE_M[1]
}
\]
is commutative.

\item Under the map $p^*\colon \cE_N\rightarrow \cE_M$ we have $p^* I_M = I_N$.
\end{itemize}
\end{definition}

\chris{A question: are dimensional reductions, per this definition, unique?  I'd really like the answer to be yes (at least for dimensional reduction along a vector space), so that I can say the twist of a dimensional reduction is equivalent to \emph{the} (unique) dimensional reduction of the twist -- assuming the dimensional reductions before and after twisting exist (which, again, I'd like them to, at least for dimensional reduction along a vector space).}

\subsection{Dimensional Reduction} \label{dim_red_section}
The idea of the \emph{dimensional reduction} of a rotation invariant classical field theory to a linear subspace of $\RR^n$ is to restrict attention to only those fields with are constant in directions perpendicular to the subspace. 

\begin{definition}
Let $i\colon \RR^m \inj \RR^n$ be the inclusion of a linear subspace, and let $p \colon \RR^n \to \RR^m$ be the corresponding orthogonal projection. Let $(E, \omega, Q, I)$ be a rotation invariant classical field theory on $\RR^n$.  The \emph{dimensional reduction} of the classical field theory from $\RR^n$ to $\RR^m$ is the classical field theory whose underlying bundle of fields is the pullback bundle $i^*E$, with the following structure.  The symplectic pairing is defined as the composite
\[(i^*\mc E \otimes i^*\mc E)(U) \iso (p^*i^*\mc E \otimes p^*i^*\mc E)(U \times (\RR^m)^\perp) \to (\mc E \otimes \mc E)(U \times (\RR^m)^\perp) \overset \omega \to \dens(U \times (\RR^m)^\perp),\]
with the observation that the density we obtain splits canonically into a density on $U$ and a constant density on $(\RR^m)^\perp$.  The classical differential $Q \colon i^*\mc E(U) \to i^*\mc E(U)$ is defined by
\[\phi \mapsto i^*(Q(p^*(\phi))).\]
Finally, the classical interaction is defined similarly: the pullback $p^*$ defines a map from $\mc O(\mc E)$ to $\mc O(i^*\mc E)$ that preserves locality \footnote{Intuitively, given a local functional on fields on $\RR^n$, restrict to a local functional on those fields constant in directions perpendicular to $\RR^m$.}: there is a canonical map defined on an open set $U \sub \RR^m$ by
\[\sym^k \mc E^\vee(U\times (\RR^m)^\perp) \to \sym^k p^*i^*\mc E^\vee(U\times (\RR^m)^\perp) \iso \sym^k i^*\mc E(U).\]
\end{definition}

Rotation invariance guarantees that the dimensional reduction is independent of the choice of $m$-dimensional subspace of $\RR^n$.

\begin{prop} \label{dim_red_SUSY_prop}
The dimensional reduction of a supersymmetric classical field theory $\mf L$ on $\RR^n$ with supersymmetry algebra $\mf A = (\so(n) \oplus \gg_R) \ltimes (\RR^n \oplus \Pi \Sigma)$ is a supersymmetric classical field theory $\mf L'$ on $\RR^m$ with supersymmetry algebra $\mf A' = (\so(m) \oplus \gg_R) \ltimes (\RR^m \oplus \Pi \Sigma')$, where $\Sigma'$ is the restriction of the representation $\Sigma$ of $\so(n)$ to a representation of the subalgebra $\so(m)$ of rotations of the subspace $\RR^m \sub \RR^n$.
\end{prop}

\begin{proof}
 \chris{Here's a proof sketch/outline/idea.} Start with a Poincar\'e invariant action functional $\mf S_{\mr{Poin}}$ in $n$-dimensions.  There is a cochain map \[F_{\mr{Poin}} \colon (C^\bullet_{\mr{Lie}}(\mf{iso}(n)) \otimes C^\bullet_{\mr{loc}}(\mf L))\to (C^\bullet_{\mr{Lie}}(\mf{iso}(m)) \otimes C^\bullet_{\mr{loc}}(p^* \mf L')),\] 
 given by the inclusion of $\mf{iso}(m) \inj \mf{iso}(n)$, and the restriction of a functional to act only those fields which are constant on $(\RR^m)^\perp$.   We'll check that this map will also be compatible with the antibracket $\{,\}$, so a Maurer Cartan element $\mf S_{\mr{Poin}}$ in $n$-dimensions induces a Maurer-Cartan element $F(\mf S_{\mr{Poin}}) = \mf S'_{\mr{Poin}}$ in $m$-dimensions, i.e a Poincar\'e invariant action functional. 
 
 Now include supertranslations, so start with an $\mf A$-invariant action functional $\mf S$ in $n$-dimensions.  There is a graded linear map $C^\bullet_{\mr{Lie}}(\mf A) \to C^\bullet_{\mr{Lie}}(\mf A')$, but it's not a cochain map because it's not compatible with the $\Gamma$-brackets.  Let's apply the graded linear map \[F \colon (C^\bullet_{\mr{Lie}}(\mf A) \otimes C^\bullet_{\mr{loc}}(\mf L))\to (C^\bullet_{\mr{Lie}}(\mf A') \otimes C^\bullet_{\mr{loc}}(p^*\mf L'))\] to $\mf S$ anyway.  We'll find that $\d F(\mf S) - F \d(\mf S)$ vanishes, because the failure of $F$ and $\d$ to commute involves the action of the translations perpendicular to $\RR^m$, which act trivially on the Poincar\'e invariant theory $\mf L'$.  We can then use the same argument as above. 
\end{proof}

\chris{we could alternatively write this in terms of BRST fields.}

\chris{By the way, I see no obstruction to proving the following.  One just needs to check that the action of an infinitesimal symmetry $X$ in $\mf A'$ on the dimensionally reduced theory is by $\phi \mapsto i^*(X(p^*(\phi)))$, so that forming the reduced classical differential commutes with twisting.  This should follow from the definition of the supersymmetry algebra action on the reduced theory from the proposition above.}
\begin{lemma} \label{commuting_twist_and_reduction_lemma}
If $\mf A$ is an $n$-dimensional supersymmetry algebra, and $Q \in \mf A$ is a square-zero supercharge, then the operations of dimensional reduction from $\RR^n$ to $\RR^m$ and twisting by $Q$ commute.
\end{lemma}

We'll conclude this section by discussing some examples of dimensional reduction in the context of generalized BF and Chern-Simons theories.  First we'll describe what happens when we dimensionally reduce generalized Chern-Simons theory along $\RR$ in a holomorphic direction.

\begin{lemma} \label{CS_diml_red_lemma}
Let $\map(X \times C_{\ol \dd}, B\gg)$ be a generalized Chern-Simons theory as in Section \ref{gen_CS_section}, where $C = \CC$ or $\CC^\times$ is a curve, and $X = (X_1)_{\ol \dd} \times (X_2)_{\mr{Dol}} \times (X_3)_{\mr{dR}}$ is arbitrary.  Split $C$ as $L \times \RR$, where $L = \RR$ or $S^1$.  The \chris{assuming uniqueness} dimensional reduction of this theory along the map $X \times C \to X \times L$ is equivalent to the generalized Chern-Simons theory $\map(X \times L_{\mr{dR}}, B\gg)$.
\end{lemma}

\chris{I haven't made the notation here match the notation in Section \ref{gen_CS_section} yet.}

\begin{proof}
\chris{sketchy}
Let us denote the complex of BV fields in our generalized Chern-Simons theory on $X \times C$ by $(E_X \boxtimes \bigwedge^\bullet (T^{0,1}_C)^\vee) \otimes \gg[1]$, with differential on the sheaf of sections given by $Q_X + \ol \dd_C + \ell_1$ and pairing on the sheaf of sections by $\omega_X \boxtimes \omega_C$. Write $p$ for the projection $C \to L$.  It's enough to observe that there is a canonical graded isomorphism $p^*\Omega^\bullet(L) \iso \Omega^{0,\bullet}(C)$ intertwining on the one hand the de Rham and Dolbeault differentials, and on the other hand the two integration maps against the orientation on $L$ and the Calabi-Yau structure on $C$.
\end{proof}

What about if we, instead, reduce generalized Chern-Simons theory in a de Rham direction?

\begin{lemma} \label{CS_to_BF_diml_red_lemma}
Consider a generalized Chern-Simons theory of the form $\map(X \times \RR_{\mr{dR}}, B\gg)$, where again $X$ is arbitrary.  The dimensional reduction of this theory along the map $X \times \RR \to X$ is equivalent to the generalized BF type theory $\map(X, T[1]B\gg)$.
\end{lemma}

\begin{proof}
\chris{...}
\end{proof}


\section{Supersymmetry} \label{sec: susy}

In this section we recall the framework for supersymmetry following \cite{ElliottSafronov} and \cite{DeligneSpinors}, we refer there for more details.

Let $V_\RR = \RR^n$ endowed with its nondegenerate bilinear form and $V=V_\RR\otimes_\RR\CC$ its complexification. Consider the Lie algebra $\so(V)$. Let us recall the following facts:
\begin{itemize}
\item If $n$ is odd, $\so(V)$ has a distinguished fundamental representation called the {\bf spin} representation $S$.

\item If $n$ is even, $\so(V)$ has a pair of distinguished fundamental representations called the {\bf semi-spin} representations $S_+, S_-$.
\end{itemize}

\begin{definition}
A {\bf spinorial representation} $\Sigma$ is a sum of spin or semi-spin representations of $\so(V)$.
\end{definition}

So, in odd dimensions we have $\Sigma=S\otimes W$ and in even dimensions we have $\Sigma=S_+\otimes W_+\oplus S_-\otimes W_-$, where $W$ denotes a multiplicity space.

\begin{definition}
Fix a spinorial representation $\Sigma$ and a nondegenerate $\so(V)$-equivariant pairing $\Gamma\colon \sym^2(\Sigma)\rightarrow V$. The {\bf supertranslation Lie algebra} is the $\so(V)$-equivariant super Lie algebra $T=\Pi\Sigma\oplus V$ whose only nontrivial bracket is given by $\Gamma$.
\end{definition}

For a given spinorial representation, the pairing $\Gamma$ is unique up to a scale, so a supertranslation Lie algebra is specified by fixing a spinorial representation. In turn, a spinorial representation is determined by the dimension of the multiplicity space, so we will talk about $\mc{N}$ or $(\mc{N}_+, \mc{N}_-)$ supertranslation Lie algebras, where the numbers are specified as follows:
\begin{itemize}
\item If $n\equiv 0, 1, 3, 4\pmod 8$, we let $\mc{N} = \dim(W)$.

\item If $n\equiv 2 \pmod 8$, we let $\mc{N}_{\pm}=\dim(W_{\pm})$.

\item If $n\equiv 5, 7\pmod 8$, we let $2\mc{N} = \dim(W)$.

\item If $n\equiv 6\pmod 8$, we let $2\mc{N}_{\pm} = \dim(W_{\pm})$.
\end{itemize}

Fix the following data:
\begin{itemize}
\item A spinorial representation $\Sigma$ of $\so(V)$.

\item An $\so(V)$-equivariant nondegenerate pairing $\Gamma\colon \sym^2(\Sigma)\rightarrow V$.

\item A complex Lie group $G_R$, the {\bf group of $R$-symmetries}, which is a subgroup of $\so(V)$-equivariant automorphisms of $(\Sigma, \Gamma)$.
\end{itemize}

Note that the supertranslation Lie algebra $T$ is a $\Spin(V_\RR)\times G_R$-equivariant super Lie algebra.

Consider a spacetime manifold $M$ which is an affine space over $V_\RR$. Let $\ISO(V_\RR) = \Spin(V_\RR)\ltimes V_\RR$ be the Poincar\'{e} group which acts in the obvious way on $M$.

\begin{definition}
A classical field theory $(E, Q, \omega, I)$ is {\bf supersymmetric} if $E\rightarrow M$ is an $\ISO(V_\RR)\times G_R$-equivariant vector bundle and the infinitesimal strict action of the translation Lie algebra $V$ on the classical theory is extended to a $\Spin(V_\RR)\times G_R$-equivariant $L_\infty$ action of the supertranslation Lie algebra $T$ on the classical theory.
\end{definition}

\subsection{Supersymmetric Yang-Mills Theory} \label{SUSY_action_section}
For minimal choices of the spinorial representation, Yang-Mills theory admits an action of the supersymmetry algebra when $n=3,4,6$ or 10.  In this subsection we will construct these supersymmetry algebra actions.  

In this section, we will consider pairs $(\mf A, \fL)$ consisting of a supersymmetry algebra and an instance of Yang-Mills theory with matter, in one of the following four situations.
\begin{itemize}
 \item Dimension $n=3$ with $\Sigma = S$, the 2-dimensional Dirac representation.
 \item Dimension $n=4$ with $\Sigma = S_+ \oplus S_-$, 4-dimensional Dirac representation.
 \item Dimension $n=6$ with $\Sigma = S_+ \otimes W$ where $W$ is a 2-dimensional symplectic vector space, the 8-dimensional symplectic Weyl representation.
 \item Dimension $n=10$ with $\Sigma = S_+$, the 16-dimensional Weyl representation.
\end{itemize}

\begin{definition}
In each case we will write $\wt \Sigma$ for the spin representation obtained from $\Sigma$ by parity reversal, so in dimensions 3 and 4 $\wt \Sigma = \Sigma$, in dimension 6 $\wt \Sigma = S_- \otimes W$, and in dimension 10 $\wt \Sigma = S_-$.
\end{definition}

We'll construct, in each of these cases, an $L_\infty$ action of the super Lie algebra $\mf A$ on the theory $\fL$.  This construction will proceed in two steps.
\begin{enumerate}
 \item There is an ordinary Lie action of the super Lie algebra $\mf A$ on the theory $\fL$, extending the standard action of the $n$-dimensional Poincar\'e algebra by isometries of $\RR^{n}$ as discussed in Section \ref{YM_section}, which is only well-defined on-shell.  In other words, there is a linear map $\delta^{(1)} \colon \mf A \to \cloc^\bu(\fL)[-1]$ which is a Lie action modulo the ideal generated by the equations of motion.
 \item We can promote this on-shell action to an $L_\infty$ map $\delta^{(1)} \colon \mf A \to \cloc^\bu(\fL)[-1]$ by including a quadratic term $\delta^{(2)}$ which ``corrects'' for the failure of the map $\delta^{(1)}$ to be a Lie map on-the-nose.
\end{enumerate}
This construction only works in the four special dimensions $3,4,6$ and 10: our argument will use results of Baez and Huerta \cite{BaezHuerta} proven using the structure of the four normed division algebras $\RR, \CC, \bb H$ and $\bb O$.  In particular, we will use the following ``3-$\psi$'s rule'', discussed in the 10-dimensional case by Schray \cite{Schray} .  Let $g(v,w)$ denote the Riemannian metric on $\RR^n$.

\begin{theorem}[{\cite[Theorem 11]{BaezHuerta}}] \label{3_psi_thm}
The multilinear map 
\[T \colon \Sigma \otimes \Sigma \otimes \Sigma \otimes \Sigma \to \CC,\]
obtained by symmetrizing the expression $g(\Gamma(\psi_1,\psi_2), \Gamma(\psi_3, \psi_4))$ in the first three variables, is zero.
\end{theorem}

We'll begin with a discussion of the on-shell supersymmetry action, discuss its failure to close off-shell, and then derive the $L_\infty$-correction to this failure.  Then we'll state and prove the main theorem of this section: the fact that our super Yang-Mills theories are supersymmetric.

So, we start by discussing the on-shell supersymmetry.  The fermionic part of the supersymmetry algebra is defined by saying that $Q \in \Sigma$ acts infinitesimally by
\[
\begin{pmatrix}
A \\ \lambda
\end{pmatrix}
\mapsto
\begin{pmatrix} A + \delta_Q A \\
\lambda + \delta_Q \lambda
\end{pmatrix}
\]
where 
\begin{align*}
\delta_Q A &= \Gamma(Q,\lambda) \\
\delta_Q \lambda &= \sd F_A Q .
\end{align*}
Here, the notation $\sd F_A$ stands for the iterated Clifford multiplication $\sd F_A = F_{ij} \gamma^i \gamma^j$.  We can check that this supersymmetry action does not close off-shell.  

\begin{lemma} \label{onshell_action_lemma}
Suppose $Q_1, Q_2 \in \Sigma$ and $(A, \lambda)$ are super Yang-Mills fields.
The following relations hold:
\begin{itemize}
\item[(1)] \label{10dsusyA} $ [\delta_{Q_1}, \delta_{Q_2}] A = \delta_{[Q_1, Q_2]} A$.
\item[(2)] \label{10dsusyL} $ [\delta_{Q_1}, \delta_{Q_2}] \lambda = \delta_{[Q_1,Q_2]} \lambda - \rho(\Gamma(Q_1,Q_2)) \sd \dd \lambda - \frac 12(Q_2, \sd \dd \lambda)Q_1 - \frac 12(Q_2, \sd \dd \lambda)Q_2$ .
\end{itemize}
Here, the commutator on the left hand side of the equations takes place in the algebra of endomorphisms of the space of fields.
\end{lemma}

We'll prove this lemma after first translating it into more convenient language: the terminology of the equivariant action functional, as introduced in Section \ref{symmetry_section}.  Our Lemma \ref{onshell_action_lemma} tells us that the supersymmetry action is a Lie algebra homomorphism only modulo the ideal generated by the equation of motion $\sd \dd \lambda = 0$.
In other words, this supersymmetry action only defined an action of the Lie algebra of supertranslations ``on-shell".  This calculation suggests introducing a second order correction to the supersymmetry action on the BV theory, which has the chance of closing off-shell.  Define a second order action depending on the antifield $\lambda^*$ to the gluino $\lambda$ by
\begin{align*}
\delta^{(2)} \colon \Sigma \otimes \Sigma \otimes \Gamma(\RR^{n}; \Pi \wt \Sigma[-1]) &\to \Gamma(\RR^{n}; \Pi \Sigma) \\
Q_1 \otimes Q_2 \otimes \lambda^* &\mapsto - \left(\rho(\Gamma(Q_1,Q_2)) \lambda^* + \frac 12 \left((Q_2, \lambda^*)Q_1 + (Q_1, \lambda^*)Q_2\right)\right).
\end{align*}

We'll rewrite this as a local functional in the theory $\fL$ coupled to the supersymmetry Lie algebra $\mf A$.

\begin{definition}
The off-shell supersymmetry action on complexified Yang-Mills theory on $\RR^{n}$, for $n=3,4,6,10$, is defined to be the cochain
\[I^{(1)} + I^{(2)} \in \clie^\bu(\mc A) \otimes \cloc^\bu(\fL),\]
where 
\begin{align*}
I^{(1)} (Q ; A, \lambda, A^*, \lambda^*) & = \<A^* , \Gamma(Q, \lambda)\> + \<\lambda^*, \sd F_A Q\> \\
I^{(2)} (Q_1,Q_2 ; \lambda^*) & =  \left\<\lambda^* \;,\; \rho(\Gamma(Q_1,Q_2)) \lambda^* + \frac 12 \left((Q_2, \lambda^*)Q_1 + (Q_1, \lambda^*)Q_2\right)\right\> .
\end{align*}
\end{definition}

We can rephrase Lemma \ref{onshell_action_lemma} in terms of these local functionals, using the interpretation where 
\begin{align*}
I^{(1)}(Q) &= \<A^* , \delta_{Q} A\> + \<\lambda^*, \delta_{Q_1} \lambda\> \\
\text{and } I^{(2)}(Q_1,Q_2) &= \<\lambda^*, \delta^{(2)}_{Q_1,Q_2}\lambda\>.
\end{align*}
Let us write $\d_{\mr{CE}}^\Gamma$ for the Chevalley-Eilenberg differential associated to the \emph{supertranslation algebra} $\CC^m \oplus \Pi \Sigma$, with bracket given by the pairing $\Gamma$.  We include the superscript $\Gamma$ here to distinguish this differential from the Chevalley-Eilenberg differential in the full super Poincar\'e algebra, which we will use below.

\begin{lemma} \label{onshell_interaction_lemma}
The functionals $I^{(1)}$ and $I^{(2)}$ above satisfy the equation
\[\{I^{(1)},I^{(1)}\} + \d_{\mr{CE}}^\Gamma I^{(1)} = -2\{S_{\mr{BV}} , I^{(2)}\}.\]
\end{lemma}

\chris{Pavel wants to replace these two proofs: first of Lemma \ref{onshell_action_lemma}, then of Lemma \ref{onshell_interaction_lemma}, with a single coordinate free argument.}
\begin{proof}[Proof of Lemma \ref{onshell_action_lemma}]
Both of the two statements are direct calculations using standard Clifford relations which cite below.
So, we calculate
\begin{align*}
[\delta_{Q_1}, \delta_{Q_2}] A &= (\Gamma(Q_2,\sd F_A Q_1) + \Gamma(Q_1,\sd F_A Q_2)) \\
&=  F_{ij}(Q_2 \gamma^k \gamma^j \gamma^i Q_1 + Q_1 \gamma^k \gamma^j \gamma^i Q_2) \\
&=  F_{ij}(Q_2 \gamma^k \gamma^j \gamma^i Q_1 + Q_2 \gamma^i \gamma^j \gamma^k Q_1)\\
&= F_{ij}(\delta^{jk}(Q_2\gamma^i Q_i) - Q_2 \gamma^i \gamma^j \gamma^k + Q_2 \gamma^i \gamma^j \gamma^k Q_1)\\
&=  F_{ij}\delta^{jk}(Q_2 \gamma^i Q_1) \\
&= \delta_{[Q_1, Q_2]} A,
\end{align*}
where on the third line we used the fact that the pairing $\Gamma(-,-)$ is symmetric -- i.e. that $\lambda_1 \gamma^i \lambda_2 = \lambda_2 \gamma^i \lambda_1$ -- and on the fourth line we used the Clifford relation $\gamma^j\gamma^j+\gamma^j\gamma^j = \delta^{jk}$.  Note that, on the gauge fields, the action is a Lie action on the nose, not only on-shell.  Similarly we can calculate, (as in \cite{Guillen}):
\begin{align*}
[\delta_{Q_1}, \delta_{Q_2}] \lambda &= (\sd F_{\Gamma(Q_2, \lambda)} Q_1 + \sd F_{\Gamma(Q_1,\lambda)} Q_2) \\
&= \frac 12((Q_2 (\gamma_j \dd_i - \gamma_i \dd_j) \lambda) (\gamma^i \gamma^j Q_1) + (1 \leftrightarrow 2)) \\
&= \frac 12((Q_2 \gamma_j \dd_i \lambda) \gamma^i \gamma^j Q_1 + (Q_1 \gamma_j \dd_i \lambda) \gamma^i \gamma^j Q_2) - \frac 12((Q_2 \gamma_i \dd_j \lambda) \gamma^i \gamma^j Q_1 + (Q_1 \gamma_i \dd_j \lambda) \gamma^i \gamma^j Q_2) \\
&= \frac 12((Q_2 \gamma_j \dd_i \lambda) \gamma^i \gamma^j Q_1 + (Q_1 \gamma_j \dd_i \lambda) \gamma^i \gamma^j Q_2) + \frac 12((Q_2 \gamma_i \dd_j \lambda) \gamma^j \gamma^j Q_1 + (Q_1 \gamma_i \dd_j \lambda) \gamma^j \gamma^i Q_2) + \\
&\quad - \frac 12((Q_2 \gamma_i \dd_j \lambda) \delta_{ij} Q_1 + \frac 12(Q_1 \gamma_i \dd_j \lambda) \delta^{ij} Q_2) \\
&= ((Q_1 \gamma_j Q_2) (\gamma^i \gamma^j \dd_i \lambda) - \frac 12(Q_2 \gamma_i \dd_i \lambda)Q_1 - \frac 12(Q_1 \gamma_i \dd_i \lambda)Q_2
\end{align*}
using the fact that 
\[(\psi_1 \gamma_j \psi_2)(\gamma^j \psi_3) + (\psi_2 \gamma_j \psi_3)(\gamma^j \psi_1) + (\psi_3 \gamma_j \psi_1)(\gamma^j \psi_2) = 0,\]
as in Theorem \ref{3_psi_thm}.  Making one more simplification using the Clifford relations, we have
\begin{align*}
[\delta_{Q_1}, \delta_{Q_2}] \lambda &= ((Q_1 \gamma_j Q_2) (\delta^{ij} \dd_i \lambda) - ((Q_1 \gamma_j Q_2) (\gamma^j \gamma^i \dd_i \lambda) - \frac 12(Q_2 \gamma_i \dd_i \lambda)Q_1 - \frac 12(Q_1 \gamma_i \dd_i \lambda)Q_2 \\
&= \delta_{[Q_1,Q_2]} \lambda - \rho(\Gamma(Q_1,Q_2)) \sd \dd \lambda - \frac 12(Q_2, \sd \dd \lambda)Q_1 - \frac 12(Q_2, \sd \dd \lambda)Q_2.
\end{align*}
\end{proof}

\begin{proof}[Proof of Lemma \ref{onshell_interaction_lemma}]
The functional $\{I^{(1)},I^{(1)}\}$ can be computed as 
\begin{align*}
\{I^{(1)},I^{(1)}\} &= \{\<A^* , \delta_{Q_1} A\> + \<\lambda^*, \delta_{Q_1} \lambda\>, \<A^* , \delta_{Q_2} A\> + \<\lambda^*, \delta_{Q_2} \lambda\> \} \\
&= 2(\langle A^*, [\delta_{Q_1}, \delta_{Q_2}] A \rangle + \langle \lambda^*, [\delta_{Q_1}, \delta_{Q_2}] \lambda \rangle) \\
 &= 2(\langle A^*, \delta_{[Q_1,Q_2]} A \rangle + \langle \lambda^*, \delta_{[Q_1,Q_2]} \lambda \rangle  - \rho(\Gamma(Q_1,Q_2)) \sd \dd \lambda - \frac 12(Q_1, \sd \dd \lambda)Q_2 - \frac 12(Q_2, \sd \dd \lambda)Q_1) \\
 &= 2\langle A^*, \delta_{[Q_1,Q_2]} A \rangle + 2\langle \lambda^*, \delta_{[Q_1,Q_2]} \lambda \rangle - 2\{S_{\mr{BV}} , I^{(2)}\} \\
 &= - \d_{\mr{CE}}^\Gamma I^{(1)} -2\{S_{\mr{BV}} , I^{(2)}\},
\end{align*}
where we applied Lemma \ref{onshell_action_lemma} on the third line. \chris{note, we should be careful to make sure that the sign in the $\d_{\mr{CE}}^\Gamma I^{(1)}$ term is correct.}
\end{proof}

The functionals $I^{(1)}$ and $I^{(2)}$ are also rotation invariant, in the following sense.  Let $\d^{\mr{rot}}_{\mr{CE}}$ denote the Chevalley Eilenberg differential in the super Lie algebra $\so(n) \ltimes \Pi \Sigma$.  The total Chevalley-Eilenberg differential, applied to a functional $F$ on the purely fermionic part of the super Poincar\'e algebra, is therefore the sum
\[\d_{\mr{CE}}(F) = \d^\Gamma_{\mr{CE}}(F) + \d^{\mr{rot}}_{\mr{CE}}(F).\]

\begin{lemma} \label{SUSY_rotation_invariance_lemma}
Let $I_{\mr{rot}}$ be the interaction associated to the action of the Lie algebra $\mf{so}(n)$ on super Yang-Mills theory from Theorem \ref{YM_Poincare_invariant_prop}.  The functionals $I^{(1)}$ and $I^{(2)}$ are Poincar\'e invariant, meaning that
\begin{equation}\label{rot_invariance_eqn}
 \d^{\mr{rot}}_{\mr{CE}} I^{(i)} = - \{I_{\mr{rot}}, I^{(i)}\},
\end{equation}
for $i=1,2$.
\end{lemma}

\begin{proof}
Let $X$ be an element of $\so(n)$, and write $X \cdot A$ and $X \cdot \lambda$ for the image of the fields $A$ and $\lambda$ under the action of the symmetry $X$.  In particular, we can write
\[I_{\mr{rot}}(A,\lambda, A^\vee, \lambda^\vee, X) = \langle A^\vee, X \cdot A \rangle + \langle \lambda^\vee, X \cdot \lambda \rangle.
\]
Therefore, the operator $\{I_{\mr{rot}}(X), -\}$ on the space of local functionals is freely generated by the operator on linear functionals in the Yang-Mills BV fields given by the action of the rotation $X$.  On the other hand, the Chevalley-Eilenberg differential $\d^{\mr{rot}}_{\mr{CE}}$ is freely generated by the operator on \emph{supertranslations} in $\Sigma$, sending $Q$ to $X \cdot Q$.

So, for the functionals at hand.  We'll use three properties of the action of $\so(n)$: that -- for spinors $Q_1$ and  $Q_2$, and vectors $v$, we have that $X \cdot \Gamma(Q_1, Q_2) = \Gamma(X \cdot Q_1, Q_2) + \Gamma(Q_1, X \cdot Q_2)$,  $(X \cdot Q_1, Q_2) + (Q_1, X \cdot Q_2) = 0$, and that $X \cdot (\rho(v)Q) = \rho(X \cdot v)Q + \rho(v)(X \cdot Q)$.  These identities all follow from the equivariance of the pairing $\Gamma$, the scalar pairing $(,)$, and the Clifford multiplication $\rho$.  Using these facts, we compute
\begin{align*}
\{I_{\mr{rot}}, I^{(1)}\}(A,\lambda, A^\vee, \lambda^\vee, Q, X) &= \<X \cdot A^* , \Gamma(Q, \lambda)\> + \<A^* , \Gamma(Q, X \cdot \lambda)\> + \<X \cdot \lambda^*, \sd F_A Q\> +  \<\lambda^*, \sd F_{X \cdot A} Q\> \\
&= -\big(A^* , X \cdot \Gamma(Q, \lambda)\> - \<A^* , \Gamma(Q, X \cdot \lambda)\>\big) - \big(\<\lambda^*, X \cdot( \sd F_A Q)\> -  \<\lambda^*, \sd F_{X \cdot A} Q\>\big)\\
&= - \<A^* , \Gamma(X \cdot Q, \lambda)\> - \<\lambda^*, \sd F_A (X \cdot Q)\>\\
&= -\d^{\mr{rot}}_{\mr{CE}} I^{(1)}(A,\lambda, A^\vee, \lambda^\vee, Q, X),
\end{align*}
and
\begin{align*}
\{I_{\mr{rot}}, I^{(2)}\}(\lambda^\vee, Q_1, Q_2, X) &= \left\<X \cdot \lambda^* \;,\; \rho(\Gamma(Q_1,Q_2)) \lambda^* + \frac 12 \left((Q_2, \lambda^*)Q_1 + (Q_1, \lambda^*)Q_2\right)\right\> + \\
&\quad + \left\<\lambda^* \;,\; \rho(\Gamma(Q_1,Q_2)) X \cdot \lambda^* + \frac 12 \left((Q_2, X \cdot \lambda^*)Q_1 + (Q_1, X \cdot \lambda^*)Q_2\right)\right\> \\
&= - \left\<\lambda^* \;,\; X \cdot(\rho(\Gamma(Q_1,Q_2)) \lambda^*) + \frac 12 \left((Q_2, \lambda^*)X \cdot Q_1 + (Q_1, \lambda^*)X \cdot Q_2\right)\right\> + \\
&\quad + \left\<\lambda^* \;,\; \rho(\Gamma(Q_1,Q_2)) X \cdot \lambda^* - \frac 12 \left((X \cdot Q_2, \lambda^*)Q_1 - (X \cdot Q_1,  \lambda^*)Q_2\right)\right\> \\
&= - \left\<\lambda^* \;,\; \rho(\Gamma(X \cdot Q_1,Q_2)) \lambda^* + \frac 12 \left((Q_2, \lambda^*)X \cdot Q_1 + (X \cdot Q_1, \lambda^*)Q_2\right)\right\> - (1 \leftrightarrow 2)\\
&= -\d^{\mr{rot}}_{\mr{CE}} I^{(2)}(\lambda^\vee, Q_1, Q_2, X).
\end{align*}
\end{proof}

Now that we've defined the supersymmetry action, we can prove the main theorem of this section: that minimal super Yang-Mills theory in dimensions $n=3,4,6$ and 10 is, indeed, a supersymmetric classical field theory.

\begin{theorem} \label{SUSY_YM_theorem}
Let $S_{\mr{BV}}$ be the BV action functional in the theory $\fL$, and write $I_{\mr{Poin}}$ for the interaction generating the Lie action of the Poincar\'e algebra. The functional
\[\fS = S_{\mr{BV}} + I_{\mr{Poin}} + I^{(1)} + I^{(2)} \in \clie^\bu(\mf A) \otimes \cloc^\bu(\fL) [-1]\]
satisfies the Maurer-Cartan equation
\begin{equation} 
\label{nd_MC}
(\d_{\rm Lie} \fS + \frac{1}{2} \left\{\fS , \fS \right\}) = 0 .
\end{equation}
\end{theorem}

\begin{proof}
We can filter the complex $\clie^\bu(\mf A) \otimes \cloc^\bu(\fL)$ by the number of anti-field components in $\fL$.  Using this filtration, the left-hand side of \ref{nd_MC} splits up as
\begin{align}
\d_{\rm Lie}  \left( \fS \right) + \frac{1}{2} \left\{ \fS , \fS \right\} &= \d_{\mr{CE}} I_{\mr{Poin}} + \frac 12 \{S_{\mr{BV}} + I_{\mr{Poin}}, S_{\mr{BV}} + I_{\mr{Poin}}\} \label{MC_BV}\\
&\quad + \{S_{\mr{BV}}, I^{(1)}\} \label{MC_1}\\
&\quad + \d_{\mr{CE}} I^{(1)} + \{I_{\mr{Poin}}, I^{(1)}\} + \{S_{\mr{BV}}, I^{(2)}\} + \frac 12 \{I^{(1)}, I^{(1)}\} \label{MC_2}\\
&\quad + \d_{\mr{CE}} I^{(2)} + \{I_{\mr{Poin}}, I^{(2)}\} + \{I^{(1)}, I^{(2)}\} \label{MC_3}\\
&\quad + \frac 12 \{I^{(2)}, I^{(2)}\}. \label{MC_4}
\end{align}
Taking these terms one at a time, we first note that the vanishing of the first term, \ref{MC_BV}, is just the fact that the BV action of the 10d super Yang-Mills theory satisfies the classical master equation, and is Poincar\'e invariant, as in Proposition \ref{YM_Poincare_invariant_prop}. The remaining terms are those involving the action of supersymmetries.

First, the term \ref{MC_1} vanishes by \cite[Proposition 14]{BaezHuerta}.  Baez and Huerta prove that in our four critical dimensions, the variation $\delta^{(1)}S_{\mr{BRST}}$ of the ordinary BRST action functional vanishes.  In particular, this implies that $\delta^{(1)}S_{\mr{BV}} = \{I^{(1)}, S_{\mr{BV}}\}$ also vanishes, since \chris{why?}.

In the next term, \ref{MC_2}, the first two summands vanish by Lemma \ref{SUSY_rotation_invariance_lemma}, and the last two summands vanish by Lemma \ref{onshell_interaction_lemma}.  Likewise, in term \ref{MC_3}, the first two summands vanish by Lemma \ref{SUSY_rotation_invariance_lemma}, and the last term vanishes by applying the 3-$\psi$ rule, Theorem \ref{3_psi_thm}.  Indeed, we expand 
\begin{align*}
 \{I^{(1)}, I^{(2)}\}  &= \left\{\langle A^* , \Gamma(Q_3, \lambda)\rangle, \left\langle\lambda^* \;,\; \rho(\Gamma(Q_1,Q_2)) \lambda^* + \frac 12 \left((Q_2, \lambda^*)Q_1 + (Q_1, \lambda^*)Q_2\right)\right\rangle\right\}\\
 &= \left\langle A^*\;,\; \Gamma\left(Q_3, \rho(\Gamma(Q_1,Q_2)) \lambda^* + \frac 12 \left((Q_2, \lambda^*)Q_1 + (Q_1, \lambda^*)Q_2\right)\right)\right\rangle + (123) \\
 &= \left\langle A^*\;,\; \Gamma(Q_3, \rho(\Gamma(Q_1,Q_2)) \lambda^* + (Q_2, \lambda^*)Q_1)\right\rangle + (123),
\end{align*}
where $(123)$ indicates symmetrization in the supercharges $Q_1, Q_2, Q_3$.  We can simplify this expression by manipulating the first summand.  Indeed,
\begin{align*}
\left\langle A^*\;,\; \Gamma(Q_3, \rho(\Gamma(Q_1,Q_2)) \lambda^*)\right\rangle &= - \left\langle A^*\;,\; \Gamma(\lambda^*, \rho(\Gamma(Q_1,Q_2)) Q_3)\right\rangle - (\lambda^*, Q_3) g(A^*, \Gamma(Q_1,Q_2)).
\end{align*}
\chris{needs an argument, and care with signs.  Probably with the coordinate free argument above we'll have to include some lemmas on spinor manipulation that we can reference here.}  Thus, after symmetrization, we have shown that
\[\{I^{(1)}, I^{(2)}\}= - \left\langle A^*\;,\; \Gamma(\lambda^*, \rho(\Gamma(Q_1,Q_2)) Q_3)\right\rangle + (123),\]
which vanishes by Theorem \ref{3_psi_thm}.

Finally, term \ref{MC_4} vanishes because the interaction $I^{(2)}$ only involves antifields: the BV antibracket of two expressions involving only antifields is always zero.
\end{proof}

We can deduce supersymmetry for other super Yang-Mills theories using dimensional reduction, specifically Proposition \ref{dim_red_SUSY_prop}.
\begin{corollary}
Let $n$ be one of the special dimensions $3,4,6$ or 10, and choose $m < n$.  Let $\Sigma'$ be the $\so(m)$ representation obtained by restriction of the $\so(n)$ representation $\Sigma$.  Then the super Poincar\'e algebra
\[\mf A' = \mf{iso}(m) \oplus \Pi \Sigma',\]
obtained by restricting the $n$-dimensional super Poincar\'e algebra associated to $\Sigma$, acts on the $m$-dimensional super Yang-Mills theory $\mf L'$ with matter representation $\Sigma'$.
\end{corollary}

\begin{proof}
This follows immediately by applying Proposition \ref{dim_red_SUSY_prop} to the $\mf A$-supersymmetric theory $\mf L$ on $\RR^n$.
\end{proof}

\begin{remark}
We can further include the action of at least a subalgebra of the algebra of R-symmetries in each example. \chris{quantify which}
\end{remark}

\chris{Pavel points out that we would like to know more, namely supersymmetry for theories with supersymmetric matter.  In order to prove this, we have to check :
\begin{enumerate}
 \item That the 4d chiral and 6d hyper multiplet are supersymmetric.
 \item That the coupling terms $I_{\mr{coupling}}$ between these multiplets and the vector multiplet satisfy $\{I_{\mr{coupling}}, I^{(i)}\} = 0$ for $i=1,2$ (the $I^{(i)}$ split into a term describing the action on the vector as above, plus a term describing the action on the matter, and in the bracket these must cancel.)
\end{enumerate}
This implies the same after dimensional reduction, which tells us supersymmetry for the remaining examples: 5d, 3d, and 2d with matter.}

\subsection{Supersymmetric twisting}

\begin{definition}
A {\bf square-zero supercharge} is a nonzero element $Q\in\Sigma$ such that $\Gamma(Q, Q)=0$.
\end{definition}

It is shown in \cite[Proposition 3.25]{ElliottSafronov} that the image of $\Gamma(Q, -)\colon \Sigma\rightarrow V$ has dimension at least $n/2$. We will use the following adjectives for square-zero supercharges depending on $d=\dim(\mathrm{im}\Gamma(Q, -))$:
\begin{itemize}
\item A supercharge $Q$ is {\bf topological} if $d = n$.

\item A supercharge $Q$ is {\bf holomorphic} if $n$ is even and $d=n/2$.
\end{itemize}

In the intermediate case we refer to $Q$ as a {\bf holomorphic-topological} (alternatively, partially topological) supercharge. The collection of all square-zero supercharges in dimensions 2 through 10 (where one restricts to supersymmetries with at most 16 supercharges) was studied in \cite{ElliottSafronov} and \cite{EagerSaberiWalcher}. In particular, orbits of square-zero supercharges under the $R$-symmetry group, $\Spin(V)$ and the obvious scaling action of $\CC^\times$ are shown in Figure \ref{fig:superchargeorbits}:
\begin{itemize}
\item The color denotes the amount of supersymmetries: red denotes 16 supercharges, orange 8 supercharges, yellow 4 supercharges and green 2 supercharges. Dashed border denotes chiral supersymmetry (i.e. $(\cN_+, 0)$ in even dimensions).

\item $d$ denotes the dimension of the image of $\Gamma(Q, -)$.

\item Rank denotes the rank of the tensor $Q\in S\otimes W$ in odd dimensions or $Q\in S_+\otimes W_+\oplus S_-\otimes W_-$ in even dimensions.

\item Arrows denote compactifications from higher to lower dimensions.
\end{itemize}

\begin{definition}
A {\bf twisting datum} is a pair $(Q, \alpha)$, where $Q\in\Sigma$ is a square-zero supercharge and $\alpha\colon U(1)\rightarrow G_R$ is a homomorphism under which $Q$ has weight $1$.
\end{definition}

We will call $\alpha$ in a twisting datum a {\bf twisting homomorphism}.

\begin{definition}
Suppose $(E, \d, \omega, I)$ is a supersymmetric classical field theory and $(Q, \alpha)$ is a twisting datum. The {\bf $Q$-twisted classical field theory} is the classical field theory $(E^Q, \d + \{\rho_2(Q), -\}, \omega, I + \rho_3(Q) + \dots)$, where
\[E^Q = \bigoplus_{n=-\infty}^\infty \Pi^n E(n)[-n]\]
for $E(n)$ the component of $E$ which has $\alpha$-weight $n$.
\end{definition}

\begin{prop}
The collection $(E^Q, \d + \{\rho_2(Q), -\}, \omega, I + \rho_3(Q) + \dots)$ is a classical field theory.
\end{prop}
\pavel{Need to explain that the degrees work out and check that you still have an elliptic complex.}

\begin{remark}
Without a twisting homomorphism $\alpha$, we obtain a $\ZZ/2$-graded classical field theory
\[(E, \d + \{\rho_2(Q), -\}, \omega, I + \rho_3(Q) + \dots),\]
where the $\ZZ/2$-grading on $E$ is the total grading.
\end{remark}

\subsection{Twisting Homomorphisms and Curved Backgrounds}


\section{Twists}

%%\begin{landscape}
%\begin{figure}
%\begin{tikzpicture}[node distance=0.35cm and 0.35cm, text width=1.8cm]
%\node (101) [s16chiral] {Rank $(1, 0)$ $d=5$};
%\node (91) [s16, below=of 101] {Rank $1$ $d=5$};
%\node (82) [s16, below=of 91] {Rank $(1, 1)$ $d=5$};
%\node (81) [s16, left=of 82] {Rank $(1, 0)$ $d=4$};
%\node (825) [right=of 82] {};
%\node (83) [s16, right=of 825] {Rank $(1, 0)$ $d=8$};
%\node (71) [s16, below=of 81] {Rank $1$ $d=4$};
%\node (72) [s16, below=of 82] {Rank $2$ $d=5$};
%\node (73) [s16, below=of 83] {Rank $1$ $d=7$};
%\node (62) [s16, below=of 71] {Rank $(1, 1)$ $d=4$};
%\node (61) [s8, left=of 62] {Rank $(1, 0)$ $d=3$};
%\node (63) [s16, below=of 72] {Rank $(2, 2)$ $d=5$};
%\node (64) [s16chiral, right=of 63] {Rank $(2, 0)$ $d=5$};
%\node (65) [s16, below=of 73, right=of 64] {Rank $(1, 1)$ $d=6$};
%\node (51) [s8, below=of 61] {Rank $1$ $d=3$};
%\node (52) [s16, below=of 62] {Rank $2$ $d=4$};
%\node (53) [s16, below=of 63] {Rank $4$ $d=5$};
%\node (54) [s16, below=of 64] {Rank $2$ $d=5$};
%\node (42) [s8, below=of 51] {Rank $(1, 1)$ $d=3$};
%\node (41) [s4, left=of 42] {Rank $(1, 0)$ $d=2$};
%\node (43) [s16, right=of 42] {Rank $(2, 2)$ $d=4$};
%\node (44) [s16, right=of 43] {Rank $(2, 1)$ $d=4$};
%\node (45) [s8, right=of 44] {Rank $(2, 0)$ $d=4$};
%\node (31) [s4, below=of 41] {Rank $1$ $d=2$};
%\node (32) [s8, below=of 42] {Rank $2$ $d=3$};
%\node (22) [s4, below=of 31] {Rank $(1, 1)$ $d=2$};
%\node (21) [s2chiral, left=of 22] {Rank $(1, 0)$ $d=1$};
%
%\draw[arrow] (101) -- (91);
%\draw[arrow] (91) -- (81);
%\draw[arrow] (91) -- (82);
%\draw[arrow] (81) -- (71);
%\draw[arrow] (82) -- (71);
%\draw[arrow] (82) -- (72);
%\draw[arrow] (83) -- (73);
%\draw[arrow] (71) -- (61);
%\draw[arrow] (71) -- (62);
%\draw[arrow] (72) -- (62);
%\draw[arrow] (72) -- (63);
%\draw[arrow] (73) -- (65);
%\draw[arrow] (61) -- (51);
%\draw[arrow] (62) -- (51);
%\draw[arrow] (62) -- (52);
%\draw[arrow] (63) -- (53);
%\draw[arrow] (64) -- (52);
%\draw[arrow] (64) -- (54);
%\draw[arrow] (65) -- (54);
%\draw[arrow] (51) -- (41);
%\draw[arrow] (51) -- (42);
%\draw[arrow] (52) -- (42);
%\draw[arrow] (52) -- (43);
%\draw[arrow] (53) -- (43);
%\draw[arrow] (54) -- (43);
%\draw[arrow] (54) -- (44);
%\draw[arrow] (54) -- (45);
%\draw[arrow] (41) -- (31);
%\draw[arrow] (42) -- (31);
%\draw[arrow] (42) -- (32);
%\draw[arrow] (43) -- (32);
%\draw[arrow] (44) -- (32);
%\draw[arrow] (45) -- (32);
%\draw[arrow] (31) -- (21);
%\draw[arrow] (31) -- (22);
%\draw[arrow] (32) -- (22);
%
%\node (2d) [dimension, left=of 21] {2d};
%\node (3d) [dimension, above=of 2d] {3d};
%\node (4d) [dimension, above=of 3d] {4d};
%\node (5d) [dimension, above=of 4d] {5d};
%\node (6d) [dimension, above=of 5d] {6d};
%\node (7d) [dimension, above=of 6d] {7d};
%\node (8d) [dimension, above=of 7d] {8d};
%\node (9d) [dimension, above=of 8d] {9d};
%\node (10d) [dimension, above=of 9d] {10d};
%\end{tikzpicture}
%\caption{Orbits of square-zero supercharges.}
%\label{fig:superchargeorbits}
%\end{figure}
%%\end{landscape}

\subsection{Standard Equivalences Between Classical Field Theories}
In this section, we will collect some standard lemmas that we'll use to simplify the descriptions of twisted supersymmetric field theories below.

\begin{lemma} \label{symplectomorphism_lemma}
A linear symplectomorphism $F \colon \mc E^\bullet \to \mc E^\bullet$ induces an equivalence of theories between $(\mc E^\bullet, Q, \omega, I)$ and $(\mc E^\bullet, F^*Q, \omega, F^*I)$.
\end{lemma}

\brian{I may be optimistic, but shouldn't the lemmas below all follow from the fact that the category we are working with is a Grothendieck abelian category? In any case, I do like that we've stated them clearly.} \chris{I'm not sure what you have in mind (for instance I'm not sure why you would need the Grothendieck condition there), but I think the lemmas below should be formally immediate in this dg category of differentiable cochain complexes setting (like the first one follows in any context where you have an exact sequence $0 \to A \to B \to C \to 0$ and the fact that if $C$ is equivalent to 0 then $A \to B$ is an equivalence, or the same with $A$ and $B \to C$.)}

\begin{lemma} \label{inclusion_and_projection_lemma}
Let $(\mc E^\bullet, Q)$ be a cochain complex, and let $(\mc C^\bullet, Q')$ be a contractible cochain complex.
\begin{enumerate}
 \item Let $F \colon \mc C^\bullet \to \mc E^\bullet$ be a degree 1 map making $(\mc C^\bullet \overset F\to \mc E^\bullet)$ into a cochain complex.  Then the canonical inclusion $\mc E^\bullet \to (\mc C^\bullet \overset F\to \mc E^\bullet)$ is a quasi-isomorphism.
 \item Let $F' \colon \mc E^\bullet \to \mc C^\bullet$ be a degree 1 map making $(\mc E^\bullet \overset {F'}\to \mc C^\bullet)$ into a cochain complex.  Then the canonical projection $(\mc E^\bullet \overset {F'}\to \mc C^\bullet) \to \mc E^\bullet$ is a quasi-isomorphism.
\end{enumerate}
\end{lemma}

\begin{lemma} \label{symplectic_composite_lemma}
Let $(\mc E^\bullet,Q)$ be a cochain complex, let $(\mc C_1^\bullet, Q_1)$ and $(\mc C_2^\bullet, Q_2)$ be contractible cochain complexes, and let $\mc C_1^\bullet \overset{F_1}\to \mc E^\bullet \overset{F_2}\to \mc C_2^\bullet$ be a pair of degree 1 maps so that the differential $Q + Q_1 + Q_2 + F_1 + F_2$ on the total complex squares to 0.  Suppose the graded vector space $\mc E^\bullet \oplus \mc C_1^\bullet \oplus C_2^\bullet$ is equipped with a $-1$-shifted symplectic structure so that $\mc C_1^\bullet \oplus \mc C_2^\bullet$ is a symplectic subspace.  Then the cochain map 
\begin{equation}
\label{symp_composite_eqn}\mc E^\bullet \to (\mc C_1^\bullet \overset{F_1}\to \mc E^\bullet \overset{F_2}\to \mc C_2^\bullet)
\end{equation}
obtained as the composite of the projection from Lemma \ref{inclusion_and_projection_lemma} (2) with a quasi-inverse to the inclusion from Lemma \ref{inclusion_and_projection_lemma} (1) is a symplectomorphism.
\end{lemma}

\begin{lemma} \label{interaction_pullback_lemma}
In the set-up of Lemma \ref{symplectic_composite_lemma}, suppose we're given an interaction $I$ on the graded vector space $\mc E^\bullet \oplus \mc C_1^\bullet \oplus \mc C_2^\bullet$, which pulls back to $I'$ under the inclusion of $\mc E^\bullet$.  Suppose that all monomial summands of the interaction $I$ which depend on fields in $\mc C_2^\bullet$ also depend on fields in $\mc C_1^\bullet$. Then the map \ref{symp_composite_eqn} is compatible with the interactions $I$ and $I'$
\end{lemma}

\begin{proof}
We defined a quasi-isomorphism of cochain complexes in Lemma \ref{symplectic_composite_lemma} to be the composite $F$ of the inclusion $i$ of $\mc E^\bullet \oplus \mc C_2^\bullet$ with a quasi-inverse to the projection onto $\mc E^\bullet$: this composite map is a twisted inclusion $\mc E^\bullet \to \mc E^\bullet \oplus \mc C_1^\bullet \oplus \mc C_2^\bullet$ of the form $F \colon \phi \mapsto (\phi, 0, f(\phi))$ for some linear map $f$.  Because, by the hypothesis, all monomial summands of $I$ involving fields in $\mc C_2^\bullet$ also include fields in $\mc C_1^\bullet$, the pullback of $I$ under the map $F$ coincides with the pullback of $I$ under the inclusion map $\phi \mapsto (\phi, 0, 0)$ as required.
\end{proof}

\subsection{A-Type Twists} \label{A_twist_section}
\chris{...}

\subsection{Dimension 10}
We'll begin our discussing of twists of super Yang-Mills theory by studying the twist of 10-dimensional $\mc N=(1,0)$ super Yang-Mills theory, with the supersymmetry action we analyzed in Section \ref{SUSY_action_section}.  In the 10d $\mc N=(1,0)$ supersymmetry algebra there is a unique $\Spin(10)$ orbit of non-trivial square-zero supercharges given by the locus of pure spinors.  These square-zero supercharges are holomorphic.

Fix a non-trivial pure spinor $Q$, or equivalently, fix a Calabi-Yau structure on $\RR^{10}$.  The stabilizer of $Q$ in $\Spin(10)$ is isomorphic to $\SU(5)$.  Let us first, therefore, decompose the component fields of 10d super Yang-Mills theory into sections of the associated bundles to irreducible representations of $\SU(5)$.  The BRST fields split as follows:
\begin{align*}
c &\mapsto c \in \Omega^0(\CC^5; \gg) \\
A &\mapsto A_{0, 1} + A_{1, 0} \in \Omega^{0,1}(\CC^5; \gg) \oplus \Omega^{1,0}(\CC^5; \gg)\\
\lambda &\mapsto \chi + \psi + B \in \Omega^0(\CC^5; \gg) \oplus \Omega^{1,0}(\CC^5; \gg) \oplus \Omega^{0,2}(\CC^5; \gg). 
\end{align*}

In terms of these component fields, the BV action functional can be written in the following way.  We'll split the action functional up into the BRST action and the antifield action.
\begin{align*}
S_{\mr{BRST}} &= \int \d^5z \left(\Lambda^2 \langle F_{0,2} \wedge F_{2,0}\rangle + \frac 12 |\Lambda F_{1,1}|^2 + \Lambda( \chi \wedge (\ol \dd_{A_{0,1}} \psi))  + \Lambda^2(B \wedge (\dd_{A_{1,0}} \psi))\right) \Omega + (B \wedge \ol \dd_{A_{0,1}} B) \\
S_{\mr{anti}} &= \int \d^5z \langle \dd_{A_{1,0}}c, A_{1,0}^\vee \rangle +  \langle \ol \dd_{A_{0,1}}c, A_{0,1}^\vee \rangle + \langle [c,c], c^\vee \rangle + \langle [\chi,c], \chi^\vee \rangle + \langle [\psi,c], \psi^\vee \rangle + \langle [B,c], B^\vee \rangle,
\end{align*}
where $F_{i,j}$ is the $(i,j)$-form component of the curvature of the gauge field $A_{0, 1} + A_{1, 0}$, and $\Omega$ is the Calabi-Yau $(0,5)$-form.  Similarly, we can write explicitly the $L_\infty$ interaction functional associated to the action of the square 0 supercharge $Q$.  It has a quadratic and a cubic component given by
\begin{align*}
I^{(1)} &= \int \langle A_{1,0}^\vee, \psi \rangle + \langle \chi^\vee \Lambda F_{1,1} \rangle + \langle B^\vee, F_{0,2} \rangle \\
I^{(2)} &= \frac 12 \int \d^5z \Omega |\chi^\vee|^2.
\end{align*}
The twisted action functional is obtained by adding these terms to the original BV action functional.

We can now calculate the 10d holomorphic twist.  This calculation is originally due to Baulieu \cite{Baulieu}.

\begin{theorem}\label{10d_twist_thm}
The holomorphic twist of 10d super Yang-Mills theory on a Calabi-Yau 5-fold $X_5$ is equivalent to holomorphic Chern-Simons theory on $X_5$.
\end{theorem}

\begin{remark}
From the point of view of supersymmetry, the twisted theory is only $\ZZ/2\ZZ$-graded, because the R-symmetry group is trivial, so there is no possible R-charge with which to regrade the twisted theory to make it $\ZZ$-graded.  
This is of course compatible with our conventions for holomorphic Chern-Simons: holomorphic Chern-Simons (with values in an ordinary Lie algebra) on an odd dimensional Calabi-Yau only defines a $\ZZ/2$-graded theory (unless $d=3$, where this can be lifted to a $\ZZ$-grading). 
As a consequence, the twisted BV-BRST complex only has an odd symplectic pairing, not a $(-1)$-symplectic pairing.
\end{remark}

\begin{proof}
We'll prove this equivalence by first describing an equivalence of the underlying classical theories, as the composite of several maps, then showing that this equivalence is compatible with the two interaction functionals, and therefore defines a morphism of classical field theories, and finally observing that by Lemma \ref{free_int_ss_lemma} this morphism is automatically an equivalence.

\begin{enumerate}
 \item The first part of our equivalence of classical field theories will be a simple change of variables.  Let $\chi'^\vee = \chi^\vee + \Lambda F_{1,1}$, and dually let $\chi' = \chi + \Lambda F_{1,1}^\vee$.  In terms of $\chi'$, the quadratic part of the twisted action functional becomes
 \begin{align*}
  S^Q &= \int \d^5z \left(\Lambda^2 \langle \ol \dd A_{0,1} \wedge \dd A_{1,0} \rangle + \frac 12 |\chi'^\vee|^2 + \Lambda( \chi' \wedge (\ol \dd \psi)) - \Lambda^2(\dd A_{0,1} \wedge \ol \dd \psi) + \Lambda^2(B \wedge (\dd \psi))\right) \Omega + (B \wedge \ol \dd B) \\
  &\quad + \langle \dd c, A_{1,0}^\vee \rangle +  \langle \ol \dd c, A_{0,1}^\vee \rangle + \langle A_{1,0}^\vee, \psi \rangle + \langle B^\vee, \ol \dd A_{0,1} \rangle.
 \end{align*}
 The classical BV complex associated to the theory after performing this change of variables is quasi-isomorphic to the classical BV complex of the original theory according to Lemma \ref{symplectomorphism_lemma}.  However, after performing the change of variables, the classical BV complex takes the form $(\Omega^0(\CC^5; \gg)_{\chi'^\vee} \overset \id \to \Pi\Omega^0(\CC^5; \gg)_{\chi'}) \to \mc E^\bullet$, where $\mc E^\bullet$ is the part of the BV complex generated by all fields other than $\chi'$ and $\chi'^\vee$, and where the map into $\mc E^\bullet$ is given by the map $\ol \dd$ from $\chi$ to $\psi^\vee$.  Therefore the inclusion of the complex $\mc E^\bullet$ is a quasi-isomorphism by Lemma \ref{inclusion_and_projection_lemma}.  We think of this as ``integrating out'' the field $\chi'$ and its antifield.
 
 \item We'll now use a similar trick to integrate out the fields $\psi, A_{1,0}$ and their antifields.  We've argued in step 1 that the free part of the $Q$-twisted theory is equivalent to the theory with action functional 
 \[  S^Q = \int \d^5z \left(\Lambda^2 \langle \ol \dd A_{0,1} \wedge \dd A_{1,0} \rangle + \Lambda^2(B \wedge (\dd \psi))\right) \Omega + (B \wedge \ol \dd B) + \langle \dd c, A_{1,0}^\vee \rangle +  \langle \ol \dd c, A_{0,1}^\vee \rangle + \langle A_{1,0}^\vee, \psi \rangle + \langle B^\vee, \ol \dd A_{0,1} \rangle.\]
 We'll begin, as in step 1, by performing a linear change of variables, setting $A'_{1,0} = A_{1,0} - \ol{A_{0,1}}$, and performing the dual change of variables on the antifields.  This change of variables has the effect of eliminating the term $\langle \dd c, A_{1,0}^\vee \rangle$ from the quadratic part of the action. \chris{This isn't quite right I think.  We need to kill that term, however, for the below argument to work.  Can we fix it?}
 
 Observe that the classical BV complex associated to this action functional can now be written in the following form:
 \[\xymatrix{
 &&\Omega^{1,0}(\CC^5;\gg)_\psi \ar[dl] \ar[dr] \ar[r] &\Omega^{1,0}(\CC^5;\gg)_{{A'}_{1,0}} \ar[dr]\\
 \Omega^0(\CC^5;\gg)_c \ar[r] &\Omega^{0,1}(\CC^5;\gg)_{A_{0,1}}  \ar[dl] \ar[r] &\Omega^{0,2}(\CC^5;\gg)_{B} \ar[dl]\ar[r] &\Omega^{0,3}(\CC^5;\gg)_{B^\vee} \ar[r] &\Omega^{0,4}(\CC^5;\gg)_{A_{0,1}^\vee} \ar[dlll] \ar[r] &\Omega^{0,5}(\CC^5;\gg)_{c^\vee} \\
 \Omega^{4,0}(\CC^5;\gg)_{{A'}_{1,0}^\vee} \ar[r]   &\Omega^{4,0}(\CC^5;\gg)_{\psi^\vee},
 }\]
 where the first and third rows are dual under the symplectic pairing.  This is exactly in the form required to apply Lemma \ref{symplectic_composite_lemma}, so applying that result tells us that the underlying free theory of the twisted 10d super Yang Mills theory is equivalent to the middle row alone, i.e. the Dolbeault complex, which is exactly the free part of holomorphic Chern-Simons theory on $\CC^5$.
 
 \item Now, let's understand how the interaction functional behaves under this equivalence of classical field theories.  The morphism from step 1 is just given by an inclusion, so the interaction on twisted 10d Yang-Mills theory is compatible with the interaction evaluated at $\chi'=\chi'^\vee=0$.  In order to make the same observation for the fields $\psi$ and $A_{1,0}$, we'll use Lemma \ref{interaction_pullback_lemma}, which we can apply using the observation that the antifields $A_{1,0}^\vee$ and $\psi^\vee$ only appear in the action together with the corresponding fields.  Take our original action functional after the change of variables from step 1, and set the fields $\chi, \psi, A_{1,0}$ and their antifields to zero (i.e, in the notation of Lemma \ref{interaction_pullback_lemma}, restrict the interaction to the complex $\mc E^\bullet$).  The resulting interaction is 
 \[
  I^Q = \int \d^5z (B \wedge [A_{0,1} \wedge B]) + \langle B^\vee, [A_{0,1} \wedge A_{0,1}] \rangle + \langle [c,c], c^\vee \rangle + \langle [A_{0,1}, c], A_{0,1}^\vee \rangle + \langle [B,c], B^\vee \rangle,
 \]
 which is, indeed, the interaction functional for holomorphic Chern-Simons theory.  The composite of the morphisms in steps 1 and 2 therefore defines a morphism of classical field theories from holomorphic Chern-Simons theory to twisted 10d super Yang-Mills theory.
 
 \item To conclude the proof, we only need to apply Lemma \ref{free_int_ss_lemma}.  The morphism of classical field theories that we've constructed induces an equivalence of the $E_1$ pages of the free-to-interacting spectral sequences associated to holomorphically twisted 10d super Yang-Mills theory and holomorphic Chern-Simons theory.  Because the spectral sequences are convergent, there is likewise an equivalence of the $E_\infty$ pages, i.e. an equivalence of classical field theories.
\end{enumerate}
\end{proof}

The above argument can be applied identically on a general Calabi-Yau 5-fold $X$.
\brian{How do we place 10d YM on an arbitrary 5-fold? 
Are you using something about sugra?
} \chris{Not an arbitrary 5-fold, but just Calabi-Yau.  I'm not thinking of using sugra.  Instead I want to say that a Calabi-Yau 5-fold has a principal $\SU(5)$ frame bundle.  Take the fields to be sections of the associated vector bundles under the appropriate representations, then use the same action functional: the Lagrangian density is $\SU(5)$ invariant and so will define a density on the 5-fold.}


\subsection{Dimension 9}
The 9-dimensional $\mc N=1$ super Yang-Mills theory is obtained by dimensional reduction of $\mc N=1$ super Yang-Mills from $\RR^{10}$ to $\RR^9$.  It has BRST fields given by a ghost, a 9d gauge field $A$, a $\gg$-valued scalar field $\phi$, and a Dirac spinor field $\lambda$.

There is a single non-zero equivalence class of square zero supercharges in 9 dimensions. Such supercharges are minimal (i.e. have 5d image).  They are stabilized by $\SU(4)$.  Like in 10 dimensions, the R-symmetry group is trivial, and so the twisted theory by such a supercharge is only $\ZZ/2\ZZ$-graded.

Let's describe the twist of 9-dimensional $\mc N=1$ theory with respect to such a supercharge.  The relevant fields for this calculation are those component fields which survived the 10d holomorphic twist of Theorem \ref{10d_twist_thm}.  These fields, in the BRST formalism, decompose, under the action of $\SU(4)$ into the following component fields:
\begin{align*}
B^{10} &= B + A \wedge \d \ol z^5 \in (\Omega^{0,2}(\CC^4; \gg) \oplus \Omega^{0,1}(\CC^4; \gg)) \otimes \Omega^0(\RR_{x^9}) \\
A^{10}_{0,1} &= A' + \phi \d \ol z^5 \in (\Omega^{0,1}(\CC^4; \gg) \oplus \Omega^0(\CC^4; \gg)) \otimes \Omega^0(\RR_{x^9})
\end{align*}
along with the ghost $c$.  We'll describe the twisted action functional using dimensional reduction.

\brian{Where is the one-form in the $\RR$-direction? Are we identifying that with $\phi$?}

\begin{theorem} \label{9d_twist_thm}
The minimal twist of 9d super Yang-Mills theory on the product of a Calabi-Yau 4-fold $X_4$ and an oriented 1-manifold $L = \RR$ or $S^1$ is equivalent to mixed Chern-Simons theory on $X_4 \times L_\mr{dR}$.
\end{theorem}

\begin{proof}
Let's apply Lemma \ref{commuting_twist_and_reduction_lemma} -- the fact that twisting commutes with dimensional reduction.  Choose a representative element $Q$ in the equivalence class of square-zero supercharges which also squares to zero in the 10d $\mc N=(1,0)$ supersymmetry algebra.  According to Lemma \ref{commuting_twist_and_reduction_lemma}, the twist of 9d $\mc N=1$ super Yang-Mills theory by the supercharge $Q$ is equivalent to the dimensional reduction of holomorphic Chern-Simons theory on $X_4 \times C$, where $C = L \times \RR$ is either the curve $\CC$ if $L = \RR$, or $\CC^\times$ if $L = S^1$. 

As such, it's sufficient to observe that mixed Chern-Simons theorem on $X_4 \times L_{\mr{dR}}$ is a dimensional reduction of holomorphic Chern-Simons theory on $X_4 \times C$, which follows by applying Lemma \ref{CS_diml_red_lemma}.
\end{proof}

\begin{remark}
One can, alternatively, prove this theorem using exactly the same methods as Theorem \ref{10d_twist_thm}; the steps that one needs to follow are identical.
\end{remark}

\subsection{Dimension 8}
The 8-dimensional $\mc N=1$ super Yang-Mills theory is obtained by dimensional reduction of $\mc N=1$ super Yang-Mills from $\RR^{10}$ to $\RR^8$.  It has BRST fields given by a ghost, an 8d gauge field $A$, a pair of $\gg$-valued scalar fields $\phi_1$ and $\phi_2$, and a Dirac spinor field $\lambda = (\lambda_+, \lambda_-) \in S_+ \oplus S_-$.

In 8 dimensions, there are 3 classes of inequivalent non-zero supercharges by which we can twist.
\begin{enumerate}
 \item Pure spinors, for which either $\lambda_+ = 0$ or $\lambda_- = 0$.  These supercharges are holomorphic.  They are stabilized by $\SU(4)$.  Twists by pure spinors can be made $\ZZ$-graded using a maximal torus in the R-symmetry group $\GL(1)$.
 \item Impure square-zero spinors with $\lambda_+ = 0$ or $\lambda_- = 0$.  These supercharges are topological.  They are stabilized by $\Spin(7) \sub \SO(8)$.  Again, twists by such supercharges can be made $\ZZ$-graded.
 \item Square-zero spinors where both $\lambda_+$ and $\lambda_-$ are non-zero.  Such supercharges have a 5-dimensional image in $\CC^8$.  Twists by such supercharges are still only $\ZZ/2\ZZ$-graded.
\end{enumerate}

When we dimensionally reduce the minimally twisted 10- or 9-dimensional super Yang-Mills theories to 8 dimensions we obtain either the twist by a supercharge of type 1, or of type 3, depending on whether we reduce along an invariant direction or a non-invariant direction, as indicated on the ``8d'' row of Figure \ref{fig:superchargeorbits}.  The topological twist (type 2 in the list above) appears in the right-most column of the figure, and is not obtained by dimensional reduction from higher dimensions.

\subsubsection{The Holomorphic and Topological Twists}
Square-zero supercharges in the 10d $\mc N=(1,0)$ supersymmetry algebra correspond to holomorphic supercharges in the 8d $\mc N=1$ supersymmetry algebra.  As such, we can calculate the holomorphic twist of 8d $\mc N=1$ super Yang-Mills by dimensionally reducing the holomorphic twist of 10d $\mc N=(1,0)$ super Yang-Mills along a copy of $\CC$.  This theory can, unlike its 10- and 9-dimensional analogues, be made canonically $\ZZ$-graded.

On the other hand, the topological twist -- the twist by an impure Weyl spinor -- of 8d $\mc N=1$ super Yang-Mills theory does not arise by dimensionally reducing a twisted theory in 9 dimensions.  This will be our first example of an A-type twist, as we discussed in Section \ref{A_twist_section} \chris{to add}: we will understand it as a one parameter family of supercharges, degenerating at zero to the holomorphic twist -- the twist by a pure Weyl spinor.

The holomorphic supercharge is stabilized by the group $\SU(4)$, and the topological supercharge is stabilized by $\spin(7)$.  If we want to stabilize the full one-parameter family describing the degeneration, we have to decompose our fields according to the action of the intersection $\SU(4) \cap \spin(7) \iso \SU(3)$.  The BRST fields decompose as
\begin{align*}
A &\mapsto A_{0,1} + A_{1,0} + \chi^0_1 + \chi^0_2 \in (\Omega^{0,1}(\CC^3) \oplus \Omega^{1,0}(\CC^3) \oplus \Omega^0(\CC^3)^2) \otimes \Omega^0(\RR^2_{x^7,x^8}) \otimes \gg\\
\lambda_+ &\mapsto \psi^+_{0,1} + \psi^+_{1,0} + \chi^+_1 + \chi^+_2 \in (\Omega^{0,1}(\CC^3) \oplus \Omega^{1,0}(\CC^3) \oplus \Omega^0(\CC^3)^2) \otimes \Omega^0(\RR^2_{x^7,x^8}) \otimes \gg\\
\lambda_- &\mapsto \psi^-_{0,1} + \psi^-_{1,0} + \chi^-_1 + \chi^-_2 \in (\Omega^{0,1}(\CC^3) \oplus \Omega^{1,0}(\CC^3) \oplus \Omega^0(\CC^3)^2) \otimes \Omega^0(\RR^2_{x^7,x^8}) \otimes \gg
\end{align*}
along with the ghost $c$ and the two scalar fields $\phi_1$ and $\phi_2$, which remain unchanged.  In terms of these component fields we can write the action of our 1-parameter family of supercharges, which we'll denote by $Q_{\mr{hol}} + tQ'$.  We choose the two supercharge $Q_{\mr{hol}}$ and $Q'$ to each be pure individually, but to have impure sum.  The action of such a supercharge can be written in terms of a quadratic and a cubic interaction as
\begin{align*}
I^{(1)}_t &= \int \langle A_{1,0}^\vee, \psi_{1,0}^- + t\psi_{1,0}^+\rangle + \langle \phi_1^\vee, \chi_1^+ + t\chi_1^- \rangle + \langle \phi_2^\vee, \chi_2^- + t\chi_2^+ \rangle + \langle (\chi_1^-)\vee + t(\chi_1^+)^\vee, \Lambda F_{1,1} \rangle + \\
&\quad + \langle (\psi^-_{1,0})^\vee + t(\psi^+_{1,0})^\vee, F_{0,2} \rangle + \langle (\psi^-_{0,1})^\vee + t(\psi^+_{0,1})^\vee, \ol \dd_{A_{0,1}} (\phi_1 + \chi_1^0) \rangle \\
I^{(2)}_t &= \int \langle (\chi_1^+)^\vee + t(\chi_1^-)^\vee, (\chi_1^+)^\vee \rangle.
\end{align*}
\chris{This is a bit of a mess, but it's a first pass guess, and I think it's correct for $t=0$.  I want to identify $F_{0,2}$ on a CY3 with a $(1,0)$ form.}

We can also write the BV action functional -- split into its BRST and antifield parts -- as
\begin{align*}
S_{\mr{BRST}} &= \\ 
S_{\mr{anti}} &= .
\end{align*}

\chris{There might be a clever trick to avoid these action functional calculations, involving the 10d calculation we already did to argue that certain fields definitely don't contribute (the ones that died in the 10d holomorphic twist, so we're left with only $A_{0,1}, \chi^0_1, \phi_1, \psi^+_{0,1}, \psi^+_{1,0}, \psi^-_{1,0}$ and $\chi^-_1$.) -- I need to think about it.}

\begin{theorem} \label{8d_holo_twist_thm}
The holomorphic twist of 8d $\mc N=1$ super Yang-Mills theory on a Calabi-Yau 4-fold $X_4$ is equivalent to holomorphic BF theory on $X_4$. If $X_4$ splits into a product $X_3 \times S$ of a Calabi-Yau 3-fold and a flat surface, then this twist degenerates to a one-parameter family of topological twists of the form $\map(X_3 \times S, B\gg)_{\mr{Hod}}$
\end{theorem}

\begin{proof}
\chris{Strategy: compute the holomorphic twist, as in 10d, and see which fields survive.  Then consider the perturbation of the twisted action.  Show that it doesn't change the fields that survive, maybe using a spectral sequence associated to the perturbation, and compute how the BV differential deforms.}
\end{proof}

\begin{remark}
Alternatively, we can calculate the holomorphic twist by applying Lemma \ref{CS_to_BF_diml_red_lemma} to the minimal twist of 9d super Yang-Mills theory.
\end{remark}

\subsubsection{The Rank 2 Twist}
We'll next discuss the example of the twist by a rank 2 spinor, where both $\lambda_+$ and $\lambda_-$ are non-zero.  We can understand this twist by a second application of Lemma \ref{CS_diml_red_lemma} to the 9-dimensional $\mc N=1$ theory, or equivalently, by applying the lemma twice to the 10-dimensional $\mc N=(1,0)$ super Yang-Mills theory.  This twisted theory is only $\ZZ/2\ZZ$-graded, just like its higher-dimensional counterparts.

\begin{theorem} \label{8d_rank2_twist_thm}
The twist by a rank 2 supercharge of 8d $\mc N=1$ super Yang-Mills theory on the product of a Calabi-Yau 3-fold $X_3$ with a pair $L_1 \times L_2$ of oriented 1-manifolds is equivalent to mixed Chern-Simons theory on $X_3 \times (L_1 \times L_2)_{\mr{dR}}$.
\end{theorem}

\begin{proof}
The proof here is identical to the proof of Theorem \ref{9d_twist_thm}, we just start with the holomorphic twist of 10d super Yang-Mills theory on $X_3 \times C_1 \times C_2$, and apply Lemma \ref{CS_diml_red_lemma} twice.
\end{proof}

\chris{we need to comment on the fact that the rank 2 supercharge is stabilized by $\SU(3)$, but I'm not sure whether this is the entire stabilizer.}


\subsection{Dimension 7}

% \cN=1 SYM, three twists identical to dimension 8

\subsection{Dimension 6}

% \cN=(1, 1) SYM, three twists identical to dimension 8

% \cN=(1, 0) SYM with matter in a pseudo-real representation, one twist. 

\subsection{Dimension 5}

% \cN=2 SYM, three twists
% \cN=1 SYM with matter in a pseudo-real representation, one twist.

\subsection{Dimension 4}

% \cN=4 SYM
% \cN=2 SYM with matter in a pseudo-real representation
% \cN=1 SYM with matter in a complex representation

\subsubsection{$\cN=1$ SYM with matter} 

The 4-dimensional $\mc N=1$ pure super Yang-Mills theory has BRST fields given by a ghost $c$, a 4d gauge field $A$, and a Lie algebra valued Weyl spinor field $\lambda$. 
In addition, $4$-dimensional supersymmetry supports a chiral (or anti-chiral) matter multiplet, which comprises a complex scalar $\phi$ and a left (right) Weyl spinor $\psi_+$ ($\psi_-$). 

The most general theory $4$-dimensional $\mc N = 1$ theory we will consider is super Yang-Mills theory valued in a Lie algebra $\fg$ coupled to a chiral multiplet with values in a complex $\fg$-representation $R$. 
This means that the fields $\phi, \psi_+$ take values in $R$ and $\Bar{\phi}, \psi_-$ take values in $R^*$. 
We have written down the action for supersymmetric Yang-Mills theory numerous times. 
The kinetic part of the action functional for $4$-dimensional $\mc N = 1$ matter is of the form
\[
S_{\rm matter} (\phi, \Bar{\phi}, \psi_+, \psi_-) = \int \d^4 x \; \<\partial_{x_i} \phi , \partial_{x_i} \Bar{\phi}\>_R + \int \d^4 x \; \<\psi_+ , \sd \dd \psi_-\>_R .
\]
Here $\<-,-\>_R$ incorporates the obvious pairing between $R$ and $R^*$. 
Super Yang-Mills couples to the matter via the action 
\[
S_{\rm couple} = \int \d^4 x \;
\]

The action of supersymmetry on the $4$-dimensional matter theory is encoded by a linear and quadratic functional:
\begin{align*}
I^{(1)} (Q_+, Q_-) & = \int \<\phi^*, (Q_+, \psi_+)\>_R + \int \<\Bar{\phi}^*, (Q_-, \psi_-)\>_R + \int \<\psi^*_+, Q_+ \d \phi\>_R + \int \<\psi_-^*, Q_- \d \Bar{\phi}\>_R \\
I^{(2)} (Q_+ \otimes Q_-) & = \int \<\psi^*_- , \rho(\Gamma(Q_+, Q_-)) \psi_+\>_R.
\end{align*}

There is a single non-zero equivalence class of square zero supercharges in $4$-dimensional $\mc N = 1$, which we fix from here on. 
Such supercharges are minimal (i.e. have 2d image).  
They are stabilized by $\SU(2)$ \brian{really $U(2)$, i think} \chris{Agreed: a minimal square zero spinor of positive helicity is stabilized by $\SU(2)_-$, but also by a torus in $\SU(2)_+$.}.  
The R-symmetry group is $U(1)$, and the twisted theory by such a supercharge can always be made $\ZZ$-graded.

We first decompose the fields of the 4-dimensional $\mc N=1$ theory with respect to the $\SU(2)$ stabilizer.
The fields of the pure Yang-Mills part decomposes as:
\begin{align*}
c & \mapsto c \in \Omega^0(\CC^2 ; \fg) \\
A & \mapsto A_{0,1} + A_{1,0} \in \Omega^{0,1}(\CC^2 ; \fg) \oplus \Omega^{1,0}(\CC^2 ; \fg) \\
\lambda & \mapsto \chi + \psi_{1,0} + B_{0,2} \in \Omega^0(\CC^2 ; \fg) \oplus \Omega^{1,0}(\CC^2 ; \fg) \oplus \Omega^{0,2}(\CC^2 ; \fg)
\end{align*}

The matter fields decompose as
\begin{align*}
\phi & \mapsto \phi \in \Omega^0 (\CC^2 ; R) \\
\psi_+ & \mapsto \gamma_{0,1} \in \Omega^{0,1}(\CC^2 ; R) \\
\psi_- & \mapsto \beta_0 + \beta_{2,0} \in \Omega^{0,0}(\CC^2 ; R^*) \oplus \Omega^{2,0}(\CC^2 ; R^*) .
\end{align*}

%
%The matter sector of the theory is labeled by the BRST fields:
%\begin{itemize}
%\item Two scalars $\varphi_{\pm} \in C^\infty (\RR^4) \otimes R$;
%\item Weyl fermions $\psi_{\pm} \in \Pi C^\infty(\RR^4) \otimes S_{\pm} \otimes R$ 
%\end{itemize} 
%which together define the (abelian) local Lie algebra
%\[
%\mathcal E_{\rm matter}^{\cN=1} = C^\infty (\RR^4) \otimes R^{\oplus 2} \oplus  \Pi C^\infty(\RR^4) \otimes (S_{+}  \otimes R \oplus S_- \otimes R) .
%\]
%\subsection{Dimension 3}

% \cN=8 SYM
% \cN=4 SYM with matter in a pseudo-real representation
% \cN=2 SYM with matter in a complex representation

\subsection{Dimension 2}

% \cN=(8, 8) SYM
% \cN=(4, 4) SYM with matter in a pseudo-real representation
% \cN=(2, 2) SYM with matter in a complex representation
%BRIAN FIND REFERENCE OF BELOW
% \cN=(0,2) SYM with matter 
% \cN=(0,4) SYM with matter
% \cN=(0,8) SYM with matter


\appendix

\section{Functional analysis} \label{appx: top}

\def\CVS{{\rm CVS}}
\def\DVS{{\rm DVS}}

\brian{We can put all our conventions for topological vector spaces, largely borrowed from \cite{Book1}, here. }

\subsection{Notations and conventions}

\begin{definition}
Let $M$ be a manifold and $E$ a (graded) vector bundle on $M$.  
For any open set $U \subset M$, define the following notations:
\begin{itemize}
\item $\cE(U)$ denotes the graded vector space of smooth sections supported on $U$;
\item $\cE_c(U)$ denotes the graded vector space of compactly supported smooth sections on $U$;
\item $\Bar{\cE}(U)$ denotes the graded vector space of distributional smooth sections on $U$;
\item $\Bar{\cE}_c(U)$ denotes the graded vector space of compactly supported distributional smooth sections on $U$.
\end{itemize}
\end{definition}

If $E$ is a vector bundle, let $E^!$ denote the ``Verdier dual" vector bundle defined by $E^* \otimes {\rm Dens}_M$, where $E^*$ is the usual linear dual vector bundle. 
For an open set $U \subset M$ let 
\[
\cE(U)^\vee = \Bar{\cE}^!_c(U)
\]
be the continuous, or distributional, dual to $\cE(U)$. 

\brian{recall definition of CVS. all our familiar spaces above are in $\CVS$. talk about tensor products.}

\begin{prop}
Let $E,F$ be vector bundles on $M,N$ respectively.
Then
\[
\cE(M) \Hat{\otimes}_\beta \cF(N) \cong \Gamma(M \times N, E \boxtimes F)
\]
\end{prop}

\begin{definition}\label{dfn: fnl}
Let $E$ be a vector bundle on $M$ and $U \subset M$.
The {\bf completed algebra of functions} on $\cE(U)$ is
\[
\cO(\cE(U)) = \prod_{n = 0}^\infty {\rm Hom}_{\rm CVS} \left(\cE(U)^{\Hat{\otimes}_\beta n}, \CC\right)_{S_n} .
\]
On global sections, we use the notation $\cO(\cE) = \cO(\cE(M))$. 
\end{definition}

\subsection{Differentiable vector spaces}
\def\Mfld{{\rm Mfld}}

Let $\Mfld$ denote the site of smooth manifolds (of arbitrary dimension) where the morphisms are smooth maps, and a cover $f : X \to Y$ is a surjective local diffeomorphism.  

\begin{definition} 
We use the following notations and naming conventions for sheaves on $\Mfld$:
\begin{itemize}
\item a {\bf smooth vector space} is a sheaf of vector spaces on $\Mfld$.
\item Let $C^\infty$ denote the smooth vector space which assigns the space of smooth functions $C^\infty(X)$ to a smooth manifold $X$.
\item a $C^\infty$-{\bf module} is a smooth vector space $V$ equipped with a structure of a module over $C^\infty$. 
If $V$ is a $C^\infty$-module, we denote $C^\infty(X, V) := V(X)$, the valued of $V$ on the smooth manifold $X$.
\item a {\bf differentiable vector space} is a $C^\infty$-module $V$ equipped with a flat connection
\[
\nabla_{X,V} : C^\infty(X, V) \to \Omega^1(X, V)
\]
for every smooth manifold $X$, which is required to be natural under pullbacks:
\[
f^* \circ \nabla_{Y,V} = \nabla_{X,V} \circ f^*
\]
for every smooth map $f : X \to Y$. 
A map of differentiable vector spaces is a map of $C^\infty$-modules which intertwines the flat connections. 
\end{itemize}
\end{definition}

\subsection{Homological algebra}

There is a functor
\[
dif_c : \CVS \to \DVS 
\]
from the category of convenient vector spaces to the category of differentiable vector spaces. 
This functor does not preserve cokernels, so $dif_c$ does not preserve cohomology. 
In particular, with a naive definition of quasi-isomorphism internal to $\CVS$, the functor $dif_c$ does not preserve quasi-isomorphisms. 
To avoid this issue, we make the following definition. 

\begin{definition}
Suppose $V,W$ are cochain complexes in convenient vector spaces. 
A cochain map $f : V \to W$ is a {\bf quasi-isomorphism} if the map
\[
dif_c (f) : dif_c(V) \to dif_c(W)
\]
is a quasi-isomorphism of differentiable vector spaces.
\end{definition}

The cochain complexes arising in classical field theory are naturally convenient vector spaces. 
The above definition is used in what we mean by quasi-isomorphism between classical field theories, as in Definition \brian{??}.

Throughout the paper we make use of familiar manipulations with spectral sequences arising from filtrations of cochain complexes internal to differentiable vector spaces. 
A main tool is the classical Eilenberg-Moore comparison theorem, which is stated as follows.

\begin{thm}[Eilenberg-Moore Comparison]
Let $f : V \to W$ be a map of filtered cochain complexes in an AB4 abelian category.
Suppose, for each integer $n$ there exists an integer $p$ such that $F_p V^n = 0 = F_p W^n$.
If there is an integer $r \geq 0$ such that the induced map on the $r$th page of the corresponding spectral sequences 
\[
f_r^{pq} :  E_r^{pq} V \to E_r^{pq} W
\]
is an isomorphism for all $p,q$, then $f : V \to W$ is a quasi-isomorphism. 
\end{thm}

The following is the natural definition of a filtered differentiable cochain complex. 

\begin{definition}
A {\bf filtered differentiable cochain complex} is a sequence of differentiable cochain complexes
\[
\cdots \to F_{n-1} V \to F_{n} V \to F_{n+1}V \to \cdots
\]
where each map is a monomorphism in each cohomological degree. 
A filtered differentiable cochain complex is {\bf complete} if the canonical maps $V \to V / F_n V$ induce an isomorphism $V \cong \lim V / F_n V$. 
\end{definition}

The category of differentiable vector space is an AB4 category, in fact, it is actually a Grothendieck abelian category. 

\begin{thm}[{\cite[Theorem 2.2.1]{Book1}}]
The category $\DVS$ is a Grothendieck abelian category.
\end{thm}

In particular, we can freely apply the Eilenberg-Moore comparison result to the types of topological cochain complexes we come across in the BV formalism. 

\begin{definition}\label{dfn: pro}
A {\bf differentiable pro-cochain complex} is a differentiable cochain complex $V$ with a filtration
\[
\cdots \to F_{n-1} V \to F_{n} V \to F_{n+1}V \to \cdots
\]
where each map is a monomorphism in each cohomological degree and such that the canonical maps $V \to V / F_n V$ induce a quasi-isomorphism $V \to \lim V / F_n V$. 
\end{definition}

\subsubsection{The classical BV complex}

\begin{lem}
If $(E, \omega, Q, I)$ is the data of a classical theory on $M$, then the BV complex
\[
\left(\cO(\cE) , Q + \{I,-\}\right)
\]
is a differentiable pro-cochain complex.
More generally, the classical factorization algebra of observables $\Obs$ is a factorization algebra in differentiable pro-cochain complexes. 
\end{lem}

The above lemma used the natural filtration on the classical BV complex given by polynomial degree of observables. 
Since the free part of the differential $Q$ always preserves this filtration, the first term of the resulting spectral sequence simply computes the cohomology of the underlying free theory. 
The differential on the higher pages depend on the interacting part of of the action. 

\begin{lemma}[Free to Interacting Spectral Sequence] \label{free_int_ss_lemma}
Let $(E^\bullet, Q, \omega, I)$ be a classical field theory with polynomial interaction.  
There is a convergent spectral sequence whose $E_1$ page is the complex of classical observables of the underlying free theory, and whose $E_\infty$ page is equivalent to the complex of classical observables of the interacting theory.
\end{lemma}
\begin{proof}
The BV complex is of the form $\left(\cO(\cE), Q + \{I,-\}\right)$.
We have already exhibited \brian{this is a general fact that should go in the functional analysis section} a differentiable pro-cochain complex structure on the BV complex via the 
natural decreasing filtration 
\[
F^k \cO (\cE) = \Sym^{\geq k} (\cE^\vee)
\]
which is compatible with both $Q$ and $\{I,-\}$ separately.
The associated graded of this filtration is the differentiable cochain complex given by
\[
\left(\prod_{n \geq 0} \Sym^n (\cE^\vee) , Q \right) .
\]
This is simply the BV complex associated to the free theory based on $(E, Q, \omega)$ obtained by forgetting $I$. 
Thus, the $E_1$ page of the corresponding spectral sequence is the BV cohomology of the underlying free theory, as desired. 
\brian{shall we say a word about convergence? This must follow from completeness of the filtration in addition to boundedness in some direction of the cohomology of $\cE$.} \chris{Does convergence require that the interaction is polynomial, or alternatively does this bound the number of differentials appearing in the spectral sequence?}
\end{proof}

\begin{lemma} \label{invert_quis_lemma}
Any quasi-isomorphism of differentiable cochain complexes admits a quasi-inverse.
\end{lemma}
\brian{I think Pavel had something to say for this lemma.}

\section{$\infty$-Algebra}
\label{appx:infinity}

\brian{Not sure if this deserves a whole appendix, but it may be good to spell out what $L_\infty$ modules are, etc..}

\pagestyle{bib}
\printbibliography

\end{document}
