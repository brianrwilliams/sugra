\documentclass[10pt, oneside]{article}

\input ./math_headers.sty

\addbibresource{Twist.bib}

\title{Twists of supersymmetric gauge theories}
\author{Chris Elliott\and Pavel Safronov \and Brian Williams}

\date{\today}

\begin{document}

\maketitle

\section{Introduction}

\section{Preliminaries}

\subsection{Shifted symplectic manifolds}

% Define QP manifolds (Z x Z/2-graded, Q doesn't have to be Hamiltonian)
% Tangent complex of a QP manifold at a classical solution forms a cyclic L_infinity algebra
% equivalence of QP manifolds

\subsection{Models of field theories}

% field theories defined in terms of vector bundles on spacetime
% explain how to obtain a QP manifold
% explain how to obtain a factorization algebra given a classical solution

\subsection{Supersymmetry}

% spinorial representations, supertranslation groups, superPoincar\'{e} groups
% list possible massless supermultiplets in dimensions [4, 10] with <= 16 supercharges
% explain what a supersymmetric field theory is (in terms of the L_infinity action of the supertranslation algebra)

\subsection{Supersymmetric twisting}

% twisting of field theories
% twisting homomorphisms
% list all possible twists in dimensions [2, 10] with <= 16 supercharges

\section{Twists}

\subsection{Dimension 10}

% \cN=(1, 0) SYM
% Construct the L_infinity action of supersymmetry

\subsection{Dimension 9}

% \cN=1 SYM

\subsection{Dimension 8}

% \cN=1 SYM

\subsection{Dimension 7}

% \cN=1 SYM

\subsection{Dimension 6}

% \cN=(1, 1) SYM

% \cN=(1, 0) SYM with matter in a pseudo-real representation. Construct the L_infinity action of supersymmetry

\subsection{Dimension 5}

% \cN=2 SYM
% \cN=1 SYM with matter in a pseudo-real representation.

\subsection{Dimension 4}

% \cN=4 SYM
% \cN=2 SYM with matter in a pseudo-real representation
% \cN=1 SYM with matter in a complex representation

\subsection{Dimension 3}

% \cN=8 SYM
% \cN=4 SYM with matter in a pseudo-real representation
% \cN=2 SYM with matter in a complex representation

\subsection{Dimension 2}

% \cN=(8, 8) SYM
% \cN=(4, 4) SYM with matter in a pseudo-real representation
% \cN=(2, 2) SYM with matter in a complex representation

\pagestyle{bib}
\printbibliography

\end{document}
