\documentclass[10pt, oneside]{article}

\input ./combined_macros.sty

\addbibresource{Twist.bib}

\title{Twists of supersymmetric gauge theories}
\author{Chris Elliott\and Pavel Safronov \and Brian Williams}

\date{\today}

\begin{document}

\maketitle

\section{Introduction}

\section{Preliminaries}

\subsection{Shifted symplectic manifolds}

% Define QP manifolds (Z x Z/2-graded, Q doesn't have to be Hamiltonian)
% Tangent complex of a QP manifold at a classical solution forms a cyclic L_infinity algebra
% equivalence of QP manifolds

\subsection{Models of field theories}

\subsection{The definition of a field theory}
% field theories defined in terms of vector bundles on spacetime

\begin{dfn}
A {\bf free classical field theory} on a manifold $M$ is the data:
\begin{itemize}
\item a finite rank $\ZZ$-graded vector bundle $E \to M$ equipped with a differential operator of cohomological degree $+1$
\[
Q \colon \cE \to \cE [1] 
\]
such that (1) $Q^2 = 0$ and (2) $(\cE , Q)$ is an elliptic complex;
\item a map of bundles
\[
\omega : E \otimes E \to {\rm Dens}_M [-1]
\]
that is (1) fiberwise nondegenerate, (2) graded skew symmetric, and (3) satisfies $\int_M \omega\<e_0, Qe_1\> = (-1)^{|e_0|} \int_M \omega(Q e_0, e_1)$ where $e_i$ are compactly supported sections of $E$ .
\end{itemize}
\end{dfn}

\brian{recall dfn of $\oloc$ of a gr vb.}

\begin{dfn}
A {\bf classical field theory} is a free classical field theory $(E, Q, \omega)$ equipped with a functional
\[
I \in \oloc^+(E)
\]
of cohomological degree zero satisfying the classical master equation 
\[
Q I + \frac{1}{2} \{I,I\} = 0 .
\]
\end{dfn} 

\begin{remark}
There is a super variant of the above two definitions. 
A $\ZZ/2$ graded field theory is a $\ZZ \times \ZZ/2$ graded vector bundle $E$ equipped with the same additional data as above.
We require that $Q, \omega, I$ all have even parity with respect to the additional $\ZZ/2$ grading. 
\end{remark}

A local functional induces an endomorphism $\{I,-\}$ on the space of all local functionals $\oloc(E)$. 
It also induces an endomorphism on the larger space of {\em all} functionals $\cO(E)$. 
The classical master equation implies that the operator $Q + \{I,-\}$ squares to zero on $\cO(E)$, and hence every classical field theory defines a cochain complex
\[
\left(\cO(E), Q + \{I,-\}\right) .
\]
We will refer to this as the classical BV complex of the theory. 

\begin{dfn}
A {\bf morphism} of classical field theories (defined on the same manifold $M$) $F : (E, Q, \omega, I) \to (E', Q', \omega', I')$ is a linear map of vector bundles
\[
F : E \to E'
\]
that intertwines the differentials $Q, Q'$, the pairings $\omega, \omega'$, and the interactions $I,I'$. 
\end{dfn}

Note that a map of classical field theories induces a map of BV complexes
\[
F^* : \left(\cO (\cE)[-1], Q + \{I,-\} \right) \to \left(\cO (\cE')[-1], Q' + \{I',-\} \right) .
\]
This allows us to make the following definition of an equivalence between classical theories. 

\begin{dfn} \label{equivalence_def}
A map of classical field theories $F : (E, Q, \omega, I) \to (E', Q', \omega', I')$ is an {\bf equivalence} if it induces a quasi-isomorphism 
\[
F^* \colon  \left(\cO(\cE')[-1], Q' + \{I',-\} \right) \xto{\simeq} \left(\cO(\cE)[-1], Q + \{I,-\} \right) .
\]
\end{dfn}

% explain how to obtain a QP manifold

% explain how to obtain a factorization algebra given a classical solution

\subsection{Supersymmetry} \label{sec: susy}

% spinorial representations, supertranslation groups, superPoincar\'{e} groups
% list possible massless supermultiplets in dimensions [4, 10] with <= 16 supercharges

\subsubsection{Supersymmetric field theory}

% explain what a supersymmetric field theory is (in terms of the L_infinity action of the supertranslation algebra)

\begin{dfn}
Let $(E,Q,\omega, I)$ be a $\ZZ/2$-graded classical field theory on $\RR^n$. 
A {\em strict action} of a super Lie algebra $\fg$ on $(E,Q,\omega, I)$ is a map of super dg Lie algebras
\[
\rho : \fg \to \left(\oloc(E)[-1] , \{-,-\}_\omega, Q + \{I,-\}_\omega\right) .
\]
More generally, an $L_\infty$ {\em action} is an $L_\infty$ map of super dg Lie algebras
\[
\rho : \fg \rightsquigarrow \left(\oloc(E)[-1] , \{-,-\}_\omega, Q + \{I,-\}_\omega\right) .
\]
\end{dfn}

\begin{rmk}
More generally, one can speak of what it means for a 
\end{rmk}
\subsection{Supersymmetric twisting}

% twisting of field theories
% twisting homomorphisms
% list all possible twists in dimensions [2, 10] with <= 16 supercharges

\section{Twists}

\subsection{Dimension 10}

Our starting point will be (complexified) 10d super Yang-Mills theory. Fix a complex reductive gauge group $G$ with Lie algebra $\gg$.  
The ordinary fields of super Yang-Mills theory on $\RR^{10}$ consist of:
\begin{itemize}
\item A connection $A \in \Omega^1(\RR^{10} ; \fg)$ on the trivial $G$-bundle;
\item A $\gg$-valued section $\lambda \in \Omega^0(\RR^{10}) \otimes \Pi S_+ \otimes \fg$ of the Weyl spinor bundle associated to the spinor representation $S_+$. 
\footnote{If we didn't complexify we would instead consider $G_\RR$ a compact connected Lie group, and a section of the Majorana-Weyl spinor bundle, which necessitates working in Lorentzian signature.  For our purposes it's interesting enough to just consider the complexified theory and avoid signature issues.  The complexified theory twists to holomorphic Chern-Simons theory with complex gauge group.}.  
\end{itemize}
These fields are acted upon by the group of gauge transformations -- $G$-valued functions on $\RR^{10}$. 
Hence, there is a single ghost for the theory given by a $\gg$-valued section of the trivial $G$-bundle $c \in \Omega^0(\RR^{10} ; \fg)$. 

We can model the stack of fields modulo gauge transformations infinitesimally near the point $0$ by the corresponding BRST complex.  This is the local super Lie algebra
\[
L \;\;\; = \begin{array}{ccccc}
& \ul{0} & & \ul{1} & \\ 
& & & & \\
& \Omega^0(\RR^{10}; \gg) & \to & \Omega^1(\RR^{10}; \gg) \oplus \Omega^0(\RR^{10}; \Pi S_+ \otimes \gg) & 
\end{array}
\]
with the de Rham differential, placed in cohomological degrees 0 and 1, with bracket induced from the Lie bracket on $\gg$.

The action functional in 10d super Yang-Mills is given by
\[S(A,\lambda) = \int_{\RR^{10}} \langle \frac{1}{2} F_A \wedge \ast F_A - (\lambda, \sd D_A \lambda)\rangle,\]
where $\langle - \rangle_\fg$ denotes an invariant pairing on $\gg$, and $(-,-)$ denotes a scalar-valued pairing $S_+ \otimes S_- \to \CC$ (there will be a unique such pairing, up to rescaling, characterized by the condition that $(\rho(v)\lambda_1,\rho(v)\lambda_2) = (\lambda_1,\lambda_2)$ for each $v \in \CC^{10}$, where $\rho$ denotes Clifford multiplication). \brian{I think we should set up this notation in Section \ref{sec: susy} above, and recount it briefly here.}

We can re-encode this data in terms of the classical BV complex (see also \cite[Section 3.1]{ElliottYoo1}).  
This is the local $L_\infty$-algebra $\fL$ on $\RR^{10}$ whose underlying cochain complex takes the form
\[
\xymatrix{
& & \ul{0} & \ul{1} & \ul{2} & \ul{3} \\
\fL & = & \Omega^0(\RR^{10}; \gg) \ar[r]^{\d} &\Omega^1(\RR^{10}; \gg) \ar[r]^{\d \ast \d} &\Omega^9(\RR^{10}; \gg) \ar[r]^{\d} &\Omega^{10}(\RR^{10}; \gg) \\
& & &\Omega^0(\RR^{10}; \Pi S_+ \otimes \gg) \ar[r]^{\ast \sd \d} &\Omega^{10}(\RR^{10}; \Pi S'_- \otimes \gg), &
}\]
with degree $(-3)$ invariant pairing $\<-,-\>$ induced by the invariant pairing on $\gg$ and the pairing $(-,-)$ between $S_+$ and $S_-$, and with degree 2 and 3 brackets given by the action of $\Omega^0(\RR^{10}; \gg)$ on everything along with
\begin{align*}
\ell_2^{\mr{Bos}} \colon \Omega^1(\RR^{10};\gg) \otimes \Omega^1(\RR^{10};\gg) &\to \Omega^{9}(\RR^{10};\gg) \\
(A \otimes B) &\mapsto [A \wedge \ast \mr d B] + [\ast \mr d  A \wedge B] + \mathrm{d} \ast[A \wedge B] \\
\ell_2^{\mr{Fer}} \colon \Omega^1(\RR^{10};\gg) \otimes \Omega^0(\RR^{10}; S_{+} \otimes \gg) &\to \Omega^{10}(\RR^{10}; S_{-} \otimes \gg) \\
(A \otimes \lambda) &\mapsto \ast \sd A \lambda
\end{align*}
in degree 2, and the map
\begin{align*}
\ell_3 \colon \Omega^1(\RR^{10};\gg) \otimes \Omega^1(\RR^{10};\gg) \otimes \Omega^1(\RR^{10};\gg) &\to \Omega^{9}(\RR^{10};\gg) \\
(A \otimes B \otimes C) &\mapsto [A \wedge \ast[B \wedge C]] + [B \wedge \ast[C \wedge A]] + [C \wedge \ast[A \wedge B]]
\end{align*}
in degree 3.

We obtain the BV action by the formula
\[
S_{BV} (\alpha) = \frac{1}{2} \<\alpha , Q_{BV} \alpha\> + \sum_{n \geq 2} \frac{1}{n!} \<\alpha, \ell_n(\alpha,\ldots, \alpha)\> 
\]
where $\alpha$ is a general BV field and $Q_{BV}$ is the linear BV differential. 
The statement that $S_{BV}$ is gauge invariant is encoded by the fact that it satisfies that classical master equation $\{S_{BV}, S_{BV}\} = 0$, which is equivalent to the statement that $S_{BV}$ determines a Mauer-Cartan element in the dg Lie algebra $\cloc^\bu(\fL)[-1]$. 

\subsubsection{The Supersymmetry Action}
In order to construct an action of the $\mc N=(1,0)$ supersymmetry algebra $\mf A$ on the 10d supersymmetric gauge theory $\fL$, we proceed in two steps.
\begin{enumerate}
 \item There is an ordinary Lie action of the super Lie algebra $\mf A$ on the theory $\fL$, extending the standard action of the Poincar\'e algebra by isometries of $\RR^{10}$ \chris{define this above}, which is only well-defined on-shell.  In other words, there is a linear map $\delta^{(1)} \colon \mf A \to \cloc^\bu(\fL)[-1]$ which is a Lie action modulo the ideal generated by the equations of motion \chris{induces a Lie homomorphism on cohomology?}.
 \item We can promote this on-shell action to an $L_\infty$ map $\delta^{(1)} \colon \mf A \to \cloc^\bu(\fL)[-1]$ by including a quadratic term $\delta^{(2)}$ which ``corrects'' for the failure of the map $\delta^{(1)}$ to be a Lie map on-the-nose.
\end{enumerate}

Since this is our first example, we will work through it in some detail, in order to motivate the formul\ae{} that will recur in later examples.  The fermionic piece of the supersymmetry algebra is defined by saying that $Q \in S_+$ acts infinitesimally by
\[
\begin{pmatrix}
A \\ \lambda
\end{pmatrix}
\mapsto
\begin{pmatrix} A + \delta_Q A \\
\lambda + \delta_Q \lambda
\end{pmatrix}
\]
where 
\begin{align*}
\delta_Q A &= \Gamma(Q,\lambda) \\
\delta_Q \lambda &= \sd F_A Q .
\end{align*}
Here, the notation $\sd F_A$ stands for the iterated Clifford multiplication $\sd F_A = F_{ij} \gamma^i \gamma^j$. 

\begin{lemma} \label{10d_onshell_lemma}
Suppose $Q_1, Q_2 \in S_+$ and $(A, \lambda)$ are fields.
The following relations hold:
\begin{itemize}
\item[(1)] \label{10dsusyA} $ [\delta_{Q_1}, \delta_{Q_2}] A = \delta_{[Q_1, Q_2]} A$.
\item[(2)] \label{10dsusyL} $ [\delta_{Q_1}, \delta_{Q_2}] \lambda = \delta_{[Q_1,Q_2]} \lambda - \rho(\Gamma(Q_1,Q_2)) \sd \dd \lambda - \frac 12(Q_2, \sd \dd \lambda)Q_1 - \frac 12(Q_2, \sd \dd \lambda)Q_2$ .
\end{itemize}
Here, the commutator on the left hand side of the equations takes place in the algebra of endomorphisms of the space of fields.
\end{lemma}

\begin{proof}
Both are direct calculations using standard Clifford relations which cite below.
So, we calculate
\begin{align*}
[\delta_{Q_1}, \delta_{Q_2}] A &= (\Gamma(Q_2,\sd F_A Q_1) + \Gamma(Q_1,\sd F_A Q_2)) \\
&=  F_{ij}(Q_2 \gamma^k \gamma^j \gamma^i Q_1 + Q_1 \gamma^k \gamma^j \gamma^i Q_2) \\
&=  F_{ij}(Q_2 \gamma^k \gamma^j \gamma^i Q_1 + Q_2 \gamma^i \gamma^j \gamma^k Q_1)\\
&= F_{ij}(\delta^{jk}(Q_2\gamma^i Q_i) - Q_2 \gamma^i \gamma^j \gamma^k + Q_2 \gamma^i \gamma^j \gamma^k Q_1)\\
&=  F_{ij}\delta^{jk}(Q_2 \gamma^i Q_1) \\
&= \delta_{[Q_1, Q_2]} A,
\end{align*}
where on the third line we used the fact that the pairing $\Gamma(-,-)$ is symmetric -- i.e. that $\lambda_1 \gamma^i \lambda_2 = \lambda_2 \gamma^i \lambda_1$ -- and on the fourth line we used the Clifford relation $\gamma^j\gamma^j+\gamma^j\gamma^j = \delta^{jk}$.  Note that, on the gauge fields, the action is a Lie action on the nose, not only on-shell.  Similarly we can calculate, following the calculation in Guillen \cite{Guillen}:
\begin{align*}
[\delta_{Q_1}, \delta_{Q_2}] \lambda &= (\sd F_{\Gamma(Q_2, \lambda)} Q_1 + \sd F_{\Gamma(Q_1,\lambda)} Q_2) \\
&= \frac 12((Q_2 (\gamma_j \dd_i - \gamma_i \dd_j) \lambda) (\gamma^i \gamma^j Q_1) + (1 \leftrightarrow 2)) \\
&= \frac 12((Q_2 \gamma_j \dd_i \lambda) \gamma^i \gamma^j Q_1 + (Q_1 \gamma_j \dd_i \lambda) \gamma^i \gamma^j Q_2) - \frac 12((Q_2 \gamma_i \dd_j \lambda) \gamma^i \gamma^j Q_1 + (Q_1 \gamma_i \dd_j \lambda) \gamma^i \gamma^j Q_2) \\
&= \frac 12((Q_2 \gamma_j \dd_i \lambda) \gamma^i \gamma^j Q_1 + (Q_1 \gamma_j \dd_i \lambda) \gamma^i \gamma^j Q_2) + \frac 12((Q_2 \gamma_i \dd_j \lambda) \gamma^j \gamma^j Q_1 + (Q_1 \gamma_i \dd_j \lambda) \gamma^j \gamma^i Q_2) + \\
&\quad - \frac 12((Q_2 \gamma_i \dd_j \lambda) \delta_{ij} Q_1 + \frac 12(Q_1 \gamma_i \dd_j \lambda) \delta^{ij} Q_2) \\
&= ((Q_1 \gamma_j Q_2) (\gamma^i \gamma^j \dd_i \lambda) - \frac 12(Q_2 \gamma_i \dd_i \lambda)Q_1 - \frac 12(Q_1 \gamma_i \dd_i \lambda)Q_2
\end{align*}
using the fact that 
\[(\psi_1 \gamma_j \psi_2)(\gamma^j \psi_3) + (\psi_2 \gamma_j \psi_3)(\gamma^j \psi_1) + (\psi_3 \gamma_j \psi_1)(\gamma^j \psi_2) = 0,\]
as in \cite[Theorem 11]{BaezHuerta}.  Making one more simplification using the Clifford relations, we have
\begin{align*}
[\delta_{Q_1}, \delta_{Q_2}] \lambda &= ((Q_1 \gamma_j Q_2) (\delta^{ij} \dd_i \lambda) - ((Q_1 \gamma_j Q_2) (\gamma^j \gamma^i \dd_i \lambda) - \frac 12(Q_2 \gamma_i \dd_i \lambda)Q_1 - \frac 12(Q_1 \gamma_i \dd_i \lambda)Q_2 \\
&= \delta_{[Q_1,Q_2]} \lambda - \rho(\Gamma(Q_1,Q_2)) \sd \dd \lambda - \frac 12(Q_2, \sd \dd \lambda)Q_1 - \frac 12(Q_2, \sd \dd \lambda)Q_2.
\end{align*}
\end{proof}

In particular the supersymmetry action is a Lie algebra homomorphism only modulo the ideal generated by the equation of motion $\sd \dd \lambda = 0$.
In other words, this supersymmetry action only defined an action of the Lie algebra of supertranslations ``on-shell".  This calculation suggests introducing a second order correction to the supersymmetry action on the BV theory, which has the chance of closing off-shell.  Define a second order action depending on the antifield $\lambda^*$ to the gluino $\lambda$ by
\begin{align*}
\delta^{(2)} \colon S_+ \otimes S_+ \otimes \Gamma(\RR^{10}; \Pi S_-[-1]) &\to \Gamma(\RR^{10}; \Pi S_+) \\
Q_1 \otimes Q_2 \otimes \lambda^* &\mapsto - \left(\rho(\Gamma(Q_1,Q_2)) \lambda^* + \frac 12 \left((Q_2, \lambda^*)Q_1 + (Q_1, \lambda^*)Q_2\right)\right).
\end{align*}

We'll rewrite this as a local functional in the theory $\fL$ coupled to the $\mc N=(1,0)$ super Lie algebra $\mf A$.

\begin{definition}
The off-shell supersymmetry action on complexified Yang-Mills theory on $\RR^{10}$ is defined to be the cochain
\[I^{(1)} + I^{(2)} \in \clie^\bu(\mc A) \otimes \cloc^\bu(\fL),\]
where 
\begin{align*}
I ^{(1)} (Q ; A, \lambda, A^*, \lambda^*) & = \<A^* , \Gamma(Q, \lambda)\> + \<\lambda^*, \sd F_A Q\> \\
I^{(2)} (Q_1,Q_2 ; \lambda^*) & = \pm \left\<\lambda^* \;,\; \rho(\Gamma(Q_1,Q_2)) \lambda^* + \frac 12 \left((Q_2, \lambda^*)Q_1 + (Q_1, \lambda^*)Q_2\right)\right\> .
\end{align*}
\end{definition}

\begin{prop}
Let $S_{\mr{BV}}$ be the BV action functional in the theory $\fL$, and write $I_{\mr{Poin}}$ for the interaction generating the Lie action of the Poincar\'e algebra \chris{we'll have to see whether we want Poincar\'e or just translation here}. The functional
\[\fS = S_{\mr{BV}} + I_{\mr{Poin}} + I^{(1)} + I^{(2)} \in \clie^\bu(\mc A) \otimes \cloc^\bu(\fL) [-1]\]
satisfies the Maurer-Cartan equation
\begin{equation} 
\label{10d_MC}
(\d_{\rm Lie} \fS + \frac{1}{2} \left\{\fS , \fS \right\} = 0 .
\end{equation}
\end{prop}

\begin{proof}
We can filter the complex $\clie^\bu(\mc A) \otimes \cloc^\bu(\fL)$ by the number of anti-field components in $\fL$.  Using this filtration, the left-hand side of \ref{10d_MC} splits up as
\begin{align}
\d_{\rm Lie}  \left( \fS \right) + \frac{1}{2} \left\{ \fS , \fS \right\} &= \d_{\mr{Lie}} I_{\mr{Poin}} + \frac 12 \{S_{\mr{BV}} + I_{\mr{Poin}}, S_{\mr{BV}} + I_{\mr{Poin}}\} \label{10d_MC_BV}\\
&\quad + \{S_{\mr{BV}}, I^{(1)}\} \label{10d_MC_1}\\
&\quad + \d_{\mr{Lie}} I^{(1)} + \{S_{\mr{BV}}, I^{(2)}\} + \{I_{\mr{Poin}}, I^{(1)}\} + \frac 12 \{I^{(1)}, I^{(1)}\} \label{10d_MC_2}\\
&\quad + \d_{\mr{Lie}} I^{(2)} + \{I_{\mr{Poin}}, I^{(2)}\} + \{I^{(1)}, I^{(2)}\} \label{10d_MC_3}\\
&\quad + \frac 12 \{I^{(2)}, I^{(2)}\}. \label{10d_MC_4}
\end{align}
Taking these terms one at a time, we first note that the vanishing of \ref{10d_MC_BV} is just the fact that the BV action of the 10d super Yang-Mills theory satisfies the classical master equation, along with the fact that this theory is Poincar\'e invariant \chris{ref some earlier discussion}.  The remaining terms are those involving the action of supersymmetries.

First, the term \ref{10d_MC_1} vanishes by \chris{We've discussed this.  First the $\{I_{\mr{Poin}}, I^{(i)}\}$ terms vanish by Poincar\'e invariance of the interaction terms.  Then \ref{10d_MC_1} is zero by Baez--Huerta.  Term \ref{10d_MC_2} is Lemma \ref{10d_onshell_lemma}. Term \ref{10d_MC_3} vanishes because, first $\d_{\mr{Lie}}I^{(2)} = 0$ since ...not sure right now..., then the remaining term vanishes by the trick where we symmetrize in three spinors.  Finally \ref{10d_MC_4} vanishes because $I^{(2)}$ only involves antifields.}
\end{proof}


\subsubsection{The Holomorphic Twist}

\newcommand{\hCS}{\mathrm{hCS}}
\chris{this definition should go elsewhere.  We also need to include a description of the holomorphic supercharge, and something about the trivial $\SU(5)$ twisting homomorphism.}

\begin{definition}
Let $X$ be a Calabi-Yau variety of odd complex dimension $d$, and let $G$ be a complex reductive group.  \emph{Holomorphic Chern-Simons theory} is the $\ZZ/2\ZZ$-graded classical field theory $(E_\hCS, Q_\hCS, \omega_\hCS, I_\hCS)$, where $E_\hCS$ is the graded vector bundle $\Omega^{0,\bullet}(X;\gg)$, $Q_{\hCS}$ is the Dolbeault differential, $\omega_\hCS$ is the density-valued pairing associated to the Calabi-Yau structure on $X$ and a choice of invariant pairing on $\gg$, and $I_\hCS$ is the cubic interactional functional
\[I_{\hCS}(\alpha) = \frac 16 [\alpha \wedge \alpha \wedge \alpha].\]
\end{definition}

\begin{theorem}
The holomorphic twist of 10d super Yang-Mills theory on a Calabi-Yau 5-fold $X$ is equivalent to holomorphic Chern-Simons theory on $X$.
\end{theorem}

\begin{proof}
We will first construct a morphism $\iota$ from holomorphic Chern-Simons theory to the twisted theory $\fL^Q$. \chris{We split the fields of 10d super Yang-Mills into $\SU(5)$ irreducible representations, so the ghost $c$, then fields $A_{0,1}, A_{1,0}, B, \psi, \chi$ and their antifields, and include $c, A_{0,1}, B, B^\vee, A_{0,1}^\vee, c^\vee$.  Easy to check that it defines a morphism to the twisted theory.}

Now, to verify that this morphism is an equivalence as in Definition \ref{equivalence_def}, we must check that the induced morphism 
\[\iota^* \colon  \left(\cO(\cE_{\mr{SYM}})[-1], Q_{\mr{SYM}} + \{I_{\mr{BV}} + I^{(1)}_Q + I^{(2)}_{Q\otimes Q},-\} \right) \xto{\simeq} \left(\cO(\cE_\hCS)[-1], Q_\hCS + \{I_\hCS,-\} \right)\]
is a quasi-isomorphism. \chris{maybe try filtering the left-hand side by the number of $\chi$ and $\chi^\vee$ fields occuring.  The 0th filtered piece should coincide with the right-hand side exactly, and maybe we'd like to see that everything else is killed off by some differential in the spectral sequence?}

\end{proof}



\subsection{Dimension 9}

% \cN=1 SYM

\subsection{Dimension 8}

% \cN=1 SYM

\subsection{Dimension 7}

% \cN=1 SYM

\subsection{Dimension 6}

% \cN=(1, 1) SYM

% \cN=(1, 0) SYM with matter in a pseudo-real representation. Construct the L_infinity action of supersymmetry

\subsection{Dimension 5}

% \cN=2 SYM
% \cN=1 SYM with matter in a pseudo-real representation.

\subsection{Dimension 4}

% \cN=4 SYM
% \cN=2 SYM with matter in a pseudo-real representation
% \cN=1 SYM with matter in a complex representation

\subsubsection{$\cN=1$ SYM with matter} 

Let $R$ be a complex representation of $\mf g$.

The matter sector of the theory is labeled by the BRST fields:
\begin{itemize}
\item Two scalars $\varphi_{\pm} \in C^\infty (\RR^4) \otimes R$;
\item Weyl fermions $\psi_{\pm} \in \Pi C^\infty(\RR^4) \otimes S_{\pm} \otimes R$ 
\end{itemize} 
which together define the (abelian) local Lie algebra
\[
\mathcal E_{\rm matter}^{\cN=1} = C^\infty (\RR^4) \otimes R^{\oplus 2} \oplus  \Pi C^\infty(\RR^4) \otimes (S_{+}  \otimes R \oplus S_- \otimes R) .
\]
\subsection{Dimension 3}

% \cN=8 SYM
% \cN=4 SYM with matter in a pseudo-real representation
% \cN=2 SYM with matter in a complex representation

\subsection{Dimension 2}

% \cN=(8, 8) SYM
% \cN=(4, 4) SYM with matter in a pseudo-real representation
% \cN=(2, 2) SYM with matter in a complex representation
%BRIAN FIND REFERENCE OF BELOW
% \cN=(0,2) SYM with matter 
% \cN=(0,4) SYM with matter
% \cN=(0,8) SYM with matter

\pagestyle{bib}
\printbibliography

\end{document}
