\documentclass[10pt, oneside]{article}

\input ./math_headers.sty

\def\cA{\mathcal A}\def\cB{\mathcal B}\def\cC{\mathcal C}\def\cD{\mathcal D}
\def\cE{\mathcal E}\def\cF{\mathcal F}\def\cG{\mathcal G}\def\cH{\mathcal H}
\def\cI{\mathcal I}\def\cJ{\mathcal J}\def\cK{\mathcal K}\def\cL{\mathcal L}
\def\cM{\mathcal M}\def\cN{\mathcal N}\def\cO{\mathcal O}\def\cP{\mathcal P}
\def\cQ{\mathcal Q}\def\cR{\mathcal R}\def\cS{\mathcal S}\def\cT{\mathcal T}
\def\cU{\mathcal U}\def\cV{\mathcal V}\def\cW{\mathcal W}\def\cX{\mathcal X}
\def\cY{\mathcal Y}\def\cZ{\mathcal Z}

\def\AA{\mathbb A}\def\BB{\mathbb B}\def\CC{\mathbb C}\def\DD{\mathbb D}
\def\EE{\mathbb E}\def\FF{\mathbb F}\def\GG{\mathbb G}\def\HH{\mathbb H}
\def\II{\mathbb I}\def\JJ{\mathbb J}\def\KK{\mathbb K}\def\LL{\mathbb L}
\def\MM{\mathbb M}\def\NN{\mathbb N}\def\OO{\mathbb O}\def\PP{\mathbb P}
\def\QQ{\mathbb Q}\def\RR{\mathbb R}\def\SS{\mathbb S}\def\TT{\mathbb T}
\def\UU{\mathbb U}\def\VV{\mathbb V}\def\WW{\mathbb W}\def\XX{\mathbb X}
\def\YY{\mathbb Y}\def\ZZ{\mathbb Z}

\def\fA{\mathfrak A}\def\fB{\mathfrak B}\def\fC{\mathfrak C}\def\fD{\mathfrak D}
\def\fE{\mathfrak E} \def\fF{\mathfrak F}\def\fG{\mathfrak G}\def\fH{\mathfrak H}
\def\fI{\mathfrak I}\def\fJ{\mathfrak J}\def\fK{\mathfrak K}\def\fL{\mathfrak L}
\def\fM{\mathfrak M}\def\fN{\mathfrak N}\def\fO{\mathfrak O}\def\fP{\mathfrak P}
\def\fQ{\mathfrak Q}\def\fR{\mathfrak R}\def\fS{\mathfrak S}\def\fT{\mathfrak T}
\def\fU{\mathfrak U}\def\fV{\mathfrak V}\def\fW{\mathfrak W}\def\fX{\mathfrak X}
\def\fY{\mathfrak Y}\def\fZ{\mathfrak Z}

\def\fa{\mathfrak a}\def\fb{\mathfrak b}\def\fc{\mathfrak c}\def\fd{\mathfrak d}
\def\fe{\mathfrak e}\def\ff{\mathfrak f}\def\fg{\mathfrak g}\def\fh{\mathfrak h} \def\fj{\mathfrak j}\def\fk{\mathfrak k}\def\fl{\mathfrak l} \def\fm{\mathfrak m}\def\fn{\mathfrak n}\def\fo{\mathfrak o}\def\fp{\mathfrak p} \def\fq{\mathfrak q}\def\fr{\mathfrak r}\def\fs{\mathfrak s}\def\ft{\mathfrak t}
\def\fu{\mathfrak u}\def\fv{\mathfrak v}\def\fw{\mathfrak w}\def\fx{\mathfrak x}
\def\fy{\mathfrak y}\def\fz{\mathfrak z}

\def\<{\langle}
\def\>{\rangle}
\def\cloc{{\rm C}_{\rm loc}}
\def\bu{\bullet}
\def\clie{{\rm C}_{\rm Lie}}

\addbibresource{Twist.bib}

\title{Twists of supersymmetric gauge theories}
\author{Chris Elliott\and Pavel Safronov \and Brian Williams}

\date{\today}

\begin{document}

\maketitle

\section{Introduction}

\section{Preliminaries}

\subsection{Shifted symplectic manifolds}

% Define QP manifolds (Z x Z/2-graded, Q doesn't have to be Hamiltonian)
% Tangent complex of a QP manifold at a classical solution forms a cyclic L_infinity algebra
% equivalence of QP manifolds

\subsection{Models of field theories}

% field theories defined in terms of vector bundles on spacetime
% explain how to obtain a QP manifold
% explain how to obtain a factorization algebra given a classical solution

\subsection{Supersymmetry} \label{sec: susy}

% spinorial representations, supertranslation groups, superPoincar\'{e} groups
% list possible massless supermultiplets in dimensions [4, 10] with <= 16 supercharges
% explain what a supersymmetric field theory is (in terms of the L_infinity action of the supertranslation algebra)

\brian{How we we feel about using $L$ for the local Lie algebra of BRST fields and $\fL$ for the BV fields?}

\subsection{Supersymmetric twisting}

% twisting of field theories
% twisting homomorphisms
% list all possible twists in dimensions [2, 10] with <= 16 supercharges

\section{Twists}

\subsection{Dimension 10}

Our starting point will be (complexified) 10d super Yang-Mills theory.  
Fix a complex reductive gauge group $G$ with Lie algebra $\gg$.  
The ordinary fields of super Yang-Mills theory on $\RR^{10}$ consist of:
\begin{itemize}
\item a single boson: a connection $A \in \Omega^1(\RR^{10} ; \fg)$ on the trivial $G$-bundle;
\item a single fermion: a $\gg$-valued section $\lambda \in \Omega^0(\RR^{10}) \otimes \Pi S_+ \otimes \fg$ of the Weyl spinor bundle associated to the spinor representation $S_+$. 
\footnote{If we didn't complexify we would instead consider $G_\RR$ a compact connected Lie group, and a section of the Majorana-Weyl spinor bundle, which necessitates working in Lorentzian signature.  I think for our purposes it's interesting enough to just consider the complexified theory and avoid signature issues.  The complexified theory twists to holomorphic Chern-Simons theory with complex gauge group.}.  
\end{itemize}
These fields are acted upon by the group of gauge transformations -- $G$-valued functions on $\RR^{10}$. 
Hence, there is a single ghost for the theory given by a $\gg$-valued section of the trivial $G$-bundle $c \in \Omega^0(\RR^{10} ; \fg)$. 

We can model the stack of fields modulo gauge transformations infinitesimally near the point $0$ by the corresponding BRST complex.  This is the local super Lie algebra
\[
L \;\;\; = \begin{array}{ccccc}
& \ul{0} & & \ul{1} & \\ 
& & & & \\
& \Omega^0(\RR^{10}; \gg) & \to & \Omega^1(\RR^{10}; \gg) \oplus \Omega^0(\RR^{10}; \Pi S_+ \otimes \gg) & 
\end{array}
\]
with the de Rham differential, placed in cohomological degrees 0 and 1, with bracket induced from the Lie bracket on $\gg$.

The action functional in 10d super Yang-Mills is given by
\[S(A,\lambda) = \int_{\RR^{10}} \langle \frac{1}{2} F_A \wedge \ast F_A - (\lambda, \sd D_A \lambda)\rangle,\]
where $\langle - \rangle_\fg$ denotes an invariant pairing on $\gg$, and $(-,-)$ denotes a scalar-valued pairing $S_+ \otimes S_- \to \CC$ (there will be a unique such pairing, up to rescaling, characterized by the condition that $(\rho(v)\lambda_1,\rho(v)\lambda_2) = (\lambda_1,\lambda_2)$ for each $v \in \CC^{10}$, where $\rho$ denotes Clifford multiplication). \brian{I think we should set up this notation in Section \ref{sec: susy} above, and recount it briefly here.}

We can re-encode this data in terms of the classical BV complex (see also \cite[Section 3.1]{ElliottYoo1}).  
This is the local $L_\infty$-algebra $\fL$ on $\RR^{10}$ whose underlying cochain complex takes the form
\[
\xymatrix{
& & \ul{0} & \ul{1} & \ul{2} & \ul{3} \\
\fL & = & \Omega^0(\RR^{10}; \gg) \ar[r]^{\d} &\Omega^1(\RR^{10}; \gg) \ar[r]^{\d \ast \d} &\Omega^9(\RR^{10}; \gg) \ar[r]^{\d} &\Omega^{10}(\RR^{10}; \gg) \\
& & &\Omega^0(\RR^{10}; \Pi S_+ \otimes \gg) \ar[r]^{\ast \sd \d} &\Omega^{10}(\RR^{10}; \Pi S'_- \otimes \gg), &
}\]
\brian{I've changed the spinor component in degree zero to $S_+$, check if that's OK.}
with degree $(-3)$ invariant pairing $\<-,-\>$ induced by the invariant pairing on $\gg$ and the pairing $(-,-)$ between $S_+$ and $S_-$, and with degree 2 and 3 brackets given by the action of $\Omega^0(\RR^{10}; \gg)$ on everything along with
\begin{align*}
\ell_2^{\mr{Bos}} \colon \Omega^1(\RR^{10};\gg) \otimes \Omega^1(\RR^{10};\gg) &\to \Omega^{9}(\RR^{10};\gg) \\
(A \otimes B) &\mapsto [A \wedge \ast \mr d B] + [\ast \mr d  A \wedge B] + \mathrm{d} \ast[A \wedge B] \\
\ell_2^{\mr{Fer}} \colon \Omega^1(\RR^{10};\gg) \otimes \Omega^0(\RR^{10}; S_{+} \otimes \gg) &\to \Omega^{10}(\RR^{10}; S_{-} \otimes \gg) \\
(A \otimes \lambda) &\mapsto \ast \sd A \lambda
\end{align*}
in degree 2, and the map
\begin{align*}
\ell_3 \colon \Omega^1(\RR^{10};\gg) \otimes \Omega^1(\RR^{10};\gg) \otimes \Omega^1(\RR^{10};\gg) &\to \Omega^{9}(\RR^{10};\gg) \\
(A \otimes B \otimes C) &\mapsto [A \wedge \ast[B \wedge C]] + [B \wedge \ast[C \wedge A]] + [C \wedge \ast[A \wedge B]]
\end{align*}
in degree 3.

We obtain the BV action by the formula
\[
S_{BV} (\alpha) = \frac{1}{2} \<\alpha , Q_{BV} \alpha\> + \sum_{n \geq 2} \frac{1}{n!} \<\alpha, \ell_n(\alpha,\ldots, \alpha)\> 
\]
where $\alpha$ is a general BV field and $Q_{BV}$ is the linear BV differential. 
The statement that $S_{BV}$ is gauge invariant is encoded by the fact that it satisfies that classical master equation $\{S_{BV}, S_{BV}\} = 0$, which is equivalent to the statement that $S_{BV}$ determines a Mauer-Cartan element in the dg Lie algebra $\cloc^\bu(\fL)[-1]$. 

\subsubsection{The naive supersymmetry action}

The naive supersymmetry action is defined by saying that bosonic piece of the supersymmetry algebra acts by isometries on $\CC^{10}$, and on the fields via pullback. 
The fermionic piece of the supersymmetry algebra is defined by saying that $Q \in S_+$ acts infinitesimally by
\[
\begin{pmatrix}
A \\ \lambda
\end{pmatrix}
\mapsto
\begin{pmatrix} A + \delta_Q A \\
\lambda + \delta_Q \lambda
\end{pmatrix}
\]
where 
\begin{align*}
\delta_Q A &= \Gamma(Q,\lambda) \\
\delta_Q \lambda &= \sd F_A Q .
\end{align*}
Here, the notation $\sd F_A$ stands for the iterated Clifford multiplication $\sd F_A = F_{ij} \gamma^i \gamma^j$.  
%To check that this defines an on-shell supersymmetry action we need to check it's compatible with the brackets in the supersymmetry algebra, up to terms in the ideal generated by the equations of motion.  

\begin{lemma}
Suppose $Q_1, Q_2 \in S_+$ and $(A, \lambda)$ are fields.
The following relations hold:
\begin{itemize}
\item[(1)] \label{10dsusyA} $ [\delta_{Q_1}, \delta_{Q_2}] A = \delta_{[Q_1, Q_2]} A$.
\item[(2)] \label{10dsusyL} $ [\delta_{Q_1}, \delta_{Q_2}] \lambda = \delta_{[Q_1,Q_2]} \lambda - \rho(\Gamma(Q_1,Q_2)) \sd \dd \lambda - \frac 12(Q_2, \sd \dd \lambda)Q_1 - \frac 12(Q_2, \sd \dd \lambda)Q_2$ .
\end{itemize}
Here, the commutator on the left hand side of the equations takes place in the algebra of endomorphisms of the space of fields.
\end{lemma}
\begin{proof}
Both are direct calculations using standard Clifford relations which cite below.
So, we calculate
\begin{align*}
[\delta_{Q_1}, \delta_{Q_2}] A &= (\Gamma(Q_2,\sd F_A Q_1) + \Gamma(Q_1,\sd F_A Q_2)) \\
&=  F_{ij}(Q_2 \gamma^k \gamma^j \gamma^i Q_1 + Q_1 \gamma^k \gamma^j \gamma^i Q_2) \\
&=  F_{ij}(Q_2 \gamma^k \gamma^j \gamma^i Q_1 + Q_2 \gamma^i \gamma^j \gamma^k Q_1)\\
&= F_{ij}(\delta^{jk}(Q_2\gamma^i Q_i) - Q_2 \gamma^i \gamma^j \gamma^k + Q_2 \gamma^i \gamma^j \gamma^k Q_1)\\
&=  F_{ij}\delta^{jk}(Q_2 \gamma^i Q_1) \\
&= \delta_{[Q_1, Q_2]} A,
\end{align*}
where on the third line we used the fact that the pairing $\Gamma(-,-)$ is symmetric -- i.e. that $\lambda_1 \gamma^i \lambda_2 = \lambda_2 \gamma^i \lambda_1$ -- and on the fourth line we used the Clifford relation $\gamma^j\gamma^j+\gamma^j\gamma^j = \delta^{jk}$.  Note that, on the gauge fields, the action is a Lie action on the nose, not only on-shell.  Similarly we can calculate, following the calculation in Guillen \cite{Guillen}:
\begin{align*}
[\delta_{Q_1}, \delta_{Q_2}] \lambda &= (\sd F_{\Gamma(Q_2, \lambda)} Q_1 + \sd F_{\Gamma(Q_1,\lambda)} Q_2) \\
&= \frac 12((Q_2 (\gamma_j \dd_i - \gamma_i \dd_j) \lambda) (\gamma^i \gamma^j Q_1) + (1 \leftrightarrow 2)) \\
&= \frac 12((Q_2 \gamma_j \dd_i \lambda) \gamma^i \gamma^j Q_1 + (Q_1 \gamma_j \dd_i \lambda) \gamma^i \gamma^j Q_2) - \frac 12((Q_2 \gamma_i \dd_j \lambda) \gamma^i \gamma^j Q_1 + (Q_1 \gamma_i \dd_j \lambda) \gamma^i \gamma^j Q_2) \\
&= \frac 12((Q_2 \gamma_j \dd_i \lambda) \gamma^i \gamma^j Q_1 + (Q_1 \gamma_j \dd_i \lambda) \gamma^i \gamma^j Q_2) + \frac 12((Q_2 \gamma_i \dd_j \lambda) \gamma^j \gamma^j Q_1 + (Q_1 \gamma_i \dd_j \lambda) \gamma^j \gamma^i Q_2) + \\
&\quad - \frac 12((Q_2 \gamma_i \dd_j \lambda) \delta_{ij} Q_1 + \frac 12(Q_1 \gamma_i \dd_j \lambda) \delta^{ij} Q_2) \\
&= ((Q_1 \gamma_j Q_2) (\gamma^i \gamma^j \dd_i \lambda) - \frac 12(Q_2 \gamma_i \dd_i \lambda)Q_1 - \frac 12(Q_1 \gamma_i \dd_i \lambda)Q_2
\end{align*}
using the fact that 
\[(\psi_1 \gamma_j \psi_2)(\gamma^j \psi_3) + (\psi_2 \gamma_j \psi_3)(\gamma^j \psi_1) + (\psi_3 \gamma_j \psi_1)(\gamma^j \psi_2) = 0,\]
as in \cite[Theorem 11]{BaezHuerta}.  Making one more simplification using the Clifford relations, we have
\begin{align*}
[\delta_{Q_1}, \delta_{Q_2}] \lambda &= ((Q_1 \gamma_j Q_2) (\delta^{ij} \dd_i \lambda) - ((Q_1 \gamma_j Q_2) (\gamma^j \gamma^i \dd_i \lambda) - \frac 12(Q_2 \gamma_i \dd_i \lambda)Q_1 - \frac 12(Q_1 \gamma_i \dd_i \lambda)Q_2 \\
&= \delta_{[Q_1,Q_2]} \lambda - \rho(\Gamma(Q_1,Q_2)) \sd \dd \lambda - \frac 12(Q_2, \sd \dd \lambda)Q_1 - \frac 12(Q_2, \sd \dd \lambda)Q_2.
\end{align*}
\end{proof}

In particular the supersymmetry action is a Lie algebra homomorphism only modulo the ideal generated by the equation of motion $\sd \dd \lambda = 0$.
In other words, this supersymmetry action only defined an action of the Lie algebra of supertranslations ``on-shell". 

\subsubsection{Off-shell formulation}

Based off the calculations in the previous subsection, we are motivated to write down the following $L_\infty$ action of the supersymmetry algebra on the space of BV fields.

The linear part of the action is simply the naive action we have already written down:
\begin{align*}
\delta^{(1)} \colon S_+ \otimes \left( \Omega^1(\RR^{10} ; \mf g) \oplus \Omega^0(\RR^{10}; \Pi S_+[-1] \otimes \mf g) \right) &\to \Omega^1(\RR^{10} ; \mf g) \oplus \Gamma(\RR^{10}; \Pi S_+[-1] \otimes \mf g) \\
Q \otimes (A + \lambda) &\mapsto  \Gamma(Q,\lambda) + \sd F_A Q .
\end{align*}
In other words, $\delta^{(1)}(Q \otimes (A + \lambda)) = \delta_Q A + \delta_Q\lambda $ in the notation of the previous subsection.
We define the following quadratic action as
\begin{align*}
\delta^{(2)} \colon S_+ \otimes S_+ \otimes \Omega^0(\RR^{10}; \Pi S_+) &\to \Omega^0(\RR^{10}; \Pi S_-' [-1]) \\
Q_1 \otimes Q_2 \otimes \lambda &\mapsto - \left(\rho(\Gamma(Q_1,Q_2)) \lambda^* + \frac 12 \left((Q_2, \lambda^*)Q_1 + (Q_1, \lambda^*)Q_2\right)\right) .
\end{align*}

The BV action involving the component $\Pi S_+$ of the supersymmetry algebra reads
\begin{align*}
I ^{(1)} (Q ; A, \lambda, A^*, \lambda^*) & = \<A^* , \Gamma(Q, \lambda)\> + \<\lambda^*, \sd F_A Q\> \\
I^{(2)} (Q_1,Q_2 ; \lambda^*) & = \pm \left\<\lambda^* \;,\; \rho(\Gamma(Q_1,Q_2)) \lambda^* + \frac 12 \left((Q_2, \lambda^*)Q_1 + (Q_1, \lambda^*)Q_2\right)\right\> .
\end{align*}

\begin{prop}
The functional
\[
\fS = S_{BV} + I^{\rm Bos} + I^{(1)} + I^{(2)} \in \clie^\bu(\fg_{\cN=1}) \otimes \cloc^\bu(\fL) [-1]
\]
satisfies the Maurer-Cartan equation
\[
\d_{\rm Lie}  \left( \fS \right) + \{S_{BV}, \fS\} + \frac{1}{2} \left\{\fS , \fS \right\} = 0 .
\]
\end{prop}

To state the above proposition differently, the functional $\fS$ prescribes an $L_\infty$ map
\[
\fS : \fg_{\cN=1} \rightsquigarrow \cloc^\bu(\fL) [-1]
\]
\brian{add more}

% \cN=(1, 0) SYM
% Construct the L_infinity action of supersymmetry

\subsection{Dimension 9}

% \cN=1 SYM

\subsection{Dimension 8}

% \cN=1 SYM

\subsection{Dimension 7}

% \cN=1 SYM

\subsection{Dimension 6}

% \cN=(1, 1) SYM

% \cN=(1, 0) SYM with matter in a pseudo-real representation. Construct the L_infinity action of supersymmetry

\subsection{Dimension 5}

% \cN=2 SYM
% \cN=1 SYM with matter in a pseudo-real representation.

\subsection{Dimension 4}

% \cN=4 SYM
% \cN=2 SYM with matter in a pseudo-real representation
% \cN=1 SYM with matter in a complex representation

\subsubsection{$\cN=1$ SYM with matter} 

Let $R$ be a complex representation of $\mf g$.

The matter sector of the theory is labeled by the BRST fields:
\begin{itemize}
\item Two scalars $\varphi_{\pm} \in C^\infty (\RR^4) \otimes R$;
\item Weyl fermions $\psi_{\pm} \in \Pi C^\infty(\RR^4) \otimes S_{\pm} \otimes R$ 
\end{itemize} 
which together define the (abelian) local Lie algebra
\[
L_{\rm matter}^{\cN=1} = C^\infty (\RR^4) \otimes R^{\oplus 2} \oplus  \Pi C^\infty(\RR^4) \otimes (S_{+}  \otimes R \oplus S_- \otimes R) .
\]
\subsection{Dimension 3}

% \cN=8 SYM
% \cN=4 SYM with matter in a pseudo-real representation
% \cN=2 SYM with matter in a complex representation

\subsection{Dimension 2}

% \cN=(8, 8) SYM
% \cN=(4, 4) SYM with matter in a pseudo-real representation
% \cN=(2, 2) SYM with matter in a complex representation

\pagestyle{bib}
\printbibliography

\end{document}
