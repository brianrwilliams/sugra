\documentclass[10pt, oneside]{article}

\input ./math_headers.sty

\newcommand{\C}{\mathbf{C}}
\newcommand{\dBRST}{\mathrm d_{\mathrm{BRST}}}
\renewcommand{\d}{\mathrm{d}}
\newcommand{\sD}{\slashed{D}}
\newcommand{\sdel}{\slashed{\partial}}
\newcommand{\map}{\ul{\mr{Map}}}
\newcommand{\cO}{\mathcal{O}}
\newcommand{\R}{\mathbf{R}}
\newcommand{\sdelbar}{\overline{\slashed{\partial}}}
\newcommand{\st}[1]{{}^*{#1}}

\newcommand{\brian}[1]{\textcolor{blue}{BW: #1}}

\addbibresource{Twist.bib}

\title{Notes on 10d Super Yang-Mills}
\author{Chris Elliott, Pavel Safronov and Brian Williams}

\date{\today}

\begin{document}

\maketitle

\section{Minimal Super Yang-Mills}
Our starting point will be the theory complexifying the usual 10d super Yang-Mills theory.  Fix a complex reductive gauge group $G$ with Lie algebra $\gg$.  The ordinary fields of super Yang-Mills theory on $\RR^{10}$ consist of a boson: a connection $A$ on the trivial $G$-bundle, and a fermion: a $\gg$-valued section $\lambda$ of the Weyl spinor bundle associated to the spinor representation $S_+$ \footnote{If we didn't complexify we would instead consider $G_\RR$ a compact connected Lie group, and a section of the Majorana-Weyl spinor bundle, which necessitates working in Lorentzian signature.  I think for our purposes it's interesting enough to just consider the complexified theory and avoid signature issues.  The complexified theory twists to holomorphic Chern-Simons theory with complex gauge group.}.  These fields are acted upon by the group of gauge transformations -- $G$-valued functions on $\RR^{10}$.

We can model the stack of fields modulo gauge transformations infinitesimally near the point $0$ by the corresponding BRST complex.  This is the local super Lie algebra
\[L_{\mr{BRST}} = \Omega^0(\RR^{10}; \gg) \to \Omega^1(\RR^{10}; \gg) \oplus \Omega^0(\RR^{10}; \Pi S_+ \otimes \gg)\]
with the de Rham differential, placed in cohomological degrees 0 and 1, with bracket induced from the Lie bracket on $\gg$.

The action functional in 10d super Yang-Mills is given by
\[S(A,\lambda) = \int_{\RR^{10}} \langle \frac{1}{2} F_A \wedge \ast F_A - (\lambda, \sd D_A \lambda)\rangle,\]
where $\langle - \rangle$ denotes an invariant pairing on $\gg$, and $(,)$ denotes a scalar-valued pairing $S_+ \otimes S_- \to \CC$ (there will be a unique such pairing, up to rescaling, characterized by the condition that $(\rho(v)\lambda_1,\rho(v)\lambda_2) = (\lambda_1,\lambda_2)$ for each $v \in \CC^{10}$, where $\rho$ denotes Clifford multiplication).

We can re-encode this data in terms of the classical BV complex (Phil and I wrote this down in \cite[Section 3.1]{ElliottYoo1}).  This is the $L_\infty$-algebra whose underlying cochain complex takes the form
\[\xymatrix{
\Omega^0(\RR^{10}; \gg) \ar[r]^{\d} &\Omega^1(\RR^{10}; \gg) \ar[r]^{\d \ast \d} &\Omega^9(\RR^{10}; \gg) \ar[r]^{\d} &\Omega^{10}(\RR^{10}; \gg) \\
&\Omega^0(\RR^{10}; \Pi S_- \otimes \gg) \ar[r]^{\ast \sd \d} &\Omega^{10}(\RR^{10}; \Pi S_- \otimes \gg), &
}\]
with degree $-3$ invariant pairing induced by the invariant pairing on $\gg$ and the pairing $(,)$ between $S_+$ and $S_-$, and with degree 2 and 3 brackets given by the action of $\Omega^0(\RR^{10}; \gg)$ on everything along with
\begin{align*}
\ell_2^{\mr{Bos}} \colon \Omega^1(\RR^{10};\gg) \otimes \Omega^1(\RR^{10};\gg) &\to \Omega^{9}(\RR^{10};\gg) \\
(A \otimes B) &\mapsto [A \wedge \ast \mr d B] + [\ast \mr d  A \wedge B] + \mathrm{d} \ast[A \wedge B] \\
\ell_2^{\mr{Fer}} \colon \Omega^1(\RR^{10};\gg) \otimes \Omega^0(\RR^{10}; S_{+} \otimes \gg) &\to \Omega^{10}(\RR^{10}; S_{-} \otimes \gg) \\
(A \otimes \lambda) &\mapsto \ast \sd A \lambda
\end{align*}
in degree 2, and the map
\begin{align*}
\ell_3 \colon \Omega^1(\RR^{10};\gg) \otimes \Omega^1(\RR^{10};\gg) \otimes \Omega^1(\RR^{10};\gg) &\to \Omega^{9}(\RR^{10};\gg) \\
(A \otimes B \otimes C) &\mapsto [A \wedge \ast[B \wedge C]] + [B \wedge \ast[C \wedge A]] + [C \wedge \ast[A \wedge B]]
\end{align*}
in degree 3.

\subsection{On-Shell Supersymmetry Action}
We can define an action of the 10d $\mc N=1$ supersymmetry algebra on the complex of BRST fields in this minimal super Yang-Mills theory.  The Poincar\'e action is clear, and the action of the supersymmetry $Q$ is generated -- in the usual Physics notation -- by the transformation \cite{BrinkSchwarzScherk}
\begin{align*}
\delta_Q A &= \Gamma(Q,\lambda) \\
\delta_Q \lambda &= \sd F_A Q,
\end{align*}
where the notation $\sd F_A$ stands for the iterated Clifford multiplication $\sd F_A = F_{ij} \gamma^i \gamma^j$.  To check that this defines an on-shell supersymmetry action we need to check it's compatible with the brackets in the supersymmetry algebra, up to terms in the ideal generated by the equations of motion.  So, we calculate
\begin{align*}
[\delta_{Q_1}, \delta_{Q_2}] A &= (\Gamma(Q_2,\sd F_A Q_1) - \Gamma(Q_1,\sd F_A Q_2)) \\
&=  F_{ij}(Q_2 \gamma^k \gamma^i \gamma^j Q_1 - Q_1 \gamma^k \gamma^i \gamma^j Q_2)\\
&=  F_{ij}(Q_2 \gamma^k \gamma^i \gamma^j Q_1 - Q_2 \gamma^j \gamma^i \gamma^k Q_1)\\
&=  F_{ij}(Q_2 \gamma^k \gamma^j \gamma^i Q_1 - Q_2 \gamma^j \gamma^k \gamma^i Q_1)\\
&=  F_{ij}\delta^{kj}(Q_2 \gamma^i Q_1) \\
&= \delta_{[Q_1, Q_2]} A,
\end{align*}
where on the third line we used the fact that the pairing $\Gamma(-,-)$ is symmetric -- i.e. that $\lambda_1 \gamma^i \lambda_2 = \lambda_2 \gamma^i \lambda_1$ -- three times, and on the fourth and fifth lines we used the Clifford relations.  Note that, on the gauge fields, the action is a Lie action on the nose, not only on-shell.  Similarly we can calculate, following the calculation in Guillen \cite{Guillen}:
\begin{align*}
[\delta_{Q_1}, \delta_{Q_2}] \lambda &= (\sd F_{\Gamma(Q_2, \lambda)} Q_1 + \sd F_{\Gamma(Q_1,\lambda)} Q_2) \\
&= \frac 12((Q_2 (\gamma_j \dd_i - \gamma_i \dd_j) \lambda) (\gamma^i \gamma^j Q_1) + (1 \leftrightarrow 2)) \\
&= \frac 12((Q_2 \gamma_j \dd_i \lambda) \gamma^i \gamma^j Q_1 + (Q_1 \gamma_j \dd_i \lambda) \gamma^i \gamma^j Q_2) - \frac 12((Q_2 \gamma_i \dd_j \lambda) \gamma^i \gamma^j Q_1 + (Q_1 \gamma_i \dd_j \lambda) \gamma^i \gamma^j Q_2) \\
&= \frac 12((Q_2 \gamma_j \dd_i \lambda) \gamma^i \gamma^j Q_1 + (Q_1 \gamma_j \dd_i \lambda) \gamma^i \gamma^j Q_2) + \frac 12((Q_2 \gamma_i \dd_j \lambda) \gamma^j \gamma^j Q_1 + (Q_1 \gamma_i \dd_j \lambda) \gamma^j \gamma^i Q_2) + \\
&\quad - \frac 12((Q_2 \gamma_i \dd_j \lambda) \delta_{ij} Q_1 + \frac 12(Q_1 \gamma_i \dd_j \lambda) \delta^{ij} Q_2) \\
&= ((Q_1 \gamma_j Q_2) (\gamma^i \gamma^j \dd_i \lambda) - \frac 12(Q_2 \gamma_i \dd_i \lambda)Q_1 - \frac 12(Q_1 \gamma_i \dd_i \lambda)Q_2
\end{align*}
using the fact that 
\[(\psi_1 \gamma_j \psi_2)(\gamma^j \psi_3) + (\psi_2 \gamma_j \psi_3)(\gamma^j \psi_1) + (\psi_3 \gamma_j \psi_1)(\gamma^j \psi_2) = 0,\]
as in \cite[Theorem 11]{BaezHuerta}.  Making one more simplification using the Clifford relations, we have
\begin{align*}
[\delta_{Q_1}, \delta_{Q_2}] \lambda &= ((Q_1 \gamma_j Q_2) (\delta^{ij} \dd_i \lambda) - ((Q_1 \gamma_j Q_2) (\gamma^j \gamma^i \dd_i \lambda) - \frac 12(Q_2 \gamma_i \dd_i \lambda)Q_1 - \frac 12(Q_1 \gamma_i \dd_i \lambda)Q_2 \\
&= \delta_{[Q_1,Q_2]} \lambda - \rho(\Gamma(Q_1,Q_2)) \sd \dd \lambda - \frac 12(Q_2, \sd \dd \lambda)Q_1 - \frac 12(Q_2, \sd \dd \lambda)Q_2.
\end{align*}
In particular the supersymmetry action is a Lie algebra homomorphism only modulo the ideal generated by the equation of motion $\sd \dd \lambda = 0$.

\begin{remark}
This suggests a second order correction to the supersymmetry action on the BV theory, which has the chance of closing off-shell.  Define a second order action depending on the antifield $\lambda^*$ to the gluino $\lambda$ by
\begin{align*}
\delta^{(2)} \colon S_+ \otimes S_+ \otimes \Gamma(\RR^{10}; \Pi S_-[-1]) &\to \Gamma(\RR^{10}; \Pi S_+) \\
Q_1 \otimes Q_2 \otimes \lambda^* &\mapsto - \left(\rho(\Gamma(Q_1,Q_2)) \lambda^* + \frac 12 \left((Q_2, \lambda^*)Q_1 + (Q_1, \lambda^*)Q_2\right)\right).
\end{align*}
\end{remark}


\subsection{$SU(5)$-invariant splitting}

We will need an expression for the Hodge star operator on K\"{a}hler manifolds, see \cite[Proposition 1.2.31]{Huybrechts}.

\begin{prop}
Let $M$ be a K\"{a}hler $d$-fold and decompose
\[\Omega^2(M; \C) = \Omega^{2, 0}(M)\oplus \Omega^{0, 2}(M)\oplus (\C\omega\oplus \Omega^{1, 1}_\perp(M)).\]
Then
\begin{enumerate}
\item The spaces $\Omega^{2, 0}(M)\oplus \Omega^{0, 2}(M)$, $\C\omega$ and $\Omega^{1, 1}_\perp(M)$ are mutually orthogonal.

\item For $\alpha\in\Omega^{2, 0}(M)\oplus \Omega^{0, 2}(M)$ we have
\[\ast \alpha = \frac{1}{(d-2)!} \alpha\wedge \omega^{d-2}.\]

\item For $\alpha\in \Omega^{1, 1}_\perp(M)$ we have
\[\ast\alpha = -\frac{1}{(d-2)!} \alpha\wedge \omega^{d-2}.\]

\item For $\alpha\in\C\omega$ we have
\[\ast \alpha = \frac{1}{(d-1)!} \alpha\wedge \omega^{d-2}.\]
\end{enumerate}
\end{prop}

\begin{corollary}
Let $M$ be a K\"{a}hler $d$-fold. Then
\[F\wedge \ast F + \frac{1}{(d-2)!} F\wedge F\wedge \omega^{d-2} = \left(4(F_{2, 0}, F_{0, 2}) + (\Lambda F_{1, 1})^2\right) \frac{\omega^d}{d!}.\]
\end{corollary}

Pick a reduction of the structure group to $SU(5)\subset SO(10)$. This gives rise to a K\"{a}hler structure $\omega$ on $\R^{10}$ and a nonvanishing holomorphic 5-form $\Omega$. We denote by $\Lambda\colon \Omega^k(M)\rightarrow \Omega^{k-2}(M)$ the dual Lefschetz operator defined by $(\Lambda \alpha, \beta) = (\alpha, \beta\wedge \omega)$.

The connection splits as $A = A_{0, 1} + A_{1, 0}$. The semi-spin representation decomposes as
\[S_+ \cong \C\oplus \Omega^{0, 2} \oplus \Omega^{1, 0},\]
which gives rise to a decomposition $\lambda = \chi + B + \psi$.

If $Q\in S_+$ is the scalar supercharge, the supersymmetry transformations become
\begin{align*}
\delta_Q A_{1, 0} &= \psi \\
\delta_Q B &= F_{0, 2} \\
\delta_Q \chi &= \frac{1}{2} \Lambda F_{1, 1}.
\end{align*}
Note that $\delta_Q^2 \chi = \frac{1}{2} \Lambda(\overline{\partial} \psi)$ and zero on other fields.

Let
\[S_{top} = \frac{1}{12}\int F\wedge F\wedge \omega^3\]
be the instanton number. Then we get
\[
S + S_{top} = \int \frac{\omega^5}{5!}\left(2(F_{2, 0}, F_{0, 2}) + \frac{1}{2} (\Lambda F_{1, 1})^2 - 2\Lambda(\overline{\partial}\psi) \chi - 2(B, \partial \psi)\right) - B\wedge \overline{\partial} B\wedge \Omega
\]

We then have
\[\delta_Q(S+S_{top}) = \int \frac{\omega^5}{5!}\left(2(F_{0, 2}, \partial \psi) + \Lambda F_{1, 1} \Lambda(\overline{\partial}\psi) - \Lambda F_{1, 1} \Lambda(\overline{\partial}\psi) - 2(F_{0, 2}, \partial \psi)\right) = 0.
\]

Note that $\delta_Q S_{top} = 0$ (it is a topological invariant). Therefore, $\delta_Q S = 0$. We have checked it for a pure spinor supercharge. The span of pure spinors in $S_+$ is an $SO(10)$-subrepresentation which by irreducibility has to be everything. Thus, any spinor is a linear combination of pure spinors and so $\delta_Q S = 0$ for any supercharge $Q$.

\subsection{An Off-Shell Supersymmetry Action?}
\brian{I realized I'm still confused with the role of Baulieu's $h$-field. 
I really suspect it's different than the usual auxiliary field story. 
It seems like Baulieu doesn't even write down the apparent off-shell SUSY module structure. 
}
\brian{I think without ever introducing an auxiliary field we can obtain the following module structure.}

\def\oloc{\mathcal{O}_{loc}}

Let $\oloc^{BRST}$ and $\oloc^{BV}$ be the local functionals for the BRST and BV fields, respectively. 
The BRST operator endows $\oloc^{BRST}$ with the structure of a cochain complex, and the BV operator together with the BV bracket endow $\oloc^{BV}[-1]$ with the structure of a dg Lie algebra. 
Note that there is a map of dg Lie algebras
\[
\oloc^{BV}[-1] \to {\rm End}(\oloc^{BRST})
\]
sending a functional $I$ to the endomorphism $\{I,-\}$.
\brian{No there is not.}

\def\cN{\mathcal{N}}

The cursory definition of the linear map
\[
\mf g_{\cN=1} \to {\rm End}(\oloc^{BRST})
\]
is not a map of Lie algebras. 
It fails to preserve the Lie bracket by a term proportional to the equations of motion in the field $\lambda$. 
I claim that Baulieu's introduction of the $h$-field does not resolve this issue. (We can see this by counting fermion number. 
The space where the failure for this to be a Lie map is $S_+$, yet we are only adding a scalar $h$ in the auxiliary, so there's no way.

Instead, what one should try is the following. 
(This doesn't ever use a holomorphic language, so maybe I'm making a silly mistake.)
There is an obvious lift of the linear map $\mf g_{\cN=1} \to {\rm End}(\oloc^{BRST})$ to the BV complex
\[
\mf g_{\cN=1} \to \oloc^{BV} [-1] .
\]
%The prescription for defining this map is simple. 
This is still, of course, not a map of dg Lie algebras.
However, I claim that there is an $L_\infty$ correction to this map. 

Fix a basis $\{Q_\alpha\}$ of $S_+$ and write a general element of the form $Q_{\alpha} = \epsilon^\alpha Q_\alpha$.
The putative, linear action, of the element $Q$ on the BV complex is through functionals of the form
\[
\int \epsilon \lambda^* \sd F_A 
\]
where $\lambda^*$ denotes the anti-field to $\lambda$. 
I claim that we can correct the action by adding a quadratic term to the action that sends a pair $Q_1 \otimes Q_2$ to the functional
\[
\int \epsilon_1 \lambda^* \epsilon_2 \lambda^*  .
\] 

In terms of an $L_\infty$ action, I'm saying that there is the usual linear map
\[
I^{(1)} : \mf g_{\cN=1} \to \oloc^{BV} 
\]
plus an $L_\infty$ correction of the form:
\[
I^{(2)} : \mf g_{\cN=1}^{\otimes 2} \to \oloc^{BV} [-2]  .
\]
You may ask why we have to stop there, but I don't have a great conceptual answer besides the computational fact that it seems like it works out. o
I claim that $I^{(1)} + I^{(2)}$ determines an off-shell action of $\mf g_{\cN=1}$ on the BV complex. 

At the level of generators, we are proposing a new transformation law of the form 
\begin{align*}
\delta A &= \d c + \Gamma(\epsilon,\lambda) \\
\delta \lambda &= [c,\lambda] + \epsilon \sd F_A  \pm \# \epsilon (\epsilon \lambda^*)  \\
\delta \lambda^* &= \sd \partial \lambda + [c, \lambda^*] + \epsilon \sd A^*  .
\end{align*}
I've included terms coming from the BV operator (which have no $\epsilon$ dependence). 
The only new term is the last one in the transformation law for $\lambda$, $\epsilon (\epsilon \lambda^*)$. 

\chris{
I'm a little confused by a few things: can I ask for clarification?
\begin{enumerate}
 \item Firstly, while I understand the map $(I \mapsto \{I, -\}) \colon \oloc^{BV}[-1] \to {\rm End}(\oloc^{BV})$, I don't understand how it lifts to ${\rm End}(\oloc^{BRST})$.  For instance (maybe a little heuristically, but I hope what I'm trying to say makes sense), let $\alpha$ be a scalar field and let $\alpha^*$ be its antifield.  Let $I = (\alpha^*)^2$ in $\oloc^{BV}$.  Then $\{(\alpha^*)^2, \alpha\} = 2\alpha^*$ is a non-trivial element of $\oloc^{BV}$ which is not in the image of $\oloc^{BRST}$, which says that my na\"ive guess for how to define your map doesn't work, so I must be supposed to do something more clever?
 \item I wanted to clarify something about your ``counting fermion number'' argument.  I thought that this argument only applied to the action of a whole supersymmetry algebra, not a single square zero supercharge.  That is, if we're trying to construct off-shell BRST supersymmetry then the space of BRST fields should yield a representation of the supersymmetry algebra.  Representations $V_0 \oplus V_1$ of the supersymmetry algebra have to have the same number of bosonic and fermionic degrees of freedom -- i.e. $\dim V_0 = \dim V_1$, because we can choose a supertranslation $Q$ which squares to a translation.  So if $\alpha$ denotes the supersymmetry algebra action, the composite of $\alpha(Q)|_{V_0} \colon V_0 \to V_1$ and $\alpha(Q)|_{V_1} \colon V_1 \to V_0$ is equal to the action of a translation, which is an isomorphism, and therefore $\alpha(Q)$ must also be an isomorphism, so $\dim V_0 = \dim V_1$.  However this argument doesn't apply anymore if you just want an action of a single square-zero odd symmetry.  Maybe you have something else in mind though?
 \item Could you include some of the calculation that the map you define at the end is an $L_\infty$ map?  This seems like it might be exactly what we're looking for.
\end{enumerate}
}

\brian{
\begin{enumerate}
\item I was being sloppy, and you are right. 
Think of BRST as functions on $B \mf g$ and BV as functions on $T^* B\mf g$.
Since vector fields include inside poly-vector fields there is a map of dg Lie algebras
\[
{\rm Der}(\cO(B \mf g)) = {\rm Vect}(B \mf g) \to \cO(T^* B \mf g) .
\]
So, really, the map should go the other way.
I'll fix the above when I have more time, but basically what I'm saying is that even though we can't write down a Lie map to derivations of BRST, we can map one to the bigger BV complex.
That's why the action looks like its by poly-vector fields, not just derivations. 
\item
I think I was also being sloppy here, and I don't see a way of making my argument precise. 
I will try to find the reference (I think it may be Berkovits) where he says something to this effect. 
\item
OK, I wanted to see if you thought it was reasonable before including the details. 
I'll work on writing that up now!
\end{enumerate}
}

\subsubsection{Abelian case}

The BV differential is
\begin{align*}
\delta_{BV} c &= 0 \\
\delta_{BV} A &= \d c \\
\delta_{BV} \lambda &= 0 \\
\delta_{BV} \lambda^* &= \sd D_A \lambda \\
\delta_{BV} A^* &= \d\ast F \\
\delta_{BV} c^* &= \d A^*
\end{align*}

\[I^{(1)} = \int (\lambda^*, \sd F \epsilon) + \int A^* \Gamma(\epsilon, \lambda)\]

\[\delta_{BV} I^{(1)} = \int (\sd D_A\lambda, \sd F\epsilon) + \int (\d\ast F) \Gamma(\epsilon, \lambda).\]
\section{Baulieu's 10 Model}

Baulieu \cite{Baulieu} considers an extension of 10d Super-Yang Mills including an auxiliary $\gg$-valued scalar field $h$.  The inclusion of this field breaks the manifest $\SO(10)$ symmetry to the subgroup $\SU(5)$, corresponding to a choice of complex structure on $\RR^{10}$.  Let's describe this classical theory in an explicitly $\SU(5)$-invariant way.  With respect to this choice of complex structure $\CC^5$ becomes Calabi-Yau: we'll denote the holomorphic top-form by $\Omega$ and the map $\Omega^{p,q}(\CC^5) \to \Omega^{p-1,q-1}(\CC^5)$ induced from the K\"ahler structure by $J$.

The ordinary fields $A$ and $\lambda$ of super Yang-Mills will decompose according to the decomposition of the vector and Weyl spinor representations of $\SO(10)$ into irreducible $\SU(5)$-representations.  So, explicitly $A$ splits up into fields
\[A_{1,0} + A_{0,1} \in \Omega^{1,0}(\CC^5; \gg) \oplus \Omega^{0,1}(\CC^5; \gg),\]
and $\lambda$ splits up into fields
\[\chi + \psi + B \in \Pi(\Omega^0(\CC^5; \gg) \oplus \Omega^{1,0}(\CC^5; \gg) \oplus \Omega^{0,2}(\CC^5; \gg)).\]
In terms of these fields, and including the auxiliary field $h$, the action functional becomes
\[S(A_{1,0}, A_{0,1}, \chi, \psi, B, h) = \int \langle J^2(F_{2, 0}\wedge F_{0,2})\Omega - \frac 12 \|h\|^2\Omega + h J(F_{1,1}) \Omega - J( \chi \wedge (\ol \dd_{A_{0,1}} \psi)) \Omega - J^2(B \wedge (\dd_{A_{1,0}} \psi) - (B \wedge \ol \dd_{A_{0,1}} B)) \Omega \rangle.\]
One derives this action functional by decomposing the 10d super Yang-Mills action functional into $\SU(5)$ irreducible component fields, then introducing a Lagrange multiplier $h$ to eliminate the $F_{1,1}^2$ term.
%SU(5) BRST and BV Fields, Action Functional, BV Complex

\subsection{Off-Shell Action of a Scalar Supersymmetry}

Consider the following square-zero supersymmetry acting on Baulieu's theory:
\begin{align*}
\delta A_{1,0} &= \psi \\
\delta \chi &= h \\
\delta B &= F_{0,2} \\
\end{align*}
acting trivially on all other (BRST) fields.  This clearly squares to zero, so defines an action of the supergroup $\Pi \CC$ on the BRST complex.  We can calculate the variation of the action functional with respect to this symmetry, and see that it vanishes.  Indeed,
\begin{align*}
\delta S &= \int \delta \langle J^2(F_{2, 0}\wedge F_{0,2})\Omega - \frac 12 \|h\|^2\Omega + h J(F_{1,1}) \Omega - J( \chi \wedge (\ol \dd_{A_{0,1}} \psi)) \Omega - J^2(B \wedge (\dd_{A_{1,0}} \psi) - (B \wedge \ol \dd_{A_{0,1}} B)) \Omega \rangle \\
&= \int \langle J^2(\dd_{A_{1,0}} \psi \wedge F_{0,2})\Omega - 0 + h J(\ol \dd_{A_{0,1}} \psi) \Omega - J( h (\ol \dd_{A_{0,1}} \psi)) \Omega - J^2(F_{0,2} \wedge \dd_{A_{1,0}} \psi)\Omega - 0\rangle \\
&= 0.
\end{align*}
This should mean that the $\Pi \CC$ action is off-shell, so extends to an action on the full BV complex (the above calculation should be equivalent to showing that $\delta$ commutes with the BV differential).

Let me also explicitly show that the BV action functional is also invariant for this square-zero supersymmetry.  So we also include the dual action
\begin{align*}
\delta \psi^* &= A_{1,0}^* \\
\delta h^* &= \chi^* \\
\delta A_{0,1}^* &= \ol \dd_{A_{0,1}} B^*.
\end{align*}
The variation of the part of the BV action functional involving the anti-fields is then calculated to vanish as follows.
\begin{align*}
&\!\!\! \delta S_{\mr{anti}} = \\
&= \int \delta \left(c^* \wedge [c, c] + {A^*_{1, 0}}\wedge (\partial c + [c,A_{1,0}]) + {A^*_{0, 1}}\wedge (\ol \dd c + [c, A_{0, 1}]) - B^*\wedge [c, B] - \chi^*\wedge [c, \chi] - \psi^*\wedge [c, \psi] \right) + h^* \wedge [c,h] \\
&= \int 0 +A^*_{1,0} \wedge [c,\psi] + \ol \dd_{A_{0,1}} B^* \wedge [c,A_{0,1}] - B^* \wedge [c,F_{0,2}] - \chi^* \wedge [c,h] - A_{1,0}^* \wedge [c,\psi] + \chi^* \wedge [c,h] \\
&= 0.
\end{align*}

\begin{remark}
If we don't adjoin the auxiliary field $h$, but do break the $\spin(10)$ symmetry down to $\SU(5)$ and consider the scalar supersymmetry with respect to the on-shell action from the previous section, then the BRST action is still preserved.  The supersymmetries are as above, except that now $\delta \chi = JF_{1,1)}$.  We have, very similarly to the calculation above,
\begin{align*}
\delta S &= \int \delta \langle J^2(F_{2, 0}\wedge F_{0,2})\Omega + J^2(F_{1,1}^2) \Omega - J( \chi \wedge (\ol \dd_{A_{0,1}} \psi)) \Omega - J^2(B \wedge (\dd_{A_{1,0}} \psi) - (B \wedge \ol \dd_{A_{0,1}} B)) \Omega \rangle \\
&= \int \langle J^2(\dd_{A_{1,0}} \psi \wedge F_{0,2})\Omega - 0 + J^2(F_{1,1} \wedge \ol \dd_{A_{0,1}} \psi) \Omega - J^2( F_{1,1} \wedge (\ol \dd_{A_{0,1}} \psi)) \Omega - J^2(F_{0,2} \wedge \dd_{A_{1,0}} \psi)\Omega - 0 \rangle \\
&= 0.
\end{align*}
\chris{Actually maybe this doesn't quite work, because of a factor of two.  We had $\delta(hJF_{1,1}) = hJ(\ol \dd \psi)$, whereas $\delta(J^2F_{1,1}^2) = 2 J^2(F_{1,1} \wedge \ol \dd \psi)$, and that extra 2 seems to break the cancellation.}

On the other hand, if try to extend this to the BV fields, we must have $\delta A^*_{1,0} = 0$ so that $\delta^2 = 0$, so we would need to modify $\delta A^*_{0,1}$ to include some term cancelling the term $\delta(\chi^*\wedge [c, \chi]) = \chi^* \wedge [c, JF_{1,1}]$, but without introducing a new term $\delta(A_{0,1}^* \wedge \ol \dd c)$.  This na\"ively doesn't look like it's possible.

\end{remark}




\section{Homotopy Data}
%We were going to check in particular that everything was a chain map

\subsection{Homotopy Transfer of the Scalar Supersymmetry}



\pagestyle{bib}
\printbibliography

\end{document}