\documentclass[10pt, oneside]{article}

\input ./math_headers.sty

\newcommand{\map}{\ul{\mr{Map}}}
\newcommand{\dBRST}{\mathrm d_{\mathrm{BRST}}}
\renewcommand{\d}{\mathrm{d}}
\newcommand{\sD}{\slashed{D}}
\newcommand{\sdel}{\slashed{\partial}}
\newcommand{\sdelbar}{\overline{\slashed{\partial}}}
\newcommand{\st}[1]{{}^*{#1}}

\addbibresource{Twist.bib}

\title{Notes on 10d Super Yang-Mills}
\author{Chris Elliott, Pavel Safronov and Brian Williams}
\date{\today}

\begin{document}

\maketitle

\section{Minimal Super Yang-Mills}
%BRST Fields, Action Functional, BV Complex

\subsection{On-Shell Supersymmetry Action}
%Failure to commute with the classical BV differential

\section{Baulieu's 10 Model}
%SU(5) BRST and BV Fields, Action Functional, BV Complex

\subsection{Off-Shell Action of a Scalar Supersymmetry}
%Does it do anything interesting to the anti-fields?

\section{Homotopy Data}
%We were going to check in particular that everything was a chain map

\subsection{Homotopy Transfer of the Scalar Supersymmetry}



\pagestyle{bib}
\printbibliography

\end{document}