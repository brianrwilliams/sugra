\documentclass[10pt, oneside]{article}

\input ./math_headers.sty

\newcommand{\map}{\ul{\mr{Map}}}
\newcommand{\dBRST}{\mathrm d_{\mathrm{BRST}}}
\renewcommand{\d}{\mathrm{d}}
\newcommand{\sD}{\slashed{D}}
\newcommand{\sdel}{\slashed{\partial}}
\newcommand{\sdelbar}{\overline{\slashed{\partial}}}
\newcommand{\st}[1]{{}^*{#1}}

\addbibresource{Twist.bib}

\title{Notes on 10d Super Yang-Mills}
\author{Chris Elliott, Pavel Safronov and Brian Williams}
\date{\today}

\begin{document}

\maketitle

\section{Minimal Super Yang-Mills}
Our starting point will be the theory complexifying the usual 10d super Yang-Mills theory.  Fix a complex reductive gauge group $G$ with Lie algebra $\gg$.  The ordinary fields of super Yang-Mills theory on $\RR^{10}$ consist of a boson: a connection $A$ on the trivial $G$-bundle, and a fermion: a $\gg$-valued section $\lambda$ of the Weyl spinor bundle associated to the spinor representation $S_+$ \footnote{If we didn't complexify we would instead consider $G_\RR$ a compact connected Lie group, and a section of the Majorana-Weyl spinor bundle, which necessitates working in Lorentzian signature.}.  These fields are acted upon by the group of gauge transformations -- $G$-valued functions on $\RR^{10}$.

We can model the stack of fields modulo gauge transformations infinitesimally near the point $0$ by the corresponding BRST complex.  This is the local super Lie algebra
\[L_{\mr{BRST}} = \Omega^0(\RR^{10}; \gg) \to \Omega^1(\RR^{10}; \gg) \oplus \Omega^0(\RR^{10}; \Pi S_+ \otimes \gg)\]
with the de Rham differential, placed in cohomological degrees 0 and 1, with bracket induced from the Lie bracket on $\gg$.

The action functional in 10d super Yang-Mills is given by
\[S(A,\lambda) = \int_{\RR^{10}} \langle F_A \wedge \ast F_A + (\lambda, \sd D_A \lambda)\rangle,\]
where $\langle - \rangle$ denotes an invariant pairing on $\gg$, and $(,)$ denotes a scalar-valued pairing $S_+ \otimes S_- \to \CC$ (there will be a unique such pairing, up to rescaling, characterized by the condition that $(\rho(v)\lambda_1,\rho(v)\lambda_2) = (\lambda_1,\lambda_2)$ for each $v \in \CC^{10}$, where $\rho$ denotes Clifford multiplication).

We can re-encode this data in terms of the classical BV complex.  This is the $L_\infty$-algebra whose underlying cochain complex takes the form
\[\xymatrix{
\Omega^0(\RR^{10}; \gg) \ar[r]^{\d} &\Omega^1(\RR^{10}; \gg) \ar[r]^{\d \ast \d} &\Omega^9(\RR^{10}; \gg) \ar[r]^{\d} &\Omega^{10}(\RR^{10}; \gg) \\
&\Omega^0(\RR^{10}; \Pi S_- \otimes \gg) \ar[r]^{\ast \sd \d} &\Omega^{10}(\RR^{10}; \Pi S_- \otimes \gg), &
}\]
with degree $-3$ invariant pairing induced by the invariant pairing on $\gg$ and the pairing $(,)$ between $S_+$ and $S_-$, and with degree 2 and 3 brackets given by the action of $\Omega^0(\RR^{10}; \gg)$ on everything along with
\begin{align*}
\ell_2^{\mr{Bos}} \colon \Omega^1(\RR^{10};\gg) \otimes \Omega^1(\RR^{10};\gg) &\to \Omega^{9}(\RR^{10};\gg) \\
(A \otimes B) &\mapsto [A \wedge \ast \mr d B] + [\ast \mr d  A \wedge B] + \mathrm{d} \ast[A \wedge B] \\
\ell_2^{\mr{Fer}} \colon \Omega^1(\RR^{10};\gg) \otimes \Omega^0(\RR^{10}; S_{+} \otimes \gg) &\to \Omega^{10}(\RR^{10}; S_{-} \otimes \gg) \\
(A \otimes \lambda) &\mapsto \ast \sd A \lambda
\end{align*}
in degree 2, and the map
\begin{align*}
\ell_3 \colon \Omega^1(\RR^{10};\gg) \otimes \Omega^1(\RR^{10};\gg) \otimes \Omega^1(\RR^{10};\gg) &\to \Omega^{9}(\RR^{10};\gg) \\
(A \otimes B \otimes C) &\mapsto [A \wedge \ast[B \wedge C]] + [B \wedge \ast[C \wedge A]] + [C \wedge \ast[A \wedge B]]
\end{align*}
in degree 3.


\subsection{On-Shell Supersymmetry Action}
%Failure to commute with the classical BV differential

\section{Baulieu's 10 Model}
%SU(5) BRST and BV Fields, Action Functional, BV Complex

\subsection{Off-Shell Action of a Scalar Supersymmetry}
%Does it do anything interesting to the anti-fields?

\section{Homotopy Data}
%We were going to check in particular that everything was a chain map

\subsection{Homotopy Transfer of the Scalar Supersymmetry}



\pagestyle{bib}
\printbibliography

\end{document}