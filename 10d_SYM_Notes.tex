\documentclass[10pt, oneside]{article}

\input ./math_headers.sty

\newcommand{\map}{\ul{\mr{Map}}}
\newcommand{\dBRST}{\mathrm d_{\mathrm{BRST}}}
\renewcommand{\d}{\mathrm{d}}
\newcommand{\sD}{\slashed{D}}
\newcommand{\sdel}{\slashed{\partial}}
\newcommand{\sdelbar}{\overline{\slashed{\partial}}}
\newcommand{\st}[1]{{}^*{#1}}

\addbibresource{Twist.bib}

\title{Notes on 10d Super Yang-Mills}
\author{Chris Elliott, Pavel Safronov and Brian Williams}
\date{\today}

\begin{document}

\maketitle

\section{Minimal Super Yang-Mills}
Our starting point will be the theory complexifying the usual 10d super Yang-Mills theory.  Fix a complex reductive gauge group $G$ with Lie algebra $\gg$.  The ordinary fields of super Yang-Mills theory on $\RR^{10}$ consist of a boson: a connection $A$ on the trivial $G$-bundle, and a fermion: a $\gg$-valued section $\lambda$ of the Weyl spinor bundle associated to the spinor representation $S_+$ \footnote{If we didn't complexify we would instead consider $G_\RR$ a compact connected Lie group, and a section of the Majorana-Weyl spinor bundle, which necessitates working in Lorentzian signature.  I think for our purposes it's interesting enough to just consider the complexified theory and avoid signature issues.  The complexified theory twists to holomorphic Chern-Simons theory with complex gauge group.}.  These fields are acted upon by the group of gauge transformations -- $G$-valued functions on $\RR^{10}$.

We can model the stack of fields modulo gauge transformations infinitesimally near the point $0$ by the corresponding BRST complex.  This is the local super Lie algebra
\[L_{\mr{BRST}} = \Omega^0(\RR^{10}; \gg) \to \Omega^1(\RR^{10}; \gg) \oplus \Omega^0(\RR^{10}; \Pi S_+ \otimes \gg)\]
with the de Rham differential, placed in cohomological degrees 0 and 1, with bracket induced from the Lie bracket on $\gg$.

The action functional in 10d super Yang-Mills is given by
\[S(A,\lambda) = \int_{\RR^{10}} \langle F_A \wedge \ast F_A + (\lambda, \sd D_A \lambda)\rangle,\]
where $\langle - \rangle$ denotes an invariant pairing on $\gg$, and $(,)$ denotes a scalar-valued pairing $S_+ \otimes S_- \to \CC$ (there will be a unique such pairing, up to rescaling, characterized by the condition that $(\rho(v)\lambda_1,\rho(v)\lambda_2) = (\lambda_1,\lambda_2)$ for each $v \in \CC^{10}$, where $\rho$ denotes Clifford multiplication).

We can re-encode this data in terms of the classical BV complex (Phil and I wrote this down in \cite[Section 3.1]{ElliottYoo1}).  This is the $L_\infty$-algebra whose underlying cochain complex takes the form
\[\xymatrix{
\Omega^0(\RR^{10}; \gg) \ar[r]^{\d} &\Omega^1(\RR^{10}; \gg) \ar[r]^{\d \ast \d} &\Omega^9(\RR^{10}; \gg) \ar[r]^{\d} &\Omega^{10}(\RR^{10}; \gg) \\
&\Omega^0(\RR^{10}; \Pi S_- \otimes \gg) \ar[r]^{\ast \sd \d} &\Omega^{10}(\RR^{10}; \Pi S_- \otimes \gg), &
}\]
with degree $-3$ invariant pairing induced by the invariant pairing on $\gg$ and the pairing $(,)$ between $S_+$ and $S_-$, and with degree 2 and 3 brackets given by the action of $\Omega^0(\RR^{10}; \gg)$ on everything along with
\begin{align*}
\ell_2^{\mr{Bos}} \colon \Omega^1(\RR^{10};\gg) \otimes \Omega^1(\RR^{10};\gg) &\to \Omega^{9}(\RR^{10};\gg) \\
(A \otimes B) &\mapsto [A \wedge \ast \mr d B] + [\ast \mr d  A \wedge B] + \mathrm{d} \ast[A \wedge B] \\
\ell_2^{\mr{Fer}} \colon \Omega^1(\RR^{10};\gg) \otimes \Omega^0(\RR^{10}; S_{+} \otimes \gg) &\to \Omega^{10}(\RR^{10}; S_{-} \otimes \gg) \\
(A \otimes \lambda) &\mapsto \ast \sd A \lambda
\end{align*}
in degree 2, and the map
\begin{align*}
\ell_3 \colon \Omega^1(\RR^{10};\gg) \otimes \Omega^1(\RR^{10};\gg) \otimes \Omega^1(\RR^{10};\gg) &\to \Omega^{9}(\RR^{10};\gg) \\
(A \otimes B \otimes C) &\mapsto [A \wedge \ast[B \wedge C]] + [B \wedge \ast[C \wedge A]] + [C \wedge \ast[A \wedge B]]
\end{align*}
in degree 3.

\subsection{On-Shell Supersymmetry Action}
We can define an action of the 10d $\mc N=1$ supersymmetry algebra on the complex of BRST fields in this minimal super Yang-Mills theory.  The Poincar\'e action is clear, and the action of the supersymmetry $Q$ is generated -- in the usual Physics notation -- by the transformation \cite{BrinkSchwarzScherk}
\begin{align*}
\delta_Q A &= \Gamma(Q,\lambda) \\
\delta_Q \lambda &= \sd F_A Q,
\end{align*}
where the notation $\sd F_A$ stands for the iterated Clifford multiplication $\sd F_A = F_{ij} \gamma^i \gamma^j$.  To check that this defines a supersymmetry action we need to check it's compatible with the brackets in the supersymmetry algebra.  So, we calculate
\begin{align*}
[\delta_{Q_1}, \delta_{Q_2}] A &= (\Gamma(Q_2,\sd F_A Q_1) - \Gamma(Q_1,\sd F_A Q_2)) \\
&=  F_{ij}(Q_2 \gamma^k \gamma^i \gamma^j Q_1 - Q_1 \gamma^k \gamma^i \gamma^j Q_2)\\
&=  F_{ij}(Q_2 \gamma^k \gamma^i \gamma^j Q_1 - Q_2 \gamma^j \gamma^i \gamma^k Q_1)\\
&=  F_{ij}(Q_2 \gamma^k \gamma^j \gamma^i Q_1 - Q_2 \gamma^j \gamma^k \gamma^i Q_1)\\
&=  F_{ij}\delta^{kj}(Q_2 \gamma^i Q_1) \\
&= \delta_{[Q_1, Q_2]} A,
\end{align*}
where on the third line we used the fact that the pairing $\Gamma(-,-)$ is symmetric -- i.e. that $\lambda_1 \gamma^i \lambda_2 = \lambda_2 \gamma^i \lambda_1$ -- three times, and on the fourth and fifth lines we used the Clifford relations.  Similarly we can calculate
\begin{align*}
[\delta_{Q_1}, \delta_{Q_2}] \lambda &= (\sd F_{\Gamma(Q_2, \lambda)} Q_1 - \sd F_{\Gamma(Q_1,\lambda)} Q_2) \\
&= (Q_2 \gamma_j \dd_i \lambda + [Q_2\gamma_i \lambda, Q_2 \gamma_j \lambda]) (\gamma^i \gamma^j Q_1) - (1 \leftrightarrow 2) \\
&= 
\end{align*}

\begin{remark}
I'm part way through trying to do these calculations myself, but for reference I found these calculations discussed in the master's thesis \cite{Guillen} we've discussed before.  For instance for the commutator of two supersymmetries acting on a spinor see equation 2.1.13.
\end{remark}



%Failure to commute with the classical BV differential

\section{Baulieu's 10 Model}

Baulieu \cite{Baulieu} considers an extension of 10d Super-Yang Mills including an auxiliary $\gg$-valued scalar field $h$.  The inclusion of this field breaks the manifest $\SO(10)$ symmetry to the subgroup $\SU(5)$, corresponding to a choice of complex structure on $\RR^{10}$.  Let's describe this classical theory in an explicitly $\SU(5)$-invariant way.  With respect to this choice of complex structure $\CC^5$ becomes Calabi-Yau: we'll denote the holomorphic top-form by $\Omega$ and the map $\Omega^{p,q}(\CC^5) \to \Omega^{p-1,q-1}(\CC^5)$ induced from the K\"ahler structure by $J$.

The ordinary fields $A$ and $\lambda$ of super Yang-Mills will decompose according to the decomposition of the vector and Weyl spinor representations of $\SO(10)$ into irreducible $\SU(5)$-representations.  So, explicitly $A$ splits up into fields
\[A_{1,0} + A_{0,1} \in \Omega^{1,0}(\CC^5; \gg) \oplus \Omega^{0,1}(\CC^5; \gg),\]
and $\lambda$ splits up into fields
\[\chi + \psi + B \in \Pi(\Omega^0(\CC^5; \gg) \oplus \Omega^{1,0}(\CC^5; \gg) \oplus \Omega^{0,2}(\CC^5; \gg)).\]
In terms of these fields, and including the auxiliary field $h$, the action functional becomes
\[S(A_{1,0}, A_{0,1}, \chi, \psi, B, h) = \int \langle (B \wedge \ol \dd_{A_{0,1}} B) + J^2(F_{2, 0}\wedge F_{0,2})\Omega + \|h\|^2\Omega + h J(F_{1,1}) \Omega + J( \chi \wedge (\ol \dd_{A_{0,1}} \psi)) \Omega + J^2(B \wedge (\dd_{A_{1,0}} \psi)) \Omega \rangle.\]
One derives this action functional by decomposing the 10d super Yang-Mills action functional into $\SU(5)$ irreducible component fields, then introducing a Lagrange multiplier $h$ to eliminate the $F_{1,1}^2$ term.
%SU(5) BRST and BV Fields, Action Functional, BV Complex

\subsection{Off-Shell Action of a Scalar Supersymmetry}
%Does it do anything interesting to the anti-fields?

\section{Homotopy Data}
%We were going to check in particular that everything was a chain map

\subsection{Homotopy Transfer of the Scalar Supersymmetry}



\pagestyle{bib}
\printbibliography

\end{document}