\documentclass[10pt, oneside]{article}

\input ./combined_macros.sty

\newcommand{\Cl}{\mathrm{Cl}}
\newcommand{\Dens}{\mathrm{Dens}}

\addbibresource{Twist.bib}

\usepackage{pdflscape}
\usetikzlibrary{shapes.geometric, arrows, positioning}

\tikzstyle{s16} = [rectangle, rounded corners, minimum width=1.8cm, minimum height=1cm,text centered, draw=black,fill=red!30]
\tikzstyle{s16chiral} = [s16, dashed]
\tikzstyle{s8 } = [rectangle, rounded corners, minimum width=1.8cm, minimum height=1cm,text centered, draw=black,fill=orange!30]
\tikzstyle{s4} = [rectangle, rounded corners, minimum width=1.8cm, minimum height=1cm,text centered, draw=black,fill=yellow!30]
\tikzstyle{s2chiral} = [rectangle, dashed, rounded corners, minimum width=1.8cm, minimum height=1cm,text centered, draw=black,fill=green!30]
\tikzstyle{dimension} = [circle, text centered, text width=0.7cm, minimum height=0.7cm, draw=black]
\tikzstyle{arrow} = [thick,->,>=stealth]

\title{4d $\cN=1$}
\author{Chris Elliott\and Pavel Safronov \and Brian Williams}

\date{\today}

\begin{document}

\maketitle

\subsection{Supersymmetric coupling}
\subsubsection{The 6d vector multiplet coupled to the hypermultiplet}

Let $S_{\rm gauge}$ and $S_{\rm matter}$ be the BV actions for the $6d$ $\cN=(1,0)$ vector and hyper multiplets, respectively. 
The coupled theory has action $S_{\rm gauge} + S_{\rm matter} + I_{\rm couple}$ where
\begin{align*}
I_{\rm couple} = & - g (\psi, [A, \psi])_U - g (\d_{A} \phi, [A,\phi])_U \\ + & g (\psi, [\lambda, \phi])_U + g (\phi, (\lambda, \psi))_U + g (\lambda, \mu(\phi, \psi))_\fg \\ + & g (\psi^*, [c, \psi])_U + g (\phi^*, [c, \phi])_U \\ + & g^2 \left(\mu(\phi, \phi), \mu(\phi, \phi) \right)_\fg .
\end{align*}
 \brian{I need to nail down this Yukawa}
\brian{
It looks like we need to choose a moment map $\mu : U^{\otimes 2} \to \fg^*$. }

The action of supersymmetry on the pure vector multiplet was constructed in Section \ref{sec: vector} and was encoded by linear and quadratic functionals that we denote $I^{(1)}_{\rm gauge}$ and $I^{(2)}_{\rm gauge}$. 
The action of supersymmetry on the pure hyper multiplet was constructed in Section \ref{sec: hyper} and was encoded by linear and quadratic functionals that we denote $I^{(1)}_{\rm matter}$ and $I^{(2)}_{\rm matter}$. 
Additionally, in the coupled theory, there are the following new terms in the action of supersymmetry:
\begin{align*}
I^{(1)}_{\rm couple} (Q) & = \int g (\psi^*, \rho([A, \phi]) Q)_U + \frac{1}{2} g (\lambda^*, Q \mu(\phi))_\fg \\
I^{(2)} (Q_1 , Q_2) & = 0 
\end{align*}
\brian{It appears there are no additional $L_\infty$ corrections}

\begin{thm}
Let $S_{\rm BV} = S_{\rm gauge} + S_{\rm matter} + I_{\rm couple}$ be the full BV action for the coupled vector to hyper multiplet and let
\begin{align*}
I^{(1)} & = I^{(1)}_{\rm gauge} + I^{(1)}_{\rm matter} + I^{(1)}_{\rm couple} \\
I^{(2)} & = I^{(2)}_{\rm gauge} + I^{(2)}_{\rm matter} .
\end{align*}
Then, $\fS = S_{\rm BV} + I^{(1)} + I^{(2)}$ satisfies the classical master equation
\begin{equation}\label{CMEcouple}
\d_{\rm Lie} \fS + \frac{1}{2} \{\fS, \fS\} = 0 .
\end{equation}
\end{thm}

We have already seen that $S_{\rm BV}$ is consistent, so satisifies the ordinary classical master equation $\{S_{\rm BV}, S_{\rm BV}\}$. 
We decompose the remaining terms in the classical master equation \ref{CMEcouple} into components based on the cardinality of the number of odd supersymmetry inputs, many of which have already been studied in Sections \ref{sec: vector} and \ref{sec: hyper}. 
The new equations we must verify are the following:
\begin{equation}\label{CMEhyper2}
\begin{array}{rrrrrr}
\{S_{\rm BV} , I^{(1)}_{\rm couple}\} + \{I_{\rm couple}, I^{(1)}_{\rm gauge} + I^{(1)}_{\rm matter}\} + \{I_{\rm couple}, I^{(1)}_{\rm couple}\} & = & 0 \\ 
\d_{CE} I^{(1)}_{\rm couple} + \{I^{(1)}_{\rm gauge} + I^{(1)}_{\rm matter}, I^{(1)}_{\rm couple}\} + \frac{1}{2} \{I_{\rm couple}^{(1)}, I_{\rm couple}^{(1)}\} & = & 0 \\
\{I_{\rm couple}^{(1)}, I^{(2)}\} & =& 0 \\
%\{I^{(2)}, I^{(2)}\} & =& 0
\end{array}
\end{equation}

\begin{lem} One has
\[
\{S_{\rm BV} , I^{(1)}_{\rm couple}\} + \{I_{\rm couple}, I^{(1)}_{\rm gauge} + I^{(1)}_{\rm matter}\} + \{I_{\rm couple}, I^{(1)}_{\rm couple}\}  =  0 
\]
\end{lem}
\begin{proof}
First, we compute
\begin{align*}
\{S_{\rm BV} , I^{(1)}_{\rm couple}\} = & g (\rho([A, \phi]) Q, \sd \dd \psi) + g (\psi^*, \rho([\d_A c, \phi]) Q) + g (\sd \dd_A \lambda, \mu(\phi) Q) + g (\lambda^*, [c, \mu(\phi) Q]) 
\end{align*}
Next,
\begin{align*}
\{I_{\rm couple} , I^{(1)}_{\rm gauge} + I^{(1)}_{\rm matter}\} = & - g (\psi, \rho(\Gamma(Q, \lambda)) \psi] - g (\d_A \phi, [\Gamma(Q, \lambda), \phi]) + g (\psi, [\rho(F_A) Q, \phi]) \\ + & g (\d_A \phi, [A, (Q, \psi)]) + g (\psi, [\lambda, (Q, \psi)]) + g (\phi^*, [c, (Q,\psi)]) +  g (\psi, (Q  , \psi) \lambda) \\ + & g^2 (\mu(\phi, \phi) , \mu(\phi, (Q, \psi)) + g ([A, \psi] , \rho(\d \phi)Q) + g ([\lambda, \phi] , \rho (\d \phi) Q) + g (\psi^*, [c , \rho(\d(\phi)) Q]) . 
\end{align*}
Finally,
\begin{align*}
\{I_{\rm couple}, I^{(1)}_{\rm couple}\} = & g^2 (\rho([A,\phi]) Q, [Q,\psi]) + g^2 (\rho([A,\phi]) Q, [\lambda, \phi]) + g^2 (\psi^*, \rho([A, [c,\phi]]) Q) \\ + & g^2 (\psi, [\mu(\phi, \phi) Q, \phi]) + g^2  (\phi, (\mu(\phi, \phi)Q, \psi) ) + g^2 (\lambda^*, \mu([c,\phi], \phi) Q) \\ + & g^2 (\mu(\phi, \psi), \mu(\phi,\phi) Q)
\end{align*}

Here, we collect all terms depending on the fields $A, \phi, \psi$:
\begin{align*}
\end{align*} 

Here, we collect all terms depending only on $\psi$ and $\lambda$:
\[
- (\psi, \rho(\Gamma(Q, \lambda)) \psi) + (\psi, [\lambda, (Q, \psi)]) + (\psi, Q) (\lambda, \psi)
\]
This expression vanishes by Proposition \ref{prop: new3psi}.

Here, we collect all terms involving only $\phi$ and $\psi$:
\begin{align*}
& \pm  (\psi, [\mu(\phi, \phi) Q, \phi]) +  (\phi, (\mu(\phi, \phi)Q, \psi) ) \\
\pm & (\mu(\phi, \phi) , \mu(\phi, (Q, \psi)) \pm (\mu(\phi, \psi), \mu(\phi,\phi)Q)
\end{align*}
\brian{I want to say each line is zero.}

Here, we collect all terms depending only on $\lambda^*, \phi$ and $c$:
\[
- (\lambda^*, \mu([c, \phi], \phi) Q) + (\lambda^*, [c, Q \mu(\phi,\phi)]) .
\]
Since $\mu : U^{\otimes 2} \to \fg^*$ is $\fg$-invariant, this expression vanishes. 
\end{proof}


\end{document}
