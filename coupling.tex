\documentclass[10pt, oneside]{article}

\input ./combined_macros.sty

\newcommand{\Cl}{\mathrm{Cl}}
\newcommand{\Dens}{\mathrm{Dens}}

\addbibresource{Twist.bib}

\usepackage{pdflscape}
\usetikzlibrary{shapes.geometric, arrows, positioning}

\tikzstyle{s16} = [rectangle, rounded corners, minimum width=1.8cm, minimum height=1cm,text centered, draw=black,fill=red!30]
\tikzstyle{s16chiral} = [s16, dashed]
\tikzstyle{s8 } = [rectangle, rounded corners, minimum width=1.8cm, minimum height=1cm,text centered, draw=black,fill=orange!30]
\tikzstyle{s4} = [rectangle, rounded corners, minimum width=1.8cm, minimum height=1cm,text centered, draw=black,fill=yellow!30]
\tikzstyle{s2chiral} = [rectangle, dashed, rounded corners, minimum width=1.8cm, minimum height=1cm,text centered, draw=black,fill=green!30]
\tikzstyle{dimension} = [circle, text centered, text width=0.7cm, minimum height=0.7cm, draw=black]
\tikzstyle{arrow} = [thick,->,>=stealth]

\title{Coupling}
\author{Chris Elliott\and Pavel Safronov \and Brian Williams}

\date{\today}

\begin{document}

\maketitle

\subsection{Supersymmetric coupling}
\subsubsection{The 6d vector multiplet coupled to the hypermultiplet}

\brian{Some conventions
\begin{itemize}
\item For $x \in \fg$ and $a \in U$ I will write $[x, a]$ for the action of $x$ on $a$. 
\end{itemize}
}

We choose a quadratic moment map $\mu : U^{\otimes 2} \to \fg^*$ which satisfying
\[
(x, \mu(a, b))_{\fg} = ([x, a], b)_U
\]
for all $x \in \fg$, and $a,b \in U$.

Let $S_{\rm gauge}$ and $S_{\rm matter}$ be the BV actions for the $6d$ $\cN=(1,0)$ vector and hyper multiplets, respectively. 
The coupled theory has action $S_{\rm gauge} + S_{\rm matter} + I_{\rm couple}$ where
\begin{align*}
I_{\rm couple} = & g (A, \mu \otimes \Gamma (\psi, \psi))_\fg - g (A, \mu(\d \phi, \phi)) - g (A, \mu(\phi, [A, \phi]))_\fg \\  - & 
%g (\psi, [\lambda, \phi])_U + g (\phi, (\lambda, \psi))_U + 
g (\lambda, \mu(\phi, \psi))_\fg \\ + & g (\psi^*, [c, \psi])_U + g (\phi^*, [c, \phi])_U \\ + & g^2 \left(\mu(\phi, \phi), \mu(\phi, \phi) \right)_\fg .
\end{align*}

%\begin{align*}
%I_{\rm couple} = & - g (A, \mu \otimes \Gamma (\psi, \psi)_\fg - g (A, \mu(\d \phi, \phi)) - g (A, \mu(\phi, [A, \phi]))_\fg \\ + & 
%%g (\psi, [\lambda, \phi])_U + g (\phi, (\lambda, \psi))_U + 
%g (\lambda, \mu(\phi, \psi))_\fg \\ + & g (c, \mu(\psi^*, \psi])_\fg + g (c, \mu(\phi^*, \phi))_\fg \\ + & g^2 \left(\mu(\phi, \phi), \mu(\phi, \phi) \right)_\fg .
%\end{align*}


The action of supersymmetry on the pure vector multiplet was constructed in Section \ref{sec: vector} and was shown to be encoded by linear and quadratic functionals that we denote $I^{(1)}_{\rm gauge}$ and $I^{(2)}_{\rm gauge}$. 
The action of supersymmetry on the pure hyper multiplet was constructed in Section \ref{sec: hyper} and was shown to be encoded by linear and quadratic functionals that we denote $I^{(1)}_{\rm matter}$ and $I^{(2)}_{\rm matter}$. 
Additionally, in the coupled theory, there are the following new terms in the action of supersymmetry:
\begin{align*}
I^{(1)}_{\rm couple} (Q) & = \int g (\psi^*, \rho([A, \phi]) Q)_U + \frac{1}{2} g (\lambda^*, Q \mu(\phi))_\fg \\
I_{\rm couple}^{(2)} (Q_1 , Q_2) & = 0 
\end{align*}
\brian{It appears there are no additional $L_\infty$ corrections}

\begin{thm}
Let $S_{\rm BV} = S_{\rm gauge} + S_{\rm matter} + I_{\rm couple}$ be the full BV action for the coupled vector to hyper multiplet and let
\begin{align*}
I^{(1)} & = I^{(1)}_{\rm gauge} + I^{(1)}_{\rm matter} + I^{(1)}_{\rm couple} \\
I^{(2)} & = I^{(2)}_{\rm gauge} + I^{(2)}_{\rm matter} .
\end{align*}
Then, $\fS = S_{\rm BV} + I^{(1)} + I^{(2)}$ satisfies the classical master equation
\begin{equation}\label{CMEcouple}
\d_{\rm Lie} \fS + \frac{1}{2} \{\fS, \fS\} = 0 .
\end{equation}
\end{thm}

We have already seen that $S_{\rm BV}$ is consistent, so satisifies the ordinary classical master equation $\{S_{\rm BV}, S_{\rm BV}\}$. 
We decompose the remaining terms in the classical master equation \ref{CMEcouple} into components based on the cardinality of the number of odd supersymmetry inputs, many of which have already been studied in Sections \ref{sec: vector} and \ref{sec: hyper}. 
The new equations we must verify are the following:
\begin{equation}\label{CMEcouple}
\begin{array}{rrrrrr}
\{S_{\rm BV} , I^{(1)}_{\rm couple}\} + \{I_{\rm couple}, I^{(1)}_{\rm gauge} + I^{(1)}_{\rm matter}\} + \{I_{\rm couple}, I^{(1)}_{\rm couple}\} & = & 0 \\ 
\{I_{\rm couple}, I^{(2)}_{\rm gauge} + I^{(2)}_{\rm matter}\} + \d_{CE} I^{(1)}_{\rm couple} + \{I^{(1)}_{\rm gauge} + I^{(1)}_{\rm matter}, I^{(1)}_{\rm couple}\} + \frac{1}{2} \{I_{\rm couple}^{(1)}, I_{\rm couple}^{(1)}\} & = & 0 \\
\{I_{\rm couple}^{(1)}, I^{(2)}\} & =& 0 \\
%\{I^{(2)}, I^{(2)}\} & =& 0
\end{array}
\end{equation}

The third equation is zero by type reasons.
\begin{lem} 
We have $\{I_{\rm couple}^{(1)}, I^{(2)}\} = 0$. 
\end{lem}
\begin{proof}
This follows from the simple observation that $I^{(2)}$ only involves the antifields $\lambda^*$, $\psi^*$ whereas $I^{(1)}_{\rm couple}$ only involves the fields $A$ and $\phi$.
So by type reasons, all terms in the bracket are zero.
\end{proof}

This leaves us to verify the first two equations of (\ref{CMEcouple}). 
We begin with the first equation. 

\begin{lem} One has
\[
\{S_{\rm BV} , I^{(1)}_{\rm couple}\} + \{I_{\rm couple}, I^{(1)}_{\rm gauge} + I^{(1)}_{\rm matter}\} + \{I_{\rm couple}, I^{(1)}_{\rm couple}\}  =  0 
\]
\end{lem}
\begin{proof}
First, we compute
\begin{align*}
\{S_{\rm BV} , I^{(1)}_{\rm couple}\} = & g (\rho([A, \phi]) Q, \sd \dd \psi) + g (\psi^*, \rho([\d_A c, \phi]) Q) + \frac{1}{2} g (\sd \dd_A \lambda, \mu(\phi, \phi) Q) + g (\lambda^*, [c, \mu(\phi) Q]) 
\end{align*}
Next,
\begin{align*}
\{I_{\rm couple} , I^{(1)}_{\rm gauge} + I^{(1)}_{\rm matter}\} = & - g (\Gamma(Q, \lambda), (\mu \otimes \Gamma)(\psi,\psi)) - g (\Gamma(Q, \lambda), \mu(\phi, \d_A \phi)) + g (\rho(F_A)Q , \mu(\phi, \psi)) \\ + & g (A, \mu((Q, \psi), \d_A \phi)) + g (A, (\mu \otimes \Gamma)(\psi, \rho(\d \phi) Q)) + g (\lambda, \mu((Q, \psi), \psi)) + g (\lambda, \mu(\phi, \rho(\d \phi)  Q)) \\ + & g (\phi^*, [c, (Q,\psi)])  + g (\psi^*, \rho( [c ,\d \phi ]) Q) + g^2 (\mu(\phi, \phi) , \mu(\phi, (Q, \psi)) . 
\end{align*}
Finally,
\begin{align*}
\{I_{\rm couple}, I^{(1)}_{\rm couple}\} = & g^2 (A, (\mu \otimes \Gamma)(\psi, \rho([A, \phi]) Q)) + g^2 (\lambda, \mu(\phi, \rho([A, \phi]) Q)) + g^2 (\mu(\phi, \phi)Q , \mu(\phi, \psi)) \\
+ &g^2 (\psi^*, \rho([A, [c,\phi]]) Q) + g^2 (\psi^*, [c, \rho([A,\phi])Q]) + g^2 (\lambda^*, \mu([c,\phi], \phi) Q)  
\end{align*}
%& g^2 (\rho([A,\phi]) Q, [Q,\psi]) + g^2 (\rho([A,\phi]) Q, [\lambda, \phi]) + g^2 (\psi^*, \rho([A, [c,\phi]]) Q) \\ + & g^2 (\mu(\phi,\phi)Q , \mu(\phi,\psi)) + g^2 (\lambda^*, \mu([c,\phi], \phi) Q) \\ + & g^2 (\mu(\phi, \psi), \mu(\phi,\phi) Q)

First, we collect all terms involving only $A, \phi$ and $\psi$.
This is obtained by extracting the $A$-dependent terms in the expression
\[
\pm (\d_A(Q, \psi), \d_A \psi) \pm (\rho(\d_A\phi)Q, \sd \dd_A \psi) .
\]
The proof that this expression vanishes is identical to the proof of Lemma \ref{lem: hyper1}. 

Here, we collect all terms involving only $\psi$ and $\lambda$:
\[
- (\psi, \rho(\Gamma(Q, \lambda)) \psi) + (\lambda, \mu((Q,\psi), \psi))
\]
By Proposition \ref{prop: new3psi} we have
\[
(\psi, \rho(\Gamma(Q, \lambda)) \psi) = (\psi, [\lambda, (Q, \psi)] ) .
\]
So, by the moment map condition the total expression involving $\psi, \lambda$ vanishes. 

Here, we collect all terms involving only $\lambda$ and $\phi$:
\[
(\lambda, \mu( \phi, \rho(\d \phi)Q)) - ([\Gamma(\lambda, Q), \phi], \d \phi). 
\]
By the moment map condition the first term is $([\lambda, \phi], \rho(\d \phi) Q)$. 
So, by the Clifford relation, these terms cancel. 

Here, we collect all terms involving only $A, \lambda$ and $\phi$:

Here, we collect all terms involving only $\phi$ and $\psi$:
\begin{align*}
\pm & (\mu(\phi, \phi) , \mu(\phi, (Q, \psi)) \pm (\mu(\phi, \psi), \mu(\phi,\phi)Q)
\end{align*}
Since $(\mu(\phi, \psi), \mu(\phi,\phi)Q) = (\mu(\phi, \phi) , \mu(\phi, (Q, \psi))$ this expression is identically zero. 
\brian{fix signs}

Here, we collect all terms involving only $\phi^*, c$ and $\psi$:
\[
(\phi^*, [c, (Q, \psi)]) \pm (\phi^*, (Q, [c,\psi])) 
\]
which clearly vanishes. 

Here, we collect all terms involving only $\psi^*$, $c$, and $\phi$:
\[
(\psi^*, [c, \rho(\d \phi) Q]) \pm (\psi^*, \rho( [c ,\d \phi ]) Q) 
\]
which clearly vanishes. 

Here, we collect all terms involving only $\psi^*$, $c$, $A$, and $\phi$:
\[
\pm (\psi^*, [c, \rho([A,\phi]) Q]) \pm (\psi^*, \rho([A,[c,\phi]]) Q) 
\]
which clearly vanishes. 
%Thus, the sum of the last two terms cancels with the first term. 

Here, we collect all terms depending only on $\lambda^*, \phi$ and $c$:
\[
- (\lambda^*, \mu([c, \phi], \phi) Q) + (\lambda^*, [c, Q \mu(\phi,\phi)]) .
\]
Since the moment map $\mu$ is $\fg$-invariant, this expression vanishes. 
\end{proof}

Finally, we verify the second equation in (\ref{CMEcouple}). 

\begin{lem}
One has
\[
\{I_{\rm couple}, I^{(2)}_{\rm gauge} + I^{(2)}_{\rm matter}\} + \d_{CE} I^{(1)}_{\rm couple} + \{I^{(1)}_{\rm gauge} + I^{(1)}_{\rm matter}, I^{(1)}_{\rm couple}\} + \frac{1}{2} \{I_{\rm couple}^{(1)}, I_{\rm couple}^{(1)}\} =  0.
\]
\end{lem}
\begin{proof}
The second and fourth terms are zero by type reasons.
Thus, we must show
\begin{equation}\label{CMEcouple2}
\{I_{\rm couple}, I^{(2)}_{\rm gauge} + I^{(2)}_{\rm matter}\} +  \{I^{(1)}_{\rm gauge} + I^{(1)}_{\rm matter}, I^{(1)}_{\rm couple}\} = 0
\end{equation}

We begin by expanding $\{I_{\rm couple}, I^{(2)}_{\rm gauge}\}$:
\begin{align*}
\{I_{\rm couple}, I^{(2)}_{\rm gauge}\} (Q_1,Q_2) = & - \frac{1}{2} (\rho(\Gamma(Q_1,Q_2)) \lambda^*, \mu(\phi,\psi)) - \frac{1}{2} (Q_1, \mu(\phi, \psi)) (Q_2, \lambda^*) - \frac{1}{2} (Q_1, \lambda^*)(Q_2 \mu(\phi,\psi)) \\ + & (\psi^*, [i_{\Gamma(Q_1,Q_2)} A , \psi]) + (\phi^*, [i_{\Gamma(Q_1,Q_2)} A , \phi])
\end{align*}

Next, we expand $\{I_{\rm couple}, I^{(2)}_{\rm matter}\}$:
\begin{align*}
\{I_{\rm couple}, I^{(2)}_{\rm matter}\} (Q_1,Q_2) = & \frac{1}{2} (\rho(\Gamma(Q_1,Q_2)) \psi^*, [A, \psi]) - \frac{1}{2}(\lambda, \mu(\phi, \rho(\Gamma(Q_1,Q_2)) \psi^*)) \\ & + (\Gamma(Q_1, Q_2), \Gamma([c, \psi^*], \psi^*)) + (\Gamma(Q_1, Q_2), \Gamma(\psi^*, [c, \psi^*])).
\end{align*} 
The last line vanishes by the $\fg$-invariance of the symplectic pairing for $U$. 

Expanding $\{I^{(1)}_{\rm gauge}, I^{(1)}_{\rm couple}\}$:
\begin{align*}
\{I^{(1)}_{\rm gauge}, I^{(1)}_{\rm couple}\}(Q_1,Q_2) = & - (\psi^*, \rho([\Gamma(Q_1, \lambda), \phi]) Q_2) -(\psi^*, \rho([\Gamma(Q_2, \lambda), \phi]) Q_1) \\ + & (A^*, \Gamma(Q_1, \mu(\phi, \phi) Q_2)) - (A^*, \Gamma(Q_2, \mu(\phi, \phi) Q_1)) . 
\end{align*}
Notice the term proportional to $A^*$ vanishes. 

\begin{align*}
\{I^{(1)}_{\rm matter}, I^{(1)}_{\rm couple}\}  (Q_1,Q_2) = & \pm (\phi^* Q_1 , \rho([A,\phi]) Q_2) \pm (\phi^* Q_2 , \rho([A,\phi]) Q_1) \\ -  &  (\psi^*, \rho([A, (Q_1, \psi)]) Q_2) - (\psi^*, \rho([A, (Q_2, \psi)]) Q_1) \\ - & (\lambda^*, \mu((Q_1, \psi), \phi) Q_2) -  (\lambda^*, \mu((Q_2, \psi), \phi) Q_1) 
\end{align*}

We collect terms involving $\phi, A$ and $\phi^*$:
\[
[i_{\Gamma(Q_1, Q_2)} A, \phi] \pm (Q_1, \rho([A,\phi])Q_2) \pm (Q_2, \rho([A,\phi])Q_1)
\]
By Proposition \ref{prop:cliffordactionproperties} the last two terms cancel with the first term. \brian{Pavel, what is the version of that identity we need to use here, that proposition isn't quite what we need. This is the only sign I'm unsure of in this proposition.}

Next, we collect terms involving $\phi, \psi$ and $\lambda^*$. 
This term is obtained by taking the $\fg$-pairing of $\mu(\phi, \psi)$ with
\[
-\frac{1}{2} \rho(\Gamma(Q_1,Q_2)) \lambda^* + \frac{1}{2} (Q_2, \lambda^*)Q_1 + \frac{1}{2} (Q_1, \lambda^*) Q_2
\]
which vanishes by Proposition \ref{prop: new3psi}.

Next, we collect terms involving $\lambda, \phi, \psi^*$.
This term can be written as pairing the antifield $\psi^*$ with the field
\[
- \frac{1}{2} \rho(\Gamma(Q_1,Q_2)) [\lambda, \phi] - \frac{1}{2} \rho([\Gamma(Q_1,\lambda), \phi]) Q_2  - \frac{1}{2} \rho([\Gamma(Q_2,\lambda), \phi]) Q_1. 
\]
This whole expression is equal to applying the $\fg$-valued field 
\[
- \frac{1}{2} \rho(\Gamma(Q_1,Q_2)) \lambda - \frac{1}{2} \rho(\Gamma(Q_1,\lambda)) Q_2  - \frac{1}{2} \rho(\Gamma(Q_2,\lambda)) Q_1
\]
to the field $\phi$ by the representation. 
This expression vanished by Proposition \ref{prop:3psi}.

Finally, we collect terms involving $\psi^*,A$ and $\psi$.
This can be written as the pairing of the antifield $\psi^*$ with the field
\[
\frac{1}{2} [A, \rho(\Gamma(Q_1,Q_2)) \psi] -\frac{1}{2} \rho([A, (Q_1, \psi)]) Q_2 - \frac{1}{2} \rho([A, (Q_2, \psi)]) Q_1 .
\] 
This expression is equal to applying the $\fg$-valued field $A$ to the field
\[
\frac{1}{2} \rho(\Gamma(Q_1,Q_2)) \psi -\frac{1}{2} (Q_1, \psi) Q_2 -\frac{1}{2} (Q_2, \psi) Q_1
\]
which vanishes by Proposition \ref{prop: new3psi}.


\end{proof}



\section{The holomorphic twist}

The holomorphic twist of the 6d $\cN=(1,0)$ vector multiplet coupled to the hyper multiplet can be described in the BV formalism as a ($\ZZ/2$-graded) generalized Chern-Simons theory on any Calabi-Yau $3$-fold $X$. 

Suppose the vector multiplet takes values in a Lie algebra $\fg$ and the hyper multiplet is valued in a symplectic $\fg$-representation $(U,\omega_U)$. 
Define the following graded Lie algebra
\[
\begin{array}{ccccccccc}
&&& \ul{0}  & & \ul{1} & & \ul{2} \\
\fg_U & = && \fg & & U & & \fg^*
\end{array}
\]
where the Lie brackets are given by:
\begin{itemize}
\item the usual Lie bracket $\fg \otimes \fg \to \fg$;
\item the representation $\fg \otimes U[-1] \to U[-1]$;
\item the coadjoint action $\fg \otimes \fg^*[-2] \to \fg^*[-2]$;
\item the moment map $\mu_k : (U[-1])^{\otimes k} \to \fg^*[-2]$ \brian{could be $L_\infty$-maps, $k=2$ is the ordinary situation.}
\end{itemize}

The Lie algebra has a natural nondegenerate invariant pairing $\omega$ of degree $(+2)$ defined by combining the symplectic pairing $\omega_U$ on $U$ together with the natural dual pairing between $\fg$ and its dual. 

Given $\fg_U$ we can consider generalized Chern-Simons theory on $(X, \Omega^{0,\bu})$ with values in $\fg_U$.
Note that this is only a $\ZZ/2$-graded theory in the BV formalism, since $\fg_U$ has a pairing of degree $+2$ and $\dim_\CC (X) = 3$. 
The underlying space of BV fields is 
\[
\Omega^{0,\bu}_X \otimes \fg_U [1]
\]
and the action functional is of Chern-Simons type $S(\alpha) = \int_X (\alpha, \dbar \alpha) + \sum_{k\geq 2}\frac{1}{(k+1)!} \int_{X} \Omega_X\wedge \langle \alpha \wedge \ell_k(\alpha, \dots, \alpha)\rangle.$

For the specific graded Lie algebra $\fg_U$ we can decompose $\alpha = A + \varphi + B$ where $A$ is $\fg$-valued, $\varphi$ is $U$-valued, and $B$ is $\fg^*$-valued.
The action then takes the form
\[
S(A + \varphi + B) = \int_X \left(B, \dbar A + \frac{1}{2} [A,A] \right)_\fg + \omega_U \left(\varphi, \dbar \varphi + [A, \varphi] \right) + \sum_{k\geq 2} (A, \mu(\varphi, \ldots, \varphi))_\fg .
\]

\end{document}
